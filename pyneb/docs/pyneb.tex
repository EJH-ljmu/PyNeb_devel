%                                                                 aa.dem
% AA vers. 8.1, LaTeX class for Astronomy & Astrophysics
% demonstration file
%                                                       (c) EDP Sciences
%-----------------------------------------------------------------------
%
%\documentclass[referee]{aa} % for a referee version
%\documentclass[onecolumn]{aa} % for a paper on 1 column  
%\documentclass[longauth]{aa} % for the long lists of affiliations 
%\documentclass[rnote]{aa} % for the research notes
%\documentclass[letter]{aa} % for the letters 
%
\documentclass{aa}  

%
%\usepackage{cite}
\usepackage{graphicx}
%%%%%%%%%%%%%%%%%%%%%%%%%%%%%%%%%%%%%%%%
\usepackage{natbib,txfonts}
%%%%%%%%%%%%%%%%%%%%%%%%%%%%%%%%%%%%%%%%
%\usepackage[options]{hyperref}
% To add links in your PDF file, use the package "hyperref"
% with options according to your LaTeX or PDFLaTeX drivers.
%
\usepackage{twoopt}
\usepackage[breaklinks=true]{hyperref} %% to avoid \citeads line fills
\bibpunct{(}{)}{;}{a}{}{,}             %% natbib format for A&A and ApJ
\makeatletter
  \newcommandtwoopt{\citeads}[3][][]{\href{http://adsabs.harvard.edu/abs/#3}%
    {\def\hyper@linkstart##1##2{}%
     \let\hyper@linkend\@empty\citealp[#1][#2]{#3}}}
  \newcommandtwoopt{\citepads}[3][][]{\href{http://adsabs.harvard.edu/abs/#3}%
    {\def\hyper@linkstart##1##2{}%
     \let\hyper@linkend\@empty\citep[#1][#2]{#3}}}
  \newcommandtwoopt{\citetads}[3][][]{\href{http://adsabs.harvard.edu/abs/#3}%
    {\def\hyper@linkstart##1##2{}%
     \let\hyper@linkend\@empty\citet[#1][#2]{#3}}}
  \newcommandtwoopt{\citeyearads}[3][][]%
    {\href{http://adsabs.harvard.edu/abs/#3}
    {\def\hyper@linkstart##1##2{}%
     \let\hyper@linkend\@empty\citeyear[#1][#2]{#3}}}
\makeatother
\newcommand{\hiir}{H{\sc~ii} regions}
\newcommand{\sii}{S{\sc~ii}}
\newcommand{\oiii}{O{\sc~ii}}

\begin{document} 


   \title{PyNeb: a new tool for the analysis of emission lines}

   \subtitle{I. Code description and validation of results}

   \author{A. AAA
          \inst{1}
          B. BBB
          \inst{2}
          \and
          C. CCC\inst{2}\fnmsep\thanks{Thank example}
          }

   \institute{Institute for Astronomy (IfA), University of Vienna,
              T\"urkenschanzstrasse 17, A-1180 Vienna\\
              \email{wuchterl@amok.ast.univie.ac.at}
         \and
             University of Alexandria, Department of Geography, ...\\
             \email{c.ptolemy@hipparch.uheaven.space}
             \thanks{The university of heaven temporarily does not
                     accept e-mails}
             }

   \date{}

% \abstract{}{}{}{}{} 
% 5 {} token are mandatory
 
  \abstract
  % context heading (optional)
  % {} leave it empty if necessary  
   {}
  % aims heading (mandatory)
   {}
  % methods heading (mandatory)
   {T}
  % results heading (mandatory)
   {}
  % conclusions heading (optional), leave it empty if necessary 
   {}

   \keywords{methods: numerical --
                atomic data --
                HII regions --
                planetary nebulae: general --
                ISM: abundances 
               }

   \maketitle
%
%________________________________________________________________

\section{Introduction}

\nocite{*}
   This paper describes PyNeb, a new package for the analysis of emission lines in the spectra of ionized nebulae. 
   %such as H~II regions and planetary nebulae (PNe). 
   The intensity of emission lines carries valuable information on the physical conditions and the chemical abundances of these objects. The physics of line formation in the relevant regime has been described in several seminal papers published throughout the last century 
\cite{1937ApJ....85..330M,1938ApJ....88...52B,1938ApJ....88..422B,2006agna.book.....O,1989agna.book.....O} 
   and can be summarized in a handful of equations. These are conceptually simple, but cannot be solved analytically and a numerical code is necessary to do the task. Furthermore, they depend on a number of parameters -the atomic data- which are only approximately known and whose determination improves by the year. A tool to do the task must therefore be prepared to include such advances. 
   
  In photoionized objects, only the lowest-lying atomic levels are populated: the atom can be represented as a simple n-level system, in which the population of each level is governed by statistical equations of collisional excitation and  de-excitation and spontaneous emission. De-excitation is mostly inefficient due to the low density values, so a great part of the energy inverted in collisions is reemitted as photons. Optical depths are also very low for most lines, so that these photons freely escapes and there is no need of a detailed treatment of radiation transfer. 

   PyNeb is the last in a lineage of tool dedicated to the analysis of emission lines, which includes FIVEL and nebular. FIVEL \cite{1987JRASC..81..195D} is a fortran program that solved the five-level atom model to determine level populations and line emissivities, which were used to apply simple temperature and density diagnostics. nebular \cite{1995PASP..107..896S,1998ASPC..145..192S} is an IRAF package that was initially based on FIVEL but extended its functionality to other tasks, such as the determination of ionic and elemental abundances, and also provided a simple model of nebular structure. In its almost twenty years of life, {\bf nebular} has enjoyed -and still enjoys- wide use in the research community for the analysis of physical conditions and chemical abundances in a variety of astrophysical contexts, including active galaxies, H II regions, and PNe. It is also widely used as a starting point by people who model gaseous nebulae with photoionization code.

In {\bf nebular}, an atom is represented as an n-level system. As described by \cite{1998ASPC..145..192S}, the software is data-driven, so that the input atomic data are not hardwired and can easily be changed without re-compiling the source code: the energy levels, statistical weights, transition probabilities, and collision strengths are all read from FITS files at run-time, allowing easy comparisons among different data sources. In addition, for a given ion, users may employ any recognized line ratio by creating an expression for the desired transitions. There is also a choice of interstellar extinction laws to deredden data.

The decision to export {\bf nebular} functionality to a modern programming environment, python, was motivated by a desire of enhancing portability, allowing the implementation of GUI and web interfaces, and making it simple for developers or advanced users to extend the original functionality of the code. PyNeb provides the same functionality of {\bf nebular} plus new analysis features, a vast range of visualization tools, and full access to the intermediate quantities of the calculation. We will describe the most relevant of these features in the following sections.

%
%__________________________________________________________________

\section{Overview of the code}

\subsection{Underlying physics}\label{sec:physics}

The spectra emitted by photoionized objects in low-density environments are dominated by permitted lines of H, He and He ions and forbidden lines of ions of abundant elements (such as O, N and Ne). The former mainly form in the downward cascade following recombination of H$^+$, He$^+$ and He$^{++}$; the main mechanism forming the latter is the radiative deexcitation following collisional excitation from the ground state.
The main task performed by PyNeb boils down to computing the line emissivities of heavy elements with respect to H$^+$. In the following, we will describe how both types of emissivity are computed.

\subsubsection{Emissivity of collisional excited lines}

The behavior of the ions of heavy elements is modeled by the PyNeb's class atom, which belong to the core of the package.
The class assumes that the atom can be described as a simple n-system level. In such a model, 
the equilibrium level populations are described by the following set of equations:

\begin{eqnarray*}
    \sum_{j\ne i} n_e n_j q_{ji} + \sum_{j>i}  A_{ji} = \sum_{j\ne i} n_e n_i q_{ij} + \sum_{j<1} A_{ij}, & i = 1, ..., n_{\rm max}    \label{eq:pop_balance}\\
    \sum_i n_i = N({\rm X)}) \label{eq:pop_closure} & \\
\end{eqnarray*}  
   
\noindent where $n_e$ is the electron density, 
$N({\rm X})$ is the total density of ion X,
$n_k$ is the density of X atoms having an electron on level $k$, 
$n_e q_{hk}$ is the rate of collisional excitation (if $h<k$) or de-excitation (if $h>k$), and $A_{hk}$ is the transition probability from level $h$ to $k$ ($h>k$).
The densities are measured in cm$^{-3}$ and the transition probabilities in s$^{-1}$.
The rates of collisional excitation and deexcitation are derived from the effective collision strengths $\Upsilon$ through the equations:

\begin{eqnarray*}
 q_{j, k} &=& \frac{8.629 \cdot 10^{-6}}{g_j}\frac{\Upsilon(k, j;T_e)}{T_e^{1/2}}\\
\end{eqnarray*}
and
\begin{eqnarray*}
q_{k, j} &= &\frac{g_j}{g_k}q_{j, k}\; e^{-\Delta E(k, j)/k_bT_e}\\
 \label{eq:coll_rate}
\end{eqnarray*}

\noindent \cite{OF00}, where $j>k$, $g_{j,k}$ are the statistical weights of levels j and k, respectively, and $e^{-\Delta E(k, j)/k_bT_e}$ is the Boltzmann factor.
The $\Upsilon$s are the result of atomic physics computations. 

Equations~\ref{eq:pop_balance} and  \ref{eq:pop_closure} represent a set of coupled equations in the unknowns $n_i$, which, for any given $N_e, T_e$ combination, can be solved for the relative level populations $n_i/n_{tot}$.
Once the populations are known, the line emissivities are easily computed as:

\begin{eqnarray}
 \epsilon_{j, k} &=& n_j A_{j, k} h\nu_{j, k} \\
 &=& N(X) \frac{n_j}{N(X)} A_{j, k} h\nu_{j, k}.
 \label{eq:emissivity}
\end{eqnarray}

Since the $n_j$ depend on $N_e$ and $T_e$, so does the ratio of lines of a same ion will also do. On the other hand, it will not depend on the
ion abundance $N($X$)$, which cancels out.

\subsubsection{Emissivity of H and He recombination lines} 

\cite{1987MNRAS.224..801H} and \cite{1995MNRAS.272...41S} performed detailed computations of the recombination cascade of H and hydrogenoid ions. Those for H$^+$ and He$^{++}$ are stored in tabular form
for a wide array of temperature and density values. The emissivity of a given line is computed by PyNeb by interpolation in the tables.

%; e.g., 
% getRecEmissivity(1.e4, 1.e3, 3, 2, atom='H1') / getRecEmissivity(1e4, 1e3, 4, 2, atom='H1')  
% returns the H$\alpha$/H$\beta$ ratio at $T_e=10000K$ and $N_e=1000$ cm$^{-3}$.

\subsubsection{Plasma diagnostics}

The result of Eq. \label{eq:stat} depends on the assumed density and temperature. For a given set of physical conditions ($N_e$, $T_e$), the relative populations
can be determined (Section~\ref{sec:atom}; the emissivity $\epsilon(\lambda)$ in a given transition $i -> j$ with $\lambda=...$ is given by:

 \begin{equation}
  \epsilon(\lambda)=n_iA_{ij}hc/\lambda
 \end{equation}
 
\subsubsection{Ionic abundance determination}

\subsubsection{ICFs and elemental abundance determinations}

If all the expected ions of a given element are observed, the elemental abundances are determined as simple sums of the ionic abundances:
\begin{equation}
X/H = \sum_i X_i/H^+
\end{equation}

In most cases, however, not all relevant ions are observed and a correcting factor, the $icf$, must be applied to the known ionic abundances.
The icf (ionization correction factors) are expressions used correct the abundance of observed ions for unseen ions to get the total elemental abundance of a given element:

X(elem) = X(ion) * icf,

where X(ions) may be the abundance of one single ion or the sum of several ions of the element. These expressions are generally obtained empirically or semiempirically, based on the results of photoionization models or a comparison between the ionization potentials of different ions. Most expressions have been devised for a specific kind of object (e.g., PNe, HII regions, etc) and should not be applied to objects of a different kind.

\subsection{Advantages of PyNeb over nebular}

There are many reasons why you might want to use PyNeb:

\begin{enumerate}
\item{} A large sample of atomic data is provided with the code.
\item{} Tools to explore the atomic data used and available are provided. 
\item{} The atomic data are easier to change.
\item{} Improved graphic capabilities
\item{} Calculation of elemental abundances with icfs
\item{} Full access to intermediate quantities
\item{} Simultaneous determination of temperature and density from pair of line ratios
\item{} Recombination intensities for some important ions added
\item{} Some ions added to the inventory of collisional lines
\item{} Any diagnostic can be used (not only predetermined ones)
\item{} Insight in emissivity properties thanks to emissivity maps
\item{} Auxiliary tools, such as Grotrian plots, tools to generate fits files from ascii data, visualization of atomic data, etc.
\item{} Modular structure: easy to embed in scripts 
\item{} Stand-alone code: doesn't require IRAF to be installed.
\end{enumerate}

\subsection{Structure of the code}

{\bf Hay que decidir si hacerlo en terminos de clases, modulos o categorias. Yo opto por eso ultimo.Una vez decidido, se adapta la figura en consecuencia.}
Fig.~\ref{fig:sketch} represents PyNeb's structure.
The main module (called pynebcore.py) is where most physics resides;
in particular, it is where the Atom class is defined. Such class is used to build Atom objects  according to the rules described above and using the data contained in the atomic database.

A further class, EmisGrid, comptes stacked emissivity grids as a function of T$T_e$, $N_e$ or all the transitions of a selected ion.

The class observation takes care of observational data: it has functions to read and sort them and correct them for extinction.

The diagnostics class is where the standard inventory of diagnostics is added and diagnostics are handled to produce plots or numerical results.



%______________________________________________ Gamma_1 (lg rho, lg e)
   \begin{figure*}
   \centering
   %%%\includegraphics{empty.eps}
   %%%\includegraphics{empty.eps}
  \includegraphics{pyneb_sketch.ps}
   \caption{Main PyNeb's modules.}
              \label{fig:sketch}%
    \end{figure*}
%

\section{Using the code}

\subsection{Computational aspects}

\subsubsection{Requirements}

As of February 2013, the code requires the following packages and libraries to function:
\begin{enumerate}
\item{} python v. 2.7x (\it{not} v. 3.x, which is a different and incompatible branch of python)
\item{} numpy \cite{numpy} v. 1.6 or later
\item{} matplotlib 
\item{} scipy
\end{enumerate}

The first two are required for the functioning of the code. matplotlib is necessary to use the graphic modules of PyNeb, but the code is prepared
to skip them if the library is not installed; in such cases, a warning that the module is not available will be issued but the code will not stop.

Finally, scipy is currently required only for the H emissivity to work (see Section\label{hb}).

\subsubsection{Retrieving and installing the code}

The code can be currently downloaded and installed with pip:

\begin{equation}
pip install --user --index-url http://URL_to_be_changed PyNeb
\end{equation}

Once the code is installed, the user must enter a python interface (such as python or ipython) and import the code:

\begin{equation}
import pyneb as pn
\end{equation}

(As experienced python users will know, the "as pn" bit is just needed to establish pn as the shortname.)


\subsection{The atom}
A PyNeb session starts by instantiating the Atom class with an element and a spectrum:

o3 = pn.Atom('O', 3)

This command creates an object (in the computational sense) with the attributes of the Atom class, fills it with the properties of O III, and assigns it to the arbitrary name o3. The atom properties can then be simply known by exploring the object's properties. For example:

o3.name
o3.spec
o3.getEnergy
o3.getA(2, 1)
o3.getOmega(tem=10000, den=100)

This transparency with respect to the input igredients is a great asset of PyNeb over nebular.
Of course, the quantities that nebular does compute are also computed by pyneb: for example,

o3.getPopulations(tem, den)
o3.getCritDensity(tem, den)

will return the populations and critical densities at the specified physical conditions.

A few simple calculations can also be performed: e.g.

o3.getTransition(4959)

will return the O III transition closest to 4959.

\subsection{Comparison with observations}

\subsubsection{Diagnostic diagrams}


If one of $N_e$ and $T_e$ is known, the observed line ratios can then serve as a diagnostic for the other variable by comparing it with the predicted value. In PyNeb, this is accomplished with the function getTemDen, e.g.:

o3.getTemDen(0.01, den=1000., wave1=4363, wave2=5007)

will return the temperature corresponding to [O III]4363/5007 =0.01 and den=1000.
Under certain circumstances, the ratio of lines from a given ion is nearly independent on one of the variables $N_e$ and $T_e$; the result of this calculation will not depend critically on the input value of that variable ($N_e$ in the above example). Examples of this behavior are the line ratios \sii\ or \oiii, which in certain regimes are diagnostics of density and temperature, respectively \cite[see][for details]{2006agna.book.....O}. Whether a give line ratio is robust (i.e., independent of the exact value of $N_e$ assumed) or not can be judged by the slope of the $N_e$, $T_e$ locus corresponding to a given intensity ratio, as can be seen in Fig.~\ref{fig:o_iii_ratio}.

%                                     Two column figure (place early!)
%______________________________________________ Gamma_1 (lg rho, lg e)
   \begin{figure*}
   \centering
   %%%\includegraphics{empty.eps}
   %%%\includegraphics{empty.eps}
  \includegraphics{o_iii_ratio.eps}
   \caption{The variation of the [OIII] 5007+4959/4363 intensity ratio as a function of electronic density and temperature.}
              \label{fig:o_iii_ratio}%
    \end{figure*}
%
This plot has been obtained with PyNeb and shows that the OIII ratio is a robust diagnostic of $T_e$ up to until $N_e\sim 30000K$.

If two different line ratios are available (from the same or different ions), it is no longer necessary to assume one of $N_e$, $T_e$ since both variables can be derived simultaneously:

tem, den = diags.getCrossTemDen('[NII] 5755/6548', '[SII] 6731/6716', 50, 1.0)

The above expression determines the $N_e$, $T_e$ pair that simultaneously reproduces the [NII] 5755/6548=50. and [SII] 6731/6716=1.0 line ratios. Fig.~\ref{fig:crossTemDen} shows the same result in graphical form and with the observational errors included.

%                                     Two column figure (place early!)
%______________________________________________ Gamma_1 (lg rho, lg e)
   \begin{figure*}
   \centering
   %%%\includegraphics{empty.eps}
   %%%\includegraphics{empty.eps}
  \includegraphics{crossTemDen.eps}
   \caption{The variation of the [OIII] 5007+4959/4363 intensity ratio as a function of electronic density and temperature.}
              \label{fig:crossTemDen}%
    \end{figure*}
%


 
\subsubsection{Simultaneous determination of $N_e$ and $T_e$}
Since the line emissivities depend on two variables, $N_e$ and $T_e$, the observed intensity of a single line does not allow to determine a solution.
If two lines are available, however, this can be done; the relevant function is getTemDen. The algorithm works by a simple bisection procedure: an  initial guess of the temperature is performed, upon which successive iterations bring the 

\subsubsection{Simultaneous determination of $N_e$ and $T_e$ with more than two line ratios}
When more than two lines ratios are available for a given object, a unique solution does in general not exist, but the Ne, Te loci of the varios line ratios can be plotted on a graph to show evidence (or lack thereof) of convergence.
It is the user responsibility, of course, to interpret physically the results.
 
\subsubsection{Ionic abundance determination}
Once the physical conditions are known and the corresponding line emissivities can be calculated, it is straightforward to determine the ionic abundance relative to H$^+$:

\begin{equation}
X/H^+ = \frac{I(\lambda)}{I(\rm{H}\beta)} \frac{\epsilon(\rm{H}\beta)}{\epsilon(\lambda)}.
\end{equation}

This calculation is accomplished by the Pyneb function getIonAbundance:

O3.getIonAbundance(100, 1.5e4, 100., wave=5007).


\subsubsection{Elemental abundance determination}

Following the general structure defined above, in PyNeb an icf formula is identified by an element ("elem", e.g. "O"), which is the element of which the total abundance is seeked for;  the ion or ions whose abundance is corrected ("atom", e.g.  "O2+O3"); and the icf proper, which is an expression involving the abundances of other ions, assumed to be known ("icf", e.g. "1 + 0.5 * He3 / (He2 + He3)"). In addition, each icf expressions is identified by a label and holds the original reference and a brief comment specifying its intended field of application.
The label is formed by an acronym of the paper and the equation  number of the icf within the paper.

The following snippet illustrates how PyNeb manages icfs:

atom\_abun = {'O2': 0.001, 'O3': 0.002, 'Ne3': 1.2e-5}
icf.getAvailableICFs('Ne')   \# lists all the available recipes for Ne
elem\_abun = pn.getElemAbundance(atom\_abun, icf\_list=['TPP85']) \# Computes the Ne abundance with the TPP06 recipe
elem\_abun = pn.getElemAbundance()  \# performs all the possible element abundance computations given the input ionic abundance set and available icf  

The first line above defines a set of ionic abundances;
the second lists the label of all the available recipes for Ne;
in the third, a specific one is selected to compute the desired abundance;
in the fourth line, all the possible element abundance computations given the present icf set are performed (these may include icfs not suitable for the object under study).

Additionally, PyNeb provides a series of single-line commands to explore the icf collection and its source papers; e.g.:

icf.getAvailableICFs()
icf.getExpression('KH01\_4g')   \# returns the analytical expression of the icf identified by the label KH01\_4g
icf.getReference('KH01\_4g')   \# returns the bibliographic reference of the source paper
icf.getURL('KH01\_4g')   \# returns the ADS URL of the source paper

%If one wants to filter out only a certain element:
%\begin{equation}
%for item in elem_abun:
%    if (item(0) == 'Ne'): print item
%\end{equation}

PyNeb provides a large set of icfs compiled from the literature; each icf is stored together with the ref, the URL and a comment with further details. 
Table~\ref{tab:icf} lists the bibliographic sources of the icfs included in the release version of PyNeb Further expressions will be added in the future, and a special function (addICF) allows to add customized expressions to the collection.


%__________________________________________________ 
   \begin{table*}
      \caption[]{ICFs included in the release version of PyNeb}
         \label{tab:icf}
\centering
\begin{tabular}{lcc}
\hline \hline
            Source      &  Elements  & Object class \\
            \hline
             Izotov et al 2006& N, O, Ne, S, Cl, Ar, Fe& H II\\
             Kinsburgh \& Barlow 1994& C, N, O, Ne, S, Ar & PNe \\
             Kwitter \& Henry 2001& He, N, O, Ne, Cl, Ar& PNe \\
             Peimbert \& Costero 1969& Ne & HII\\
             Perez-Montero, Haegele, Contini, \& Diaz 2007& Ne, Ar & HII \\
             Rodriguez \& Rubin 2004& Fe & HII\\
             Stasinska 1978 & Ne & HII \\
             Torres-Peimbert \& Peimbert 1977& N, O, Ne & PNe \\
            \hline
         \end{tabular}
   \end{table*}
%


\subsection{Atomic data}
It is clear that the result of such calculations also depend on the assumed transition probabilities and collision strengths. These quantities are the result of 
complex atomic physics calculations and are subject to uncertainties. While for some atoms the uncertainty is relatively low, others atom are far more complex to model and the resulting data are subject to large uncertainties.In such cases, the result of nebular analysis may depend strongly on the assumed atomic data set. To address this issue, PyNeb provides a large number of atomic data determinations and a number of tools to assess the differences between them. The atomic data base and the related tools are an integral part of PyNeb's philosophy and will be described in Paper II of this series.

%__________________________________________________________________
%__________________________________________________________________



\subsection{Limitation of the code and misc cautions}
wavelengths
atomic data references
atomic data: see Paper II
bugs
H case B
Both the $A$s and the $\Upsilon$s are subject to uncertainties. The discussion of these and the description of PyNeb's atomic data base will be the object of \cite{PyNeb2}.


\begin{enumerate}
  \item{} The density is above the critical density of one of the ions: the method can still be applied, but the result is no longer density-independent. 
  \item{} Non-homogeneous physical conditions: the object under analysis has a range of physical condition regimes and the line arise from all of them. 
  \item{} Uncertainties in the atomic data: see \cite{PyNeb2}
  \item{} Cross-dependence of temperature and density: see above
  \item{} Ions form in different zones: (three zones model, etc.)
\end{enumerate}


what pyneb can't do:

\begin{enumerate}
  \item{} account for energy balance
  \item{} resolve radiation transfer
  \item{} compute a complex physical model
  \item{} replace the researcher's critical eye
\end{enumerate}

%__________________________________________________________________

\subsection{Documentation}

A necessary aspect of a widespread package is to be well documented. PyNeb currently has several sources of documentation:

\begin{enumerate}
\item{} The automatically generated documentation, which is available both in html format or in pdf format.
\item{} The code docstrings, which can be xplored through standard python or ipython commands (e.g., addICF?) 
\item{} A short, discursive introduction to PyNeb
\item{} A series of well commented scripts
\item{} The illustrating papers (Paper I and II)
\end{enumerate}

In addition, we plan to maintain an informal helpdesk active for quick questions or comments:
pyneb66@gmail.com.

\section{Code validation}


\section{Future developments and concluding remarks}

Some of the intended enhancements include:  
add new ions, including $s$-process elements;
determine abundances for He and other light ions from recombination lines;
perform the error analysis of $T_e$, $N_e$ and abundances;
add recipes for computing total elemental abundances from ionic abundances, using common or user-defined ICF formulae; and
create VO-compatible web services to provide PyNeb over the internet.

%                                     Two column figure (place early!)
%______________________________________________ Gamma_1 (lg rho, lg e)
   \begin{figure*}
   \centering
   %%%\includegraphics{empty.eps}
   %%%\includegraphics{empty.eps}
  \includegraphics{emissivity.eps}
   \caption{Emission-line diagnostic plot of a planetary nebula.}
              \label{em_line}%
    \end{figure*}
%
   PyNeb is still under development, but it already features many functionalities and visualization tools, including most original {\bf nebular} functionalities.
As an example, we show a plot of emission-line diagnostics obtained from the spectrum of a single object (Fig. 1) and the emissivity map of [O{\sc~iii}] $\lambda 5007$ as a function of temperature ($T_e$) and density ($N_e$) (Fig. 2). Among other features, the code can also generate contour maps of diagnostic line ratios as a function of $T_e$ and $N_e$; perform simultaneous determinations of $T_e$ and $N_e$ from pairs of line ratios; and easily switch among different atomic data.


%__________________________________________________________________

The scientific community is invited to use the code for their projects. We would be grateful for any feedback, query or comment. Whenever you use PyNeb for calculations that lead to a published paper, you are kindly asked to cite the code as:
<<include reference here>>

%__________________________________________________________________

\section{Ideas varias}

%Tests and validation
%Performance
%General features
%Flow chart


\begin{acknowledgements}
      PNAyA whatever, IAC sabbatical stay, NOAO grants, etc
\end{acknowledgements}

\bibliography{my_refs}{}
\bibliographystyle{plain}

%\bibitem[1998]{Sal98}
%{{Shaw}, R.~A., \etal} 1998, in ADASS VII, edited by
 % {R.~Albrecht \etal}, \textit{ASP Conf. Ser.}, 192, 145

%\bibitem[1998]{Sal98b}�{{Shaw}, R.~A., {de La Pena}, M.~D., {Katsanis}, R.~M., \& {Williams}, R.~E.}
 % 1998, in ADASS VII, edited by
 % {R.~Albrecht, R.~N.~Hook, \& H.~A.~Bushouse}, \textit{ASP Conf. Ser.}, 192

%\bibitem[1995]{SD95}
%{Shaw}, R.~A., \& {Dufour}, R.~J. 1995, \textit{PASP}, 107, 896

\end{document}

%%%%%%%%%%%%%%%%%%%%%%%%%%%%%%%%%%%%%%%%%%%%%%%%%%%%%%%%%%%%%%
Examples for figures using graphicx
A guide "Using Imported Graphics in LaTeX2e"  (Keith Reckdahl)
is available on a lot of LaTeX public servers or ctan mirrors.
The file is : epslatex.pdf 
%%%%%%%%%%%%%%%%%%%%%%%%%%%%%%%%%%%%%%%%%%%%%%%%%%%%%%%%%%%%%%

%_____________________________________________________________
%                 A figure as large as the width of the column
%-------------------------------------------------------------
   \begin{figure}
   \centering
   \includegraphics[width=\hsize]{empty.eps}
      \caption{Vibrational stability equation of state
               $S_{\mathrm{vib}}(\lg e, \lg \rho)$.
               $>0$ means vibrational stability.
              }
         \label{FigVibStab}
   \end{figure}
%
%_____________________________________________________________
%                                    One column rotated figure
%-------------------------------------------------------------
   \begin{figure}
   \centering
   \includegraphics[angle=-90,width=3cm]{empty.eps}
      \caption{Vibrational stability equation of state
               $S_{\mathrm{vib}}(\lg e, \lg \rho)$.
               $>0$ means vibrational stability.
              }
         \label{FigVibStab}
   \end{figure}
%
%_____________________________________________________________
%                        Figure with caption on the right side 
%-------------------------------------------------------------
   \begin{figure}
   \sidecaption
   \includegraphics[width=3cm]{empty.eps}
      \caption{Vibrational stability equation of state
               $S_{\mathrm{vib}}(\lg e, \lg \rho)$.
               $>0$ means vibrational stability.
              }
         \label{FigVibStab}
   \end{figure}
%
%_____________________________________________________________
%
%_____________________________________________________________
%                                Figure with a new BoundingBox 
%-------------------------------------------------------------
   \begin{figure}
   \centering
   \includegraphics[bb=10 20 100 300,width=3cm,clip]{empty.eps}
      \caption{Vibrational stability equation of state
               $S_{\mathrm{vib}}(\lg e, \lg \rho)$.
               $>0$ means vibrational stability.
              }
         \label{FigVibStab}
   \end{figure}
%
%_____________________________________________________________
%
%_____________________________________________________________
%                                      The "resizebox" command 
%-------------------------------------------------------------
   \begin{figure}
   \resizebox{\hsize}{!}
            {\includegraphics[bb=10 20 100 300,clip]{empty.eps}
      \caption{Vibrational stability equation of state
               $S_{\mathrm{vib}}(\lg e, \lg \rho)$.
               $>0$ means vibrational stability.
              }
         \label{FigVibStab}
   \end{figure}
%
%______________________________________________________________
%
%_____________________________________________________________
%                                             Two column Figure 
%-------------------------------------------------------------
   \begin{figure*}
   \resizebox{\hsize}{!}
            {\includegraphics[bb=10 20 100 300,clip]{empty.eps}
      \caption{Vibrational stability equation of state
               $S_{\mathrm{vib}}(\lg e, \lg \rho)$.
               $>0$ means vibrational stability.
              }
         \label{FigVibStab}
   \end{figure*}
%
%______________________________________________________________
%
%_____________________________________________________________
%                                             Simple A&A Table
%_____________________________________________________________
%
\begin{table}
\caption{Nonlinear Model Results}             % title of Table
\label{table:1}      % is used to refer this table in the text
\centering                          % used for centering table
\begin{tabular}{c c c c}        % centered columns (4 columns)
\hline\hline                 % inserts double horizontal lines
HJD & $E$ & Method\#2 & Method\#3 \\    % table heading 
\hline                        % inserts single horizontal line
   1 & 50 & $-837$ & 970 \\      % inserting body of the table
   2 & 47 & 877    & 230 \\
   3 & 31 & 25     & 415 \\
   4 & 35 & 144    & 2356 \\
   5 & 45 & 300    & 556 \\ 
\hline                                   %inserts single line
\end{tabular}
\end{table}
%
%_____________________________________________________________
%                                             Two column Table 
%_____________________________________________________________
%
\begin{table*}
\caption{Nonlinear Model Results}             
\label{table:1}      
\centering          
\begin{tabular}{c c c c l l l }     % 7 columns 
\hline\hline       
                      % To combine 4 columns into a single one 
HJD & $E$ & Method\#2 & \multicolumn{4}{c}{Method\#3}\\ 
\hline                    
   1 & 50 & $-837$ & 970 & 65 & 67 & 78\\  
   2 & 47 & 877    & 230 & 567& 55 & 78\\
   3 & 31 & 25     & 415 & 567& 55 & 78\\
   4 & 35 & 144    & 2356& 567& 55 & 78 \\
   5 & 45 & 300    & 556 & 567& 55 & 78\\
\hline                  
\end{tabular}
\end{table*}
%
%-------------------------------------------------------------
%                                          Table with notes 
%-------------------------------------------------------------
%
% A single note
\begin{table}
\caption{\label{t7}Spectral types and photometry for stars in the
  region.}
\centering
\begin{tabular}{lccc}
\hline\hline
Star&Spectral type&RA(J2000)&Dec(J2000)\\
\hline
69           &B1\,V     &09 15 54.046 & $-$50 00 26.67\\
49           &B0.7\,V   &*09 15 54.570& $-$50 00 03.90\\
LS~1267~(86) &O8\,V     &09 15 52.787&11.07\\
24.6         &7.58      &1.37 &0.20\\
\hline
LS~1262      &B0\,V     &09 15 05.17&11.17\\
MO 2-119     &B0.5\,V   &09 15 33.7 &11.74\\
LS~1269      &O8.5\,V   &09 15 56.60&10.85\\
\hline
\end{tabular}
\tablefoot{The top panel shows likely members of Pismis~11. The second
panel contains likely members of Alicante~5. The bottom panel
displays stars outside the clusters.}
\end{table}
%
% More notes
%
\begin{table}
\caption{\label{t7}Spectral types and photometry for stars in the
  region.}
\centering
\begin{tabular}{lccc}
\hline\hline
Star&Spectral type&RA(J2000)&Dec(J2000)\\
\hline
69           &B1\,V     &09 15 54.046 & $-$50 00 26.67\\
49           &B0.7\,V   &*09 15 54.570& $-$50 00 03.90\\
LS~1267~(86) &O8\,V     &09 15 52.787&11.07\tablefootmark{a}\\
24.6         &7.58\tablefootmark{1}&1.37\tablefootmark{a}   &0.20\tablefootmark{a}\\
\hline
LS~1262      &B0\,V     &09 15 05.17&11.17\tablefootmark{b}\\
MO 2-119     &B0.5\,V   &09 15 33.7 &11.74\tablefootmark{c}\\
LS~1269      &O8.5\,V   &09 15 56.60&10.85\tablefootmark{d}\\
\hline
\end{tabular}
\tablefoot{The top panel shows likely members of Pismis~11. The second
panel contains likely members of Alicante~5. The bottom panel
displays stars outside the clusters.\\
\tablefoottext{a}{Photometry for MF13, LS~1267 and HD~80077 from
Dupont et al.}
\tablefoottext{b}{Photometry for LS~1262, LS~1269 from
Durand et al.}
\tablefoottext{c}{Photometry for MO2-119 from
Mathieu et al.}
}
\end{table}
%
%-------------------------------------------------------------
%                                       Table with references 
%-------------------------------------------------------------
%
\begin{table*}[h]
 \caption[]{\label{nearbylistaa2}List of nearby SNe used in this work.}
\begin{tabular}{lccc}
 \hline \hline
  SN name &
  Epoch &
 Bands &
  References \\
 &
  (with respect to $B$ maximum) &
 &
 \\ \hline
1981B   & 0 & {\it UBV} & 1\\
1986G   &  $-$3, $-$1, 0, 1, 2 & {\it BV}  & 2\\
1989B   & $-$5, $-$1, 0, 3, 5 & {\it UBVRI}  & 3, 4\\
1990N   & 2, 7 & {\it UBVRI}  & 5\\
1991M   & 3 & {\it VRI}  & 6\\
\hline
\noalign{\smallskip}
\multicolumn{4}{c}{ SNe 91bg-like} \\
\noalign{\smallskip}
\hline
1991bg   & 1, 2 & {\it BVRI}  & 7\\
1999by   & $-$5, $-$4, $-$3, 3, 4, 5 & {\it UBVRI}  & 8\\
\hline
\noalign{\smallskip}
\multicolumn{4}{c}{ SNe 91T-like} \\
\noalign{\smallskip}
\hline
1991T   & $-$3, 0 & {\it UBVRI}  &  9, 10\\
2000cx  & $-$3, $-$2, 0, 1, 5 & {\it UBVRI}  & 11\\ %
\hline
\end{tabular}
\tablebib{(1)~\citet{branch83};
(2) \citet{phillips87}; (3) \citet{barbon90}; (4) \citet{wells94};
(5) \citet{mazzali93}; (6) \citet{gomez98}; (7) \citet{kirshner93};
(8) \citet{patat96}; (9) \citet{salvo01}; (10) \citet{branch03};
(11) \citet{jha99}.
}
\end{table}
%_____________________________________________________________
%                      A rotated Two column Table in landscape  
%-------------------------------------------------------------
\begin{sidewaystable*}
\caption{Summary for ISOCAM sources with mid-IR excess 
(YSO candidates).}\label{YSOtable}
\centering
\begin{tabular}{crrlcl} 
\hline\hline             
ISO-L1551 & $F_{6.7}$~[mJy] & $\alpha_{6.7-14.3}$ 
& YSO type$^{d}$ & Status & Comments\\
\hline
  \multicolumn{6}{c}{\it New YSO candidates}\\ % To combine 6 columns into a single one
\hline
  1 & 1.56 $\pm$ 0.47 & --    & Class II$^{c}$ & New & Mid\\
  2 & 0.79:           & 0.97: & Class II ?     & New & \\
  3 & 4.95 $\pm$ 0.68 & 3.18  & Class II / III & New & \\
  5 & 1.44 $\pm$ 0.33 & 1.88  & Class II       & New & \\
\hline
  \multicolumn{6}{c}{\it Previously known YSOs} \\
\hline
  61 & 0.89 $\pm$ 0.58 & 1.77 & Class I & \object{HH 30} & Circumstellar disk\\
  96 & 38.34 $\pm$ 0.71 & 37.5& Class II& MHO 5          & Spectral type\\
\hline
\end{tabular}
\end{sidewaystable*}
%_____________________________________________________________
%                      A rotated One column Table in landscape  
%-------------------------------------------------------------
\begin{sidewaystable}
\caption{Summary for ISOCAM sources with mid-IR excess 
(YSO candidates).}\label{YSOtable}
\centering
\begin{tabular}{crrlcl} 
\hline\hline             
ISO-L1551 & $F_{6.7}$~[mJy] & $\alpha_{6.7-14.3}$ 
& YSO type$^{d}$ & Status & Comments\\
\hline
  \multicolumn{6}{c}{\it New YSO candidates}\\ % To combine 6 columns into a single one
\hline
  1 & 1.56 $\pm$ 0.47 & --    & Class II$^{c}$ & New & Mid\\
  2 & 0.79:           & 0.97: & Class II ?     & New & \\
  3 & 4.95 $\pm$ 0.68 & 3.18  & Class II / III & New & \\
  5 & 1.44 $\pm$ 0.33 & 1.88  & Class II       & New & \\
\hline
  \multicolumn{6}{c}{\it Previously known YSOs} \\
\hline
  61 & 0.89 $\pm$ 0.58 & 1.77 & Class I & \object{HH 30} & Circumstellar disk\\
  96 & 38.34 $\pm$ 0.71 & 37.5& Class II& MHO 5          & Spectral type\\
\hline
\end{tabular}
\end{sidewaystable}
%
%_____________________________________________________________
%                              Table longer than a single page  
%-------------------------------------------------------------
% All long tables will be placed automatically at the end, after 
%                                        \end{thebibliography}
%
\begin{longtab}
\begin{longtable}{lllrrr}
\caption{\label{kstars} Sample stars with absolute magnitude}\\
\hline\hline
Catalogue& $M_{V}$ & Spectral & Distance & Mode & Count Rate \\
\hline
\endfirsthead
\caption{continued.}\\
\hline\hline
Catalogue& $M_{V}$ & Spectral & Distance & Mode & Count Rate \\
\hline
\endhead
\hline
\endfoot
%%
Gl 33    & 6.37 & K2 V & 7.46 & S & 0.043170\\
Gl 66AB  & 6.26 & K2 V & 8.15 & S & 0.260478\\
Gl 68    & 5.87 & K1 V & 7.47 & P & 0.026610\\
         &      &      &      & H & 0.008686\\
Gl 86 
\footnote{Source not included in the HRI catalog. See Sect.~5.4.2 for details.}
         & 5.92 & K0 V & 10.91& S & 0.058230\\
\end{longtable}
\end{longtab}
%
%_____________________________________________________________
%                              Table longer than a single page
%                                             and in landscape 
%  In the preamble, use:       \usepackage{lscape}
%-------------------------------------------------------------
% All long tables will be placed automatically at the end, after
%                                        \end{thebibliography}
%
\begin{longtab}
\begin{landscape}
\begin{longtable}{lllrrr}
\caption{\label{kstars} Sample stars with absolute magnitude}\\
\hline\hline
Catalogue& $M_{V}$ & Spectral & Distance & Mode & Count Rate \\
\hline
\endfirsthead
\caption{continued.}\\
\hline\hline
Catalogue& $M_{V}$ & Spectral & Distance & Mode & Count Rate \\
\hline
\endhead
\hline
\endfoot
%%
Gl 33    & 6.37 & K2 V & 7.46 & S & 0.043170\\
Gl 66AB  & 6.26 & K2 V & 8.15 & S & 0.260478\\
Gl 68    & 5.87 & K1 V & 7.47 & P & 0.026610\\
         &      &      &      & H & 0.008686\\
Gl 86
\footnote{Source not included in the HRI catalog. See Sect.~5.4.2 for details.}
         & 5.92 & K0 V & 10.91& S & 0.058230\\
\end{longtable}
\end{landscape}
\end{longtab}
%
% Online Material
%_____________________________________________________________
%        Online appendices have to be placed at the end, after
%                                        \end{thebibliography}
%-------------------------------------------------------------
\end{thebibliography}

\Online

\begin{appendix} %First online appendix
\section{Background galaxy number counts and shear noise-levels}
Because the optical images used in this analysis...

\begin{figure*}
\centering
\includegraphics[width=16.4cm,clip]{1787f24.ps}
\caption{Plotted above...}
\label{appfig}
\end{figure*}

Because the optical images...
\end{appendix}

\begin{appendix} %Second online appendix
These studies, however, have faced...
\end{appendix}

\end{document}
%
%_____________________________________________________________
%        Some tables or figures are in the printed version and
%                      some are only in the electronic version
%-------------------------------------------------------------
%
% Leave all the tables or figures in the text, at their right place 
% and use the commands \onlfig{} and \onltab{}. These elements
% will be automatically placed at the end, in the section
% Online material.

\documentclass{aa}
...
\begin{document}
text of the paper...
\begin{figure*}%f1
\includegraphics[width=10.9cm]{1787f01.eps}
\caption{Shown in greyscale is a...}
\label{cl12301}}
\end{figure*}
...
from the intrinsic ellipticity distribution.
% Figure 2 available electronically only
\onlfig{
\begin{figure*}%f2
\includegraphics[width=11.6cm]{1787f02.eps}
\caption {Shown in greyscale...}
\label{cl1018}
\end{figure*}
}

% Figure 3 available electronically only
\onlfig{
\begin{figure*}%f3
\includegraphics[width=11.2cm]{1787f03.eps}
\caption{Shown in panels...}
\label{cl1059}
\end{figure*}
}

\begin{figure*}%f4
\includegraphics[width=10.9cm]{1787f04.eps}
\caption{Shown in greyscale is...}
\label{cl1232}}
\end{figure*}

\begin{table}%t1
\caption{Complexes characterisation.}\label{starbursts}
\centering
\begin{tabular}{lccc}
\hline \hline
Complex & $F_{60}$ & 8.6 &  No. of  \\
...
\hline
\end{tabular}
\end{table}
The second method produces...

% Figure 5 available electronically only
\onlfig{
\begin{figure*}%f5
\includegraphics[width=11.2cm]{1787f05.eps}
\caption{Shown in panels...}
\label{cl1238}}
\end{figure*}
}

As can be seen, in general the deeper...
% Table 2 available electronically only
\onltab{
\begin{table*}%t2
\caption{List of the LMC stellar complexes...}\label{Properties}
\centering
\begin{tabular}{lccccccccc}
\hline  \hline
Stellar & RA & Dec & ...
...
\hline
\end{tabular}
\end{table*}
}

% Table 3 available electronically only
\onltab{
\begin{table*}%t3
\caption{List of the derived...}\label{IrasFluxes}
\centering
\begin{tabular}{lcccccccccc}
\hline \hline
Stellar & $f12$ & $L12$ &...
...
\hline
\end{tabular}
\end{table*}
}
%
%-------------------------------------------------------------
%     For the online material, table longer than a single page
%                 In the preamble for landscape case, use : 
%                                          \usepackage{lscape}
%-------------------------------------------------------------
\documentclass{aa}
\usepackage[varg]{txfonts}
\usepackage{graphicx}
\usepackage{lscape}

\begin{document}
text of the paper
% Table will be print automatically at the end, in the section Online material.
\onllongtab{
\begin{longtable}{lrcrrrrrrrrl}
\caption{Line data and abundances ...}\\
\hline
\hline
Def & mol & Ion & $\lambda$ & $\chi$ & $\log gf$ & N & e &  rad & $\delta$ & $\delta$ 
red & References \\
\hline
\endfirsthead
\caption{Continued.} \\
\hline
Def & mol & Ion & $\lambda$ & $\chi$ & $\log gf$ & B & C &  rad & $\delta$ & $\delta$ 
red & References \\
\hline
\endhead
\hline
\endfoot
\hline
\endlastfoot
A & CH & 1 &3638 & 0.002 & $-$2.551 &  &  &  & $-$150 & 150 &  Jorgensen et al. (1996) \\                    
\end{longtable}
}% End onllongtab

% Or for landscape, large table:

\onllongtab{
\begin{landscape}
\begin{longtable}{lrcrrrrrrrrl}
...
\end{longtable}
\end{landscape}
}% End onllongtab
