\hypertarget{chebyshev_8py_source}{}\section{chebyshev.\+py}
\label{chebyshev_8py_source}\index{pyneb/utils/chebyshev.\+py@{pyneb/utils/chebyshev.\+py}}

\begin{DoxyCode}
\hypertarget{chebyshev_8py_source_l00001}{}\hyperlink{namespacepyneb_1_1utils_1_1chebyshev}{00001} \textcolor{stringliteral}{"""}
00002 \textcolor{stringliteral}{Objects for dealing with Chebyshev series.}
00003 \textcolor{stringliteral}{}
00004 \textcolor{stringliteral}{This module provides a number of objects (mostly functions) useful for}
00005 \textcolor{stringliteral}{dealing with Chebyshev series, including a `Chebyshev` class that}
00006 \textcolor{stringliteral}{encapsulates the usual arithmetic operations.  (General information}
00007 \textcolor{stringliteral}{on how this module represents and works with such polynomials is in the}
00008 \textcolor{stringliteral}{docstring for its "parent" sub-package, `numpy.polynomial`).}
00009 \textcolor{stringliteral}{}
00010 \textcolor{stringliteral}{Constants}
00011 \textcolor{stringliteral}{---------}
00012 \textcolor{stringliteral}{- `chebdomain` -- Chebyshev series default domain, [-1,1].}
00013 \textcolor{stringliteral}{- `chebzero` -- (Coefficients of the) Chebyshev series that evaluates}
00014 \textcolor{stringliteral}{  identically to 0.}
00015 \textcolor{stringliteral}{- `chebone` -- (Coefficients of the) Chebyshev series that evaluates}
00016 \textcolor{stringliteral}{  identically to 1.}
00017 \textcolor{stringliteral}{- `chebx` -- (Coefficients of the) Chebyshev series for the identity map,}
00018 \textcolor{stringliteral}{  ``f(x) = x``.}
00019 \textcolor{stringliteral}{}
00020 \textcolor{stringliteral}{Arithmetic}
00021 \textcolor{stringliteral}{----------}
00022 \textcolor{stringliteral}{- `chebadd` -- add two Chebyshev series.}
00023 \textcolor{stringliteral}{- `chebsub` -- subtract one Chebyshev series from another.}
00024 \textcolor{stringliteral}{- `chebmul` -- multiply two Chebyshev series.}
00025 \textcolor{stringliteral}{- `chebdiv` -- divide one Chebyshev series by another.}
00026 \textcolor{stringliteral}{- `chebpow` -- raise a Chebyshev series to an positive integer power}
00027 \textcolor{stringliteral}{- `chebval` -- evaluate a Chebyshev series at given points.}
00028 \textcolor{stringliteral}{}
00029 \textcolor{stringliteral}{Calculus}
00030 \textcolor{stringliteral}{--------}
00031 \textcolor{stringliteral}{- `chebder` -- differentiate a Chebyshev series.}
00032 \textcolor{stringliteral}{- `chebint` -- integrate a Chebyshev series.}
00033 \textcolor{stringliteral}{}
00034 \textcolor{stringliteral}{Misc Functions}
00035 \textcolor{stringliteral}{--------------}
00036 \textcolor{stringliteral}{- `chebfromroots` -- create a Chebyshev series with specified roots.}
00037 \textcolor{stringliteral}{- `chebroots` -- find the roots of a Chebyshev series.}
00038 \textcolor{stringliteral}{- `chebvander` -- Vandermonde-like matrix for Chebyshev polynomials.}
00039 \textcolor{stringliteral}{- `chebfit` -- least-squares fit returning a Chebyshev series.}
00040 \textcolor{stringliteral}{- `chebpts1` -- Chebyshev points of the first kind.}
00041 \textcolor{stringliteral}{- `chebpts2` -- Chebyshev points of the second kind.}
00042 \textcolor{stringliteral}{- `chebtrim` -- trim leading coefficients from a Chebyshev series.}
00043 \textcolor{stringliteral}{- `chebline` -- Chebyshev series representing given straight line.}
00044 \textcolor{stringliteral}{- `cheb2poly` -- convert a Chebyshev series to a polynomial.}
00045 \textcolor{stringliteral}{- `poly2cheb` -- convert a polynomial to a Chebyshev series.}
00046 \textcolor{stringliteral}{}
00047 \textcolor{stringliteral}{Classes}
00048 \textcolor{stringliteral}{-------}
00049 \textcolor{stringliteral}{- `Chebyshev` -- A Chebyshev series class.}
00050 \textcolor{stringliteral}{}
00051 \textcolor{stringliteral}{See also}
00052 \textcolor{stringliteral}{--------}
00053 \textcolor{stringliteral}{`numpy.polynomial`}
00054 \textcolor{stringliteral}{}
00055 \textcolor{stringliteral}{Notes}
00056 \textcolor{stringliteral}{-----}
00057 \textcolor{stringliteral}{The implementations of multiplication, division, integration, and}
00058 \textcolor{stringliteral}{differentiation use the algebraic identities [1]\_:}
00059 \textcolor{stringliteral}{}
00060 \textcolor{stringliteral}{.. math ::}
00061 \textcolor{stringliteral}{    T\_n(x) = \(\backslash\)\(\backslash\)frac\{z^n + z^\{-n\}\}\{2\} \(\backslash\)\(\backslash\)\(\backslash\)\(\backslash\)}
00062 \textcolor{stringliteral}{    z\(\backslash\)\(\backslash\)frac\{dx\}\{dz\} = \(\backslash\)\(\backslash\)frac\{z - z^\{-1\}\}\{2\}.}
00063 \textcolor{stringliteral}{}
00064 \textcolor{stringliteral}{where}
00065 \textcolor{stringliteral}{}
00066 \textcolor{stringliteral}{.. math :: x = \(\backslash\)\(\backslash\)frac\{z + z^\{-1\}\}\{2\}.}
00067 \textcolor{stringliteral}{}
00068 \textcolor{stringliteral}{These identities allow a Chebyshev series to be expressed as a finite,}
00069 \textcolor{stringliteral}{symmetric Laurent series.  In this module, this sort of Laurent series}
00070 \textcolor{stringliteral}{is referred to as a "z-series."}
00071 \textcolor{stringliteral}{}
00072 \textcolor{stringliteral}{References}
00073 \textcolor{stringliteral}{----------}
00074 \textcolor{stringliteral}{.. [1] A. T. Benjamin, et al., "Combinatorial Trigonometry with Chebyshev}
00075 \textcolor{stringliteral}{  Polynomials," *Journal of Statistical Planning and Inference 14*, 2008}
00076 \textcolor{stringliteral}{  (preprint: http://www.math.hmc.edu/~benjamin/papers/CombTrig.pdf, pg. 4)}
00077 \textcolor{stringliteral}{}
00078 \textcolor{stringliteral}{"""}
00079 \textcolor{keyword}{from} \_\_future\_\_ \textcolor{keyword}{import} division
00080 
\hypertarget{chebyshev_8py_source_l00081}{}\hyperlink{namespacepyneb_1_1utils_1_1chebyshev_a9cb4e1ba55fc61d6dd46d5c6f534edbc}{00081} \_\_all\_\_ = [\textcolor{stringliteral}{'chebzero'}, \textcolor{stringliteral}{'chebone'}, \textcolor{stringliteral}{'chebx'}, \textcolor{stringliteral}{'chebdomain'}, \textcolor{stringliteral}{'chebline'},
00082         \textcolor{stringliteral}{'chebadd'}, \textcolor{stringliteral}{'chebsub'}, \textcolor{stringliteral}{'chebmulx'}, \textcolor{stringliteral}{'chebmul'}, \textcolor{stringliteral}{'chebdiv'}, \textcolor{stringliteral}{'chebpow'},
00083         \textcolor{stringliteral}{'chebval'}, \textcolor{stringliteral}{'chebder'}, \textcolor{stringliteral}{'chebint'}, \textcolor{stringliteral}{'cheb2poly'}, \textcolor{stringliteral}{'poly2cheb'},
00084         \textcolor{stringliteral}{'chebfromroots'}, \textcolor{stringliteral}{'chebvander'}, \textcolor{stringliteral}{'chebfit'}, \textcolor{stringliteral}{'chebtrim'}, \textcolor{stringliteral}{'chebroots'},
00085         \textcolor{stringliteral}{'chebpts1'}, \textcolor{stringliteral}{'chebpts2'}, \textcolor{stringliteral}{'Chebyshev'}]
00086 
00087 \textcolor{keyword}{import} numpy \textcolor{keyword}{as} np
00088 \textcolor{keyword}{import} numpy.linalg \textcolor{keyword}{as} la
00089 \textcolor{keyword}{import} polyutils \textcolor{keyword}{as} pu
00090 \textcolor{keyword}{import} warnings
00091 \textcolor{keyword}{from} polytemplate \textcolor{keyword}{import} polytemplate
00092 
\hypertarget{chebyshev_8py_source_l00093}{}\hyperlink{namespacepyneb_1_1utils_1_1chebyshev_a66b8c25023851573f25dc418071de82b}{00093} chebtrim = pu.trimcoef
00094 
00095 \textcolor{comment}{#}
00096 \textcolor{comment}{# A collection of functions for manipulating z-series. These are private}
00097 \textcolor{comment}{# functions and do minimal error checking.}
00098 \textcolor{comment}{#}
00099 
\hypertarget{chebyshev_8py_source_l00100}{}\hyperlink{namespacepyneb_1_1utils_1_1chebyshev_a659e346f7cdd9fd058850b26f7e95b17}{00100} \textcolor{keyword}{def }\hyperlink{namespacepyneb_1_1utils_1_1chebyshev_a659e346f7cdd9fd058850b26f7e95b17}{\_cseries\_to\_zseries}(cs) :
00101     \textcolor{stringliteral}{"""Covert Chebyshev series to z-series.}
00102 \textcolor{stringliteral}{}
00103 \textcolor{stringliteral}{    Covert a Chebyshev series to the equivalent z-series. The result is}
00104 \textcolor{stringliteral}{    never an empty array. The dtype of the return is the same as that of}
00105 \textcolor{stringliteral}{    the input. No checks are run on the arguments as this routine is for}
00106 \textcolor{stringliteral}{    internal use.}
00107 \textcolor{stringliteral}{}
00108 \textcolor{stringliteral}{    Parameters}
00109 \textcolor{stringliteral}{    ----------}
00110 \textcolor{stringliteral}{    cs : 1-d ndarray}
00111 \textcolor{stringliteral}{        Chebyshev coefficients, ordered from low to high}
00112 \textcolor{stringliteral}{}
00113 \textcolor{stringliteral}{    Returns}
00114 \textcolor{stringliteral}{    -------}
00115 \textcolor{stringliteral}{    zs : 1-d ndarray}
00116 \textcolor{stringliteral}{        Odd length symmetric z-series, ordered from  low to high.}
00117 \textcolor{stringliteral}{}
00118 \textcolor{stringliteral}{    """}
00119     n = cs.size
00120     zs = np.zeros(2*n-1, dtype=cs.dtype)
00121     zs[n-1:] = cs/2
00122     \textcolor{keywordflow}{return} zs + zs[::-1]
00123 
\hypertarget{chebyshev_8py_source_l00124}{}\hyperlink{namespacepyneb_1_1utils_1_1chebyshev_a2a88474ce0ea12fb26f82b7116752dc1}{00124} \textcolor{keyword}{def }\hyperlink{namespacepyneb_1_1utils_1_1chebyshev_a2a88474ce0ea12fb26f82b7116752dc1}{\_zseries\_to\_cseries}(zs) :
00125     \textcolor{stringliteral}{"""Covert z-series to a Chebyshev series.}
00126 \textcolor{stringliteral}{}
00127 \textcolor{stringliteral}{    Covert a z series to the equivalent Chebyshev series. The result is}
00128 \textcolor{stringliteral}{    never an empty array. The dtype of the return is the same as that of}
00129 \textcolor{stringliteral}{    the input. No checks are run on the arguments as this routine is for}
00130 \textcolor{stringliteral}{    internal use.}
00131 \textcolor{stringliteral}{}
00132 \textcolor{stringliteral}{    Parameters}
00133 \textcolor{stringliteral}{    ----------}
00134 \textcolor{stringliteral}{    zs : 1-d ndarray}
00135 \textcolor{stringliteral}{        Odd length symmetric z-series, ordered from  low to high.}
00136 \textcolor{stringliteral}{}
00137 \textcolor{stringliteral}{    Returns}
00138 \textcolor{stringliteral}{    -------}
00139 \textcolor{stringliteral}{    cs : 1-d ndarray}
00140 \textcolor{stringliteral}{        Chebyshev coefficients, ordered from  low to high.}
00141 \textcolor{stringliteral}{}
00142 \textcolor{stringliteral}{    """}
00143     n = (zs.size + 1)//2
00144     cs = zs[n-1:].copy()
00145     cs[1:n] *= 2
00146     \textcolor{keywordflow}{return} cs
00147 
\hypertarget{chebyshev_8py_source_l00148}{}\hyperlink{namespacepyneb_1_1utils_1_1chebyshev_a46d958933482afac66ea41d143155dae}{00148} \textcolor{keyword}{def }\hyperlink{namespacepyneb_1_1utils_1_1chebyshev_a46d958933482afac66ea41d143155dae}{\_zseries\_mul}(z1, z2) :
00149     \textcolor{stringliteral}{"""Multiply two z-series.}
00150 \textcolor{stringliteral}{}
00151 \textcolor{stringliteral}{    Multiply two z-series to produce a z-series.}
00152 \textcolor{stringliteral}{}
00153 \textcolor{stringliteral}{    Parameters}
00154 \textcolor{stringliteral}{    ----------}
00155 \textcolor{stringliteral}{    z1, z2 : 1-d ndarray}
00156 \textcolor{stringliteral}{        The arrays must be 1-d but this is not checked.}
00157 \textcolor{stringliteral}{}
00158 \textcolor{stringliteral}{    Returns}
00159 \textcolor{stringliteral}{    -------}
00160 \textcolor{stringliteral}{    product : 1-d ndarray}
00161 \textcolor{stringliteral}{        The product z-series.}
00162 \textcolor{stringliteral}{}
00163 \textcolor{stringliteral}{    Notes}
00164 \textcolor{stringliteral}{    -----}
00165 \textcolor{stringliteral}{    This is simply convolution. If symmetic/anti-symmetric z-series are}
00166 \textcolor{stringliteral}{    denoted by S/A then the following rules apply:}
00167 \textcolor{stringliteral}{}
00168 \textcolor{stringliteral}{    S*S, A*A -> S}
00169 \textcolor{stringliteral}{    S*A, A*S -> A}
00170 \textcolor{stringliteral}{}
00171 \textcolor{stringliteral}{    """}
00172     \textcolor{keywordflow}{return} np.convolve(z1, z2)
00173 
\hypertarget{chebyshev_8py_source_l00174}{}\hyperlink{namespacepyneb_1_1utils_1_1chebyshev_ad00f2653194746fa98c3421eb7c26f2c}{00174} \textcolor{keyword}{def }\hyperlink{namespacepyneb_1_1utils_1_1chebyshev_ad00f2653194746fa98c3421eb7c26f2c}{\_zseries\_div}(z1, z2) :
00175     \textcolor{stringliteral}{"""Divide the first z-series by the second.}
00176 \textcolor{stringliteral}{}
00177 \textcolor{stringliteral}{    Divide `z1` by `z2` and return the quotient and remainder as z-series.}
00178 \textcolor{stringliteral}{    Warning: this implementation only applies when both z1 and z2 have the}
00179 \textcolor{stringliteral}{    same symmetry, which is sufficient for present purposes.}
00180 \textcolor{stringliteral}{}
00181 \textcolor{stringliteral}{    Parameters}
00182 \textcolor{stringliteral}{    ----------}
00183 \textcolor{stringliteral}{    z1, z2 : 1-d ndarray}
00184 \textcolor{stringliteral}{        The arrays must be 1-d and have the same symmetry, but this is not}
00185 \textcolor{stringliteral}{        checked.}
00186 \textcolor{stringliteral}{}
00187 \textcolor{stringliteral}{    Returns}
00188 \textcolor{stringliteral}{    -------}
00189 \textcolor{stringliteral}{}
00190 \textcolor{stringliteral}{    (quotient, remainder) : 1-d ndarrays}
00191 \textcolor{stringliteral}{        Quotient and remainder as z-series.}
00192 \textcolor{stringliteral}{}
00193 \textcolor{stringliteral}{    Notes}
00194 \textcolor{stringliteral}{    -----}
00195 \textcolor{stringliteral}{    This is not the same as polynomial division on account of the desired form}
00196 \textcolor{stringliteral}{    of the remainder. If symmetic/anti-symmetric z-series are denoted by S/A}
00197 \textcolor{stringliteral}{    then the following rules apply:}
00198 \textcolor{stringliteral}{}
00199 \textcolor{stringliteral}{    S/S -> S,S}
00200 \textcolor{stringliteral}{    A/A -> S,A}
00201 \textcolor{stringliteral}{}
00202 \textcolor{stringliteral}{    The restriction to types of the same symmetry could be fixed but seems like}
00203 \textcolor{stringliteral}{    uneeded generality. There is no natural form for the remainder in the case}
00204 \textcolor{stringliteral}{    where there is no symmetry.}
00205 \textcolor{stringliteral}{}
00206 \textcolor{stringliteral}{    """}
00207     z1 = z1.copy()
00208     z2 = z2.copy()
00209     len1 = len(z1)
00210     len2 = len(z2)
00211     \textcolor{keywordflow}{if} len2 == 1 :
00212         z1 /= z2
00213         \textcolor{keywordflow}{return} z1, z1[:1]*0
00214     \textcolor{keywordflow}{elif} len1 < len2 :
00215         \textcolor{keywordflow}{return} z1[:1]*0, z1
00216     \textcolor{keywordflow}{else} :
00217         dlen = len1 - len2
00218         scl = z2[0]
00219         z2 /= scl
00220         quo = np.empty(dlen + 1, dtype=z1.dtype)
00221         i = 0
00222         j = dlen
00223         \textcolor{keywordflow}{while} i < j :
00224             r = z1[i]
00225             quo[i] = z1[i]
00226             quo[dlen - i] = r
00227             tmp = r*z2
00228             z1[i:i+len2] -= tmp
00229             z1[j:j+len2] -= tmp
00230             i += 1
00231             j -= 1
00232         r = z1[i]
00233         quo[i] = r
00234         tmp = r*z2
00235         z1[i:i+len2] -= tmp
00236         quo /= scl
00237         rem = z1[i+1:i-1+len2].copy()
00238         \textcolor{keywordflow}{return} quo, rem
00239 
\hypertarget{chebyshev_8py_source_l00240}{}\hyperlink{namespacepyneb_1_1utils_1_1chebyshev_adb39b6e9d3aba3bf3784a3364ec3e6ca}{00240} \textcolor{keyword}{def }\hyperlink{namespacepyneb_1_1utils_1_1chebyshev_adb39b6e9d3aba3bf3784a3364ec3e6ca}{\_zseries\_der}(zs) :
00241     \textcolor{stringliteral}{"""Differentiate a z-series.}
00242 \textcolor{stringliteral}{}
00243 \textcolor{stringliteral}{    The derivative is with respect to x, not z. This is achieved using the}
00244 \textcolor{stringliteral}{    chain rule and the value of dx/dz given in the module notes.}
00245 \textcolor{stringliteral}{}
00246 \textcolor{stringliteral}{    Parameters}
00247 \textcolor{stringliteral}{    ----------}
00248 \textcolor{stringliteral}{    zs : z-series}
00249 \textcolor{stringliteral}{        The z-series to differentiate.}
00250 \textcolor{stringliteral}{}
00251 \textcolor{stringliteral}{    Returns}
00252 \textcolor{stringliteral}{    -------}
00253 \textcolor{stringliteral}{    derivative : z-series}
00254 \textcolor{stringliteral}{        The derivative}
00255 \textcolor{stringliteral}{}
00256 \textcolor{stringliteral}{    Notes}
00257 \textcolor{stringliteral}{    -----}
00258 \textcolor{stringliteral}{    The zseries for x (ns) has been multiplied by two in order to avoid}
00259 \textcolor{stringliteral}{    using floats that are incompatible with Decimal and likely other}
00260 \textcolor{stringliteral}{    specialized scalar types. This scaling has been compensated by}
00261 \textcolor{stringliteral}{    multiplying the value of zs by two also so that the two cancels in the}
00262 \textcolor{stringliteral}{    division.}
00263 \textcolor{stringliteral}{}
00264 \textcolor{stringliteral}{    """}
00265     n = len(zs)//2
00266     ns = np.array([-1, 0, 1], dtype=zs.dtype)
00267     zs *= np.arange(-n, n+1)*2
00268     d, r = \hyperlink{namespacepyneb_1_1utils_1_1chebyshev_ad00f2653194746fa98c3421eb7c26f2c}{\_zseries\_div}(zs, ns)
00269     \textcolor{keywordflow}{return} d
00270 
\hypertarget{chebyshev_8py_source_l00271}{}\hyperlink{namespacepyneb_1_1utils_1_1chebyshev_a39d3ae73408fcde399d87f5cfd5e7c98}{00271} \textcolor{keyword}{def }\hyperlink{namespacepyneb_1_1utils_1_1chebyshev_a39d3ae73408fcde399d87f5cfd5e7c98}{\_zseries\_int}(zs) :
00272     \textcolor{stringliteral}{"""Integrate a z-series.}
00273 \textcolor{stringliteral}{}
00274 \textcolor{stringliteral}{    The integral is with respect to x, not z. This is achieved by a change}
00275 \textcolor{stringliteral}{    of variable using dx/dz given in the module notes.}
00276 \textcolor{stringliteral}{}
00277 \textcolor{stringliteral}{    Parameters}
00278 \textcolor{stringliteral}{    ----------}
00279 \textcolor{stringliteral}{    zs : z-series}
00280 \textcolor{stringliteral}{        The z-series to integrate}
00281 \textcolor{stringliteral}{}
00282 \textcolor{stringliteral}{    Returns}
00283 \textcolor{stringliteral}{    -------}
00284 \textcolor{stringliteral}{    integral : z-series}
00285 \textcolor{stringliteral}{        The indefinite integral}
00286 \textcolor{stringliteral}{}
00287 \textcolor{stringliteral}{    Notes}
00288 \textcolor{stringliteral}{    -----}
00289 \textcolor{stringliteral}{    The zseries for x (ns) has been multiplied by two in order to avoid}
00290 \textcolor{stringliteral}{    using floats that are incompatible with Decimal and likely other}
00291 \textcolor{stringliteral}{    specialized scalar types. This scaling has been compensated by}
00292 \textcolor{stringliteral}{    dividing the resulting zs by two.}
00293 \textcolor{stringliteral}{}
00294 \textcolor{stringliteral}{    """}
00295     n = 1 + len(zs)//2
00296     ns = np.array([-1, 0, 1], dtype=zs.dtype)
00297     zs = \hyperlink{namespacepyneb_1_1utils_1_1chebyshev_a46d958933482afac66ea41d143155dae}{\_zseries\_mul}(zs, ns)
00298     div = np.arange(-n, n+1)*2
00299     zs[:n] /= div[:n]
00300     zs[n+1:] /= div[n+1:]
00301     zs[n] = 0
00302     \textcolor{keywordflow}{return} zs
00303 
00304 \textcolor{comment}{#}
00305 \textcolor{comment}{# Chebyshev series functions}
00306 \textcolor{comment}{#}
00307 
00308 
\hypertarget{chebyshev_8py_source_l00309}{}\hyperlink{namespacepyneb_1_1utils_1_1chebyshev_a1f91018c86380adf38318e4e224279e7}{00309} \textcolor{keyword}{def }\hyperlink{namespacepyneb_1_1utils_1_1chebyshev_a1f91018c86380adf38318e4e224279e7}{poly2cheb}(pol) :
00310     \textcolor{stringliteral}{"""}
00311 \textcolor{stringliteral}{    Convert a polynomial to a Chebyshev series.}
00312 \textcolor{stringliteral}{}
00313 \textcolor{stringliteral}{    Convert an array representing the coefficients of a polynomial (relative}
00314 \textcolor{stringliteral}{    to the "standard" basis) ordered from lowest degree to highest, to an}
00315 \textcolor{stringliteral}{    array of the coefficients of the equivalent Chebyshev series, ordered}
00316 \textcolor{stringliteral}{    from lowest to highest degree.}
00317 \textcolor{stringliteral}{}
00318 \textcolor{stringliteral}{    Parameters}
00319 \textcolor{stringliteral}{    ----------}
00320 \textcolor{stringliteral}{    pol : array\_like}
00321 \textcolor{stringliteral}{        1-d array containing the polynomial coefficients}
00322 \textcolor{stringliteral}{}
00323 \textcolor{stringliteral}{    Returns}
00324 \textcolor{stringliteral}{    -------}
00325 \textcolor{stringliteral}{    cs : ndarray}
00326 \textcolor{stringliteral}{        1-d array containing the coefficients of the equivalent Chebyshev}
00327 \textcolor{stringliteral}{        series.}
00328 \textcolor{stringliteral}{}
00329 \textcolor{stringliteral}{    See Also}
00330 \textcolor{stringliteral}{    --------}
00331 \textcolor{stringliteral}{    cheb2poly}
00332 \textcolor{stringliteral}{}
00333 \textcolor{stringliteral}{    Notes}
00334 \textcolor{stringliteral}{    -----}
00335 \textcolor{stringliteral}{    The easy way to do conversions between polynomial basis sets}
00336 \textcolor{stringliteral}{    is to use the convert method of a class instance.}
00337 \textcolor{stringliteral}{}
00338 \textcolor{stringliteral}{    Examples}
00339 \textcolor{stringliteral}{    --------}
00340 \textcolor{stringliteral}{    >>> from numpy import polynomial as P}
00341 \textcolor{stringliteral}{    >>> p = P.Polynomial(range(4))}
00342 \textcolor{stringliteral}{    >>> p}
00343 \textcolor{stringliteral}{    Polynomial([ 0.,  1.,  2.,  3.], [-1.,  1.])}
00344 \textcolor{stringliteral}{    >>> c = p.convert(kind=P.Chebyshev)}
00345 \textcolor{stringliteral}{    >>> c}
00346 \textcolor{stringliteral}{    Chebyshev([ 1.  ,  3.25,  1.  ,  0.75], [-1.,  1.])}
00347 \textcolor{stringliteral}{    >>> P.poly2cheb(range(4))}
00348 \textcolor{stringliteral}{    array([ 1.  ,  3.25,  1.  ,  0.75])}
00349 \textcolor{stringliteral}{}
00350 \textcolor{stringliteral}{    """}
00351     [pol] = pu.as\_series([pol])
00352     deg = len(pol) - 1
00353     res = 0
00354     \textcolor{keywordflow}{for} i \textcolor{keywordflow}{in} range(deg, -1, -1) :
00355         res = \hyperlink{namespacepyneb_1_1utils_1_1chebyshev_a06a85eb72ae35925336a6b550dec3f02}{chebadd}(\hyperlink{namespacepyneb_1_1utils_1_1chebyshev_ab09f8accee50bd964e03071a659a7b5f}{chebmulx}(res), pol[i])
00356     \textcolor{keywordflow}{return} res
00357 
00358 
\hypertarget{chebyshev_8py_source_l00359}{}\hyperlink{namespacepyneb_1_1utils_1_1chebyshev_a640344fc14aa36e14d0e93eba4411188}{00359} \textcolor{keyword}{def }\hyperlink{namespacepyneb_1_1utils_1_1chebyshev_a640344fc14aa36e14d0e93eba4411188}{cheb2poly}(cs) :
00360     \textcolor{stringliteral}{"""}
00361 \textcolor{stringliteral}{    Convert a Chebyshev series to a polynomial.}
00362 \textcolor{stringliteral}{}
00363 \textcolor{stringliteral}{    Convert an array representing the coefficients of a Chebyshev series,}
00364 \textcolor{stringliteral}{    ordered from lowest degree to highest, to an array of the coefficients}
00365 \textcolor{stringliteral}{    of the equivalent polynomial (relative to the "standard" basis) ordered}
00366 \textcolor{stringliteral}{    from lowest to highest degree.}
00367 \textcolor{stringliteral}{}
00368 \textcolor{stringliteral}{    Parameters}
00369 \textcolor{stringliteral}{    ----------}
00370 \textcolor{stringliteral}{    cs : array\_like}
00371 \textcolor{stringliteral}{        1-d array containing the Chebyshev series coefficients, ordered}
00372 \textcolor{stringliteral}{        from lowest order term to highest.}
00373 \textcolor{stringliteral}{}
00374 \textcolor{stringliteral}{    Returns}
00375 \textcolor{stringliteral}{    -------}
00376 \textcolor{stringliteral}{    pol : ndarray}
00377 \textcolor{stringliteral}{        1-d array containing the coefficients of the equivalent polynomial}
00378 \textcolor{stringliteral}{        (relative to the "standard" basis) ordered from lowest order term}
00379 \textcolor{stringliteral}{        to highest.}
00380 \textcolor{stringliteral}{}
00381 \textcolor{stringliteral}{    See Also}
00382 \textcolor{stringliteral}{    --------}
00383 \textcolor{stringliteral}{    poly2cheb}
00384 \textcolor{stringliteral}{}
00385 \textcolor{stringliteral}{    Notes}
00386 \textcolor{stringliteral}{    -----}
00387 \textcolor{stringliteral}{    The easy way to do conversions between polynomial basis sets}
00388 \textcolor{stringliteral}{    is to use the convert method of a class instance.}
00389 \textcolor{stringliteral}{}
00390 \textcolor{stringliteral}{    Examples}
00391 \textcolor{stringliteral}{    --------}
00392 \textcolor{stringliteral}{    >>> from numpy import polynomial as P}
00393 \textcolor{stringliteral}{    >>> c = P.Chebyshev(range(4))}
00394 \textcolor{stringliteral}{    >>> c}
00395 \textcolor{stringliteral}{    Chebyshev([ 0.,  1.,  2.,  3.], [-1.,  1.])}
00396 \textcolor{stringliteral}{    >>> p = c.convert(kind=P.Polynomial)}
00397 \textcolor{stringliteral}{    >>> p}
00398 \textcolor{stringliteral}{    Polynomial([ -2.,  -8.,   4.,  12.], [-1.,  1.])}
00399 \textcolor{stringliteral}{    >>> P.cheb2poly(range(4))}
00400 \textcolor{stringliteral}{    array([ -2.,  -8.,   4.,  12.])}
00401 \textcolor{stringliteral}{}
00402 \textcolor{stringliteral}{    """}
00403     \textcolor{keyword}{from} polynomial \textcolor{keyword}{import} polyadd, polysub, polymulx
00404 
00405     [cs] = pu.as\_series([cs])
00406     n = len(cs)
00407     \textcolor{keywordflow}{if} n < 3:
00408         \textcolor{keywordflow}{return} cs
00409     \textcolor{keywordflow}{else}:
00410         c0 = cs[-2]
00411         c1 = cs[-1]
00412         \textcolor{comment}{# i is the current degree of c1}
00413         \textcolor{keywordflow}{for} i \textcolor{keywordflow}{in} range(n - 1, 1, -1) :
00414             tmp = c0
00415             c0 = polysub(cs[i - 2], c1)
00416             c1 = polyadd(tmp, polymulx(c1)*2)
00417         \textcolor{keywordflow}{return} polyadd(c0, polymulx(c1))
00418 
00419 
00420 \textcolor{comment}{#}
00421 \textcolor{comment}{# These are constant arrays are of integer type so as to be compatible}
00422 \textcolor{comment}{# with the widest range of other types, such as Decimal.}
00423 \textcolor{comment}{#}
00424 
00425 \textcolor{comment}{# Chebyshev default domain.}
\hypertarget{chebyshev_8py_source_l00426}{}\hyperlink{namespacepyneb_1_1utils_1_1chebyshev_a57e110567544748d0de429d3743d1241}{00426} chebdomain = np.array([-1,1])
00427 
00428 \textcolor{comment}{# Chebyshev coefficients representing zero.}
\hypertarget{chebyshev_8py_source_l00429}{}\hyperlink{namespacepyneb_1_1utils_1_1chebyshev_ad707fa9ec865920c2d8eedf6dfedf21c}{00429} chebzero = np.array([0])
00430 
00431 \textcolor{comment}{# Chebyshev coefficients representing one.}
\hypertarget{chebyshev_8py_source_l00432}{}\hyperlink{namespacepyneb_1_1utils_1_1chebyshev_a26a6cedd505a161858435461e1645503}{00432} chebone = np.array([1])
00433 
00434 \textcolor{comment}{# Chebyshev coefficients representing the identity x.}
\hypertarget{chebyshev_8py_source_l00435}{}\hyperlink{namespacepyneb_1_1utils_1_1chebyshev_a00f984b78341cd7920c46d1a6acacc5f}{00435} chebx = np.array([0,1])
00436 
\hypertarget{chebyshev_8py_source_l00437}{}\hyperlink{namespacepyneb_1_1utils_1_1chebyshev_ad14ae6340a5a9e3c93c1bb720f4e5a04}{00437} \textcolor{keyword}{def }\hyperlink{namespacepyneb_1_1utils_1_1chebyshev_ad14ae6340a5a9e3c93c1bb720f4e5a04}{chebline}(off, scl) :
00438     \textcolor{stringliteral}{"""}
00439 \textcolor{stringliteral}{    Chebyshev series whose graph is a straight line.}
00440 \textcolor{stringliteral}{}
00441 \textcolor{stringliteral}{}
00442 \textcolor{stringliteral}{}
00443 \textcolor{stringliteral}{    Parameters}
00444 \textcolor{stringliteral}{    ----------}
00445 \textcolor{stringliteral}{    off, scl : scalars}
00446 \textcolor{stringliteral}{        The specified line is given by ``off + scl*x``.}
00447 \textcolor{stringliteral}{}
00448 \textcolor{stringliteral}{    Returns}
00449 \textcolor{stringliteral}{    -------}
00450 \textcolor{stringliteral}{    y : ndarray}
00451 \textcolor{stringliteral}{        This module's representation of the Chebyshev series for}
00452 \textcolor{stringliteral}{        ``off + scl*x``.}
00453 \textcolor{stringliteral}{}
00454 \textcolor{stringliteral}{    See Also}
00455 \textcolor{stringliteral}{    --------}
00456 \textcolor{stringliteral}{    polyline}
00457 \textcolor{stringliteral}{}
00458 \textcolor{stringliteral}{    Examples}
00459 \textcolor{stringliteral}{    --------}
00460 \textcolor{stringliteral}{    >>> import numpy.polynomial.chebyshev as C}
00461 \textcolor{stringliteral}{    >>> C.chebline(3,2)}
00462 \textcolor{stringliteral}{    array([3, 2])}
00463 \textcolor{stringliteral}{    >>> C.chebval(-3, C.chebline(3,2)) # should be -3}
00464 \textcolor{stringliteral}{    -3.0}
00465 \textcolor{stringliteral}{}
00466 \textcolor{stringliteral}{    """}
00467     \textcolor{keywordflow}{if} scl != 0 :
00468         \textcolor{keywordflow}{return} np.array([off,scl])
00469     \textcolor{keywordflow}{else} :
00470         \textcolor{keywordflow}{return} np.array([off])
00471 
\hypertarget{chebyshev_8py_source_l00472}{}\hyperlink{namespacepyneb_1_1utils_1_1chebyshev_abedf5cf14ea9229321fc5e4043c42a0d}{00472} \textcolor{keyword}{def }\hyperlink{namespacepyneb_1_1utils_1_1chebyshev_abedf5cf14ea9229321fc5e4043c42a0d}{chebfromroots}(roots) :
00473     \textcolor{stringliteral}{"""}
00474 \textcolor{stringliteral}{    Generate a Chebyshev series with the given roots.}
00475 \textcolor{stringliteral}{}
00476 \textcolor{stringliteral}{    Return the array of coefficients for the C-series whose roots (a.k.a.}
00477 \textcolor{stringliteral}{    "zeros") are given by *roots*.  The returned array of coefficients is}
00478 \textcolor{stringliteral}{    ordered from lowest order "term" to highest, and zeros of multiplicity}
00479 \textcolor{stringliteral}{    greater than one must be included in *roots* a number of times equal}
00480 \textcolor{stringliteral}{    to their multiplicity (e.g., if `2` is a root of multiplicity three,}
00481 \textcolor{stringliteral}{    then [2,2,2] must be in *roots*).}
00482 \textcolor{stringliteral}{}
00483 \textcolor{stringliteral}{    Parameters}
00484 \textcolor{stringliteral}{    ----------}
00485 \textcolor{stringliteral}{    roots : array\_like}
00486 \textcolor{stringliteral}{        Sequence containing the roots.}
00487 \textcolor{stringliteral}{}
00488 \textcolor{stringliteral}{    Returns}
00489 \textcolor{stringliteral}{    -------}
00490 \textcolor{stringliteral}{    out : ndarray}
00491 \textcolor{stringliteral}{        1-d array of the C-series' coefficients, ordered from low to}
00492 \textcolor{stringliteral}{        high.  If all roots are real, ``out.dtype`` is a float type;}
00493 \textcolor{stringliteral}{        otherwise, ``out.dtype`` is a complex type, even if all the}
00494 \textcolor{stringliteral}{        coefficients in the result are real (see Examples below).}
00495 \textcolor{stringliteral}{}
00496 \textcolor{stringliteral}{    See Also}
00497 \textcolor{stringliteral}{    --------}
00498 \textcolor{stringliteral}{    polyfromroots}
00499 \textcolor{stringliteral}{}
00500 \textcolor{stringliteral}{    Notes}
00501 \textcolor{stringliteral}{    -----}
00502 \textcolor{stringliteral}{    What is returned are the :math:`c\_i` such that:}
00503 \textcolor{stringliteral}{}
00504 \textcolor{stringliteral}{    .. math::}
00505 \textcolor{stringliteral}{}
00506 \textcolor{stringliteral}{        \(\backslash\)\(\backslash\)sum\_\{i=0\}^\{n\} c\_i*T\_i(x) = \(\backslash\)\(\backslash\)prod\_\{i=0\}^\{n\} (x - roots[i])}
00507 \textcolor{stringliteral}{}
00508 \textcolor{stringliteral}{    where ``n == len(roots)`` and :math:`T\_i(x)` is the `i`-th Chebyshev}
00509 \textcolor{stringliteral}{    (basis) polynomial over the domain `[-1,1]`.  Note that, unlike}
00510 \textcolor{stringliteral}{    `polyfromroots`, due to the nature of the C-series basis set, the}
00511 \textcolor{stringliteral}{    above identity *does not* imply :math:`c\_n = 1` identically (see}
00512 \textcolor{stringliteral}{    Examples).}
00513 \textcolor{stringliteral}{}
00514 \textcolor{stringliteral}{    Examples}
00515 \textcolor{stringliteral}{    --------}
00516 \textcolor{stringliteral}{    >>> import numpy.polynomial.chebyshev as C}
00517 \textcolor{stringliteral}{    >>> C.chebfromroots((-1,0,1)) # x^3 - x relative to the standard basis}
00518 \textcolor{stringliteral}{    array([ 0.  , -0.25,  0.  ,  0.25])}
00519 \textcolor{stringliteral}{    >>> j = complex(0,1)}
00520 \textcolor{stringliteral}{    >>> C.chebfromroots((-j,j)) # x^2 + 1 relative to the standard basis}
00521 \textcolor{stringliteral}{    array([ 1.5+0.j,  0.0+0.j,  0.5+0.j])}
00522 \textcolor{stringliteral}{}
00523 \textcolor{stringliteral}{    """}
00524     \textcolor{keywordflow}{if} len(roots) == 0 :
00525         \textcolor{keywordflow}{return} np.ones(1)
00526     \textcolor{keywordflow}{else} :
00527         [roots] = pu.as\_series([roots], trim=\textcolor{keyword}{False})
00528         prd = np.array([1], dtype=roots.dtype)
00529         \textcolor{keywordflow}{for} r \textcolor{keywordflow}{in} roots:
00530             prd = \hyperlink{namespacepyneb_1_1utils_1_1chebyshev_ac07c70b217882a8c0a731ce9fdf743f0}{chebsub}(\hyperlink{namespacepyneb_1_1utils_1_1chebyshev_ab09f8accee50bd964e03071a659a7b5f}{chebmulx}(prd), r*prd)
00531         \textcolor{keywordflow}{return} prd
00532 
00533 
\hypertarget{chebyshev_8py_source_l00534}{}\hyperlink{namespacepyneb_1_1utils_1_1chebyshev_a06a85eb72ae35925336a6b550dec3f02}{00534} \textcolor{keyword}{def }\hyperlink{namespacepyneb_1_1utils_1_1chebyshev_a06a85eb72ae35925336a6b550dec3f02}{chebadd}(c1, c2):
00535     \textcolor{stringliteral}{"""}
00536 \textcolor{stringliteral}{    Add one Chebyshev series to another.}
00537 \textcolor{stringliteral}{}
00538 \textcolor{stringliteral}{    Returns the sum of two Chebyshev series `c1` + `c2`.  The arguments}
00539 \textcolor{stringliteral}{    are sequences of coefficients ordered from lowest order term to}
00540 \textcolor{stringliteral}{    highest, i.e., [1,2,3] represents the series ``T\_0 + 2*T\_1 + 3*T\_2``.}
00541 \textcolor{stringliteral}{}
00542 \textcolor{stringliteral}{    Parameters}
00543 \textcolor{stringliteral}{    ----------}
00544 \textcolor{stringliteral}{    c1, c2 : array\_like}
00545 \textcolor{stringliteral}{        1-d arrays of Chebyshev series coefficients ordered from low to}
00546 \textcolor{stringliteral}{        high.}
00547 \textcolor{stringliteral}{}
00548 \textcolor{stringliteral}{    Returns}
00549 \textcolor{stringliteral}{    -------}
00550 \textcolor{stringliteral}{    out : ndarray}
00551 \textcolor{stringliteral}{        Array representing the Chebyshev series of their sum.}
00552 \textcolor{stringliteral}{}
00553 \textcolor{stringliteral}{    See Also}
00554 \textcolor{stringliteral}{    --------}
00555 \textcolor{stringliteral}{    chebsub, chebmul, chebdiv, chebpow}
00556 \textcolor{stringliteral}{}
00557 \textcolor{stringliteral}{    Notes}
00558 \textcolor{stringliteral}{    -----}
00559 \textcolor{stringliteral}{    Unlike multiplication, division, etc., the sum of two Chebyshev series}
00560 \textcolor{stringliteral}{    is a Chebyshev series (without having to "reproject" the result onto}
00561 \textcolor{stringliteral}{    the basis set) so addition, just like that of "standard" polynomials,}
00562 \textcolor{stringliteral}{    is simply "component-wise."}
00563 \textcolor{stringliteral}{}
00564 \textcolor{stringliteral}{    Examples}
00565 \textcolor{stringliteral}{    --------}
00566 \textcolor{stringliteral}{    >>> from numpy.polynomial import chebyshev as C}
00567 \textcolor{stringliteral}{    >>> c1 = (1,2,3)}
00568 \textcolor{stringliteral}{    >>> c2 = (3,2,1)}
00569 \textcolor{stringliteral}{    >>> C.chebadd(c1,c2)}
00570 \textcolor{stringliteral}{    array([ 4.,  4.,  4.])}
00571 \textcolor{stringliteral}{}
00572 \textcolor{stringliteral}{    """}
00573     \textcolor{comment}{# c1, c2 are trimmed copies}
00574     [c1, c2] = pu.as\_series([c1, c2])
00575     \textcolor{keywordflow}{if} len(c1) > len(c2) :
00576         c1[:c2.size] += c2
00577         ret = c1
00578     \textcolor{keywordflow}{else} :
00579         c2[:c1.size] += c1
00580         ret = c2
00581     \textcolor{keywordflow}{return} pu.trimseq(ret)
00582 
00583 
\hypertarget{chebyshev_8py_source_l00584}{}\hyperlink{namespacepyneb_1_1utils_1_1chebyshev_ac07c70b217882a8c0a731ce9fdf743f0}{00584} \textcolor{keyword}{def }\hyperlink{namespacepyneb_1_1utils_1_1chebyshev_ac07c70b217882a8c0a731ce9fdf743f0}{chebsub}(c1, c2):
00585     \textcolor{stringliteral}{"""}
00586 \textcolor{stringliteral}{    Subtract one Chebyshev series from another.}
00587 \textcolor{stringliteral}{}
00588 \textcolor{stringliteral}{    Returns the difference of two Chebyshev series `c1` - `c2`.  The}
00589 \textcolor{stringliteral}{    sequences of coefficients are from lowest order term to highest, i.e.,}
00590 \textcolor{stringliteral}{    [1,2,3] represents the series ``T\_0 + 2*T\_1 + 3*T\_2``.}
00591 \textcolor{stringliteral}{}
00592 \textcolor{stringliteral}{    Parameters}
00593 \textcolor{stringliteral}{    ----------}
00594 \textcolor{stringliteral}{    c1, c2 : array\_like}
00595 \textcolor{stringliteral}{        1-d arrays of Chebyshev series coefficients ordered from low to}
00596 \textcolor{stringliteral}{        high.}
00597 \textcolor{stringliteral}{}
00598 \textcolor{stringliteral}{    Returns}
00599 \textcolor{stringliteral}{    -------}
00600 \textcolor{stringliteral}{    out : ndarray}
00601 \textcolor{stringliteral}{        Of Chebyshev series coefficients representing their difference.}
00602 \textcolor{stringliteral}{}
00603 \textcolor{stringliteral}{    See Also}
00604 \textcolor{stringliteral}{    --------}
00605 \textcolor{stringliteral}{    chebadd, chebmul, chebdiv, chebpow}
00606 \textcolor{stringliteral}{}
00607 \textcolor{stringliteral}{    Notes}
00608 \textcolor{stringliteral}{    -----}
00609 \textcolor{stringliteral}{    Unlike multiplication, division, etc., the difference of two Chebyshev}
00610 \textcolor{stringliteral}{    series is a Chebyshev series (without having to "reproject" the result}
00611 \textcolor{stringliteral}{    onto the basis set) so subtraction, just like that of "standard"}
00612 \textcolor{stringliteral}{    polynomials, is simply "component-wise."}
00613 \textcolor{stringliteral}{}
00614 \textcolor{stringliteral}{    Examples}
00615 \textcolor{stringliteral}{    --------}
00616 \textcolor{stringliteral}{    >>> from numpy.polynomial import chebyshev as C}
00617 \textcolor{stringliteral}{    >>> c1 = (1,2,3)}
00618 \textcolor{stringliteral}{    >>> c2 = (3,2,1)}
00619 \textcolor{stringliteral}{    >>> C.chebsub(c1,c2)}
00620 \textcolor{stringliteral}{    array([-2.,  0.,  2.])}
00621 \textcolor{stringliteral}{    >>> C.chebsub(c2,c1) # -C.chebsub(c1,c2)}
00622 \textcolor{stringliteral}{    array([ 2.,  0., -2.])}
00623 \textcolor{stringliteral}{}
00624 \textcolor{stringliteral}{    """}
00625     \textcolor{comment}{# c1, c2 are trimmed copies}
00626     [c1, c2] = pu.as\_series([c1, c2])
00627     \textcolor{keywordflow}{if} len(c1) > len(c2) :
00628         c1[:c2.size] -= c2
00629         ret = c1
00630     \textcolor{keywordflow}{else} :
00631         c2 = -c2
00632         c2[:c1.size] += c1
00633         ret = c2
00634     \textcolor{keywordflow}{return} pu.trimseq(ret)
00635 
00636 
\hypertarget{chebyshev_8py_source_l00637}{}\hyperlink{namespacepyneb_1_1utils_1_1chebyshev_ab09f8accee50bd964e03071a659a7b5f}{00637} \textcolor{keyword}{def }\hyperlink{namespacepyneb_1_1utils_1_1chebyshev_ab09f8accee50bd964e03071a659a7b5f}{chebmulx}(cs):
00638     \textcolor{stringliteral}{"""Multiply a Chebyshev series by x.}
00639 \textcolor{stringliteral}{}
00640 \textcolor{stringliteral}{    Multiply the polynomial `cs` by x, where x is the independent}
00641 \textcolor{stringliteral}{    variable.}
00642 \textcolor{stringliteral}{}
00643 \textcolor{stringliteral}{}
00644 \textcolor{stringliteral}{    Parameters}
00645 \textcolor{stringliteral}{    ----------}
00646 \textcolor{stringliteral}{    cs : array\_like}
00647 \textcolor{stringliteral}{        1-d array of Chebyshev series coefficients ordered from low to}
00648 \textcolor{stringliteral}{        high.}
00649 \textcolor{stringliteral}{}
00650 \textcolor{stringliteral}{    Returns}
00651 \textcolor{stringliteral}{    -------}
00652 \textcolor{stringliteral}{    out : ndarray}
00653 \textcolor{stringliteral}{        Array representing the result of the multiplication.}
00654 \textcolor{stringliteral}{}
00655 \textcolor{stringliteral}{    Notes}
00656 \textcolor{stringliteral}{    -----}
00657 \textcolor{stringliteral}{    .. versionadded:: 1.5.0}
00658 \textcolor{stringliteral}{}
00659 \textcolor{stringliteral}{    """}
00660     \textcolor{comment}{# cs is a trimmed copy}
00661     [cs] = pu.as\_series([cs])
00662     \textcolor{comment}{# The zero series needs special treatment}
00663     \textcolor{keywordflow}{if} len(cs) == 1 \textcolor{keywordflow}{and} cs[0] == 0:
00664         \textcolor{keywordflow}{return} cs
00665 
00666     prd = np.empty(len(cs) + 1, dtype=cs.dtype)
00667     prd[0] = cs[0]*0
00668     prd[1] = cs[0]
00669     \textcolor{keywordflow}{if} len(cs) > 1:
00670         tmp = cs[1:]/2
00671         prd[2:] = tmp
00672         prd[0:-2] += tmp
00673     \textcolor{keywordflow}{return} prd
00674 
00675 
\hypertarget{chebyshev_8py_source_l00676}{}\hyperlink{namespacepyneb_1_1utils_1_1chebyshev_aca5aab8be2c47559d4489fd1dac11f75}{00676} \textcolor{keyword}{def }\hyperlink{namespacepyneb_1_1utils_1_1chebyshev_aca5aab8be2c47559d4489fd1dac11f75}{chebmul}(c1, c2):
00677     \textcolor{stringliteral}{"""}
00678 \textcolor{stringliteral}{    Multiply one Chebyshev series by another.}
00679 \textcolor{stringliteral}{}
00680 \textcolor{stringliteral}{    Returns the product of two Chebyshev series `c1` * `c2`.  The arguments}
00681 \textcolor{stringliteral}{    are sequences of coefficients, from lowest order "term" to highest,}
00682 \textcolor{stringliteral}{    e.g., [1,2,3] represents the series ``T\_0 + 2*T\_1 + 3*T\_2``.}
00683 \textcolor{stringliteral}{}
00684 \textcolor{stringliteral}{    Parameters}
00685 \textcolor{stringliteral}{    ----------}
00686 \textcolor{stringliteral}{    c1, c2 : array\_like}
00687 \textcolor{stringliteral}{        1-d arrays of Chebyshev series coefficients ordered from low to}
00688 \textcolor{stringliteral}{        high.}
00689 \textcolor{stringliteral}{}
00690 \textcolor{stringliteral}{    Returns}
00691 \textcolor{stringliteral}{    -------}
00692 \textcolor{stringliteral}{    out : ndarray}
00693 \textcolor{stringliteral}{        Of Chebyshev series coefficients representing their product.}
00694 \textcolor{stringliteral}{}
00695 \textcolor{stringliteral}{    See Also}
00696 \textcolor{stringliteral}{    --------}
00697 \textcolor{stringliteral}{    chebadd, chebsub, chebdiv, chebpow}
00698 \textcolor{stringliteral}{}
00699 \textcolor{stringliteral}{    Notes}
00700 \textcolor{stringliteral}{    -----}
00701 \textcolor{stringliteral}{    In general, the (polynomial) product of two C-series results in terms}
00702 \textcolor{stringliteral}{    that are not in the Chebyshev polynomial basis set.  Thus, to express}
00703 \textcolor{stringliteral}{    the product as a C-series, it is typically necessary to "re-project"}
00704 \textcolor{stringliteral}{    the product onto said basis set, which typically produces}
00705 \textcolor{stringliteral}{    "un-intuitive" (but correct) results; see Examples section below.}
00706 \textcolor{stringliteral}{}
00707 \textcolor{stringliteral}{    Examples}
00708 \textcolor{stringliteral}{    --------}
00709 \textcolor{stringliteral}{    >>> from numpy.polynomial import chebyshev as C}
00710 \textcolor{stringliteral}{    >>> c1 = (1,2,3)}
00711 \textcolor{stringliteral}{    >>> c2 = (3,2,1)}
00712 \textcolor{stringliteral}{    >>> C.chebmul(c1,c2) # multiplication requires "reprojection"}
00713 \textcolor{stringliteral}{    array([  6.5,  12. ,  12. ,   4. ,   1.5])}
00714 \textcolor{stringliteral}{}
00715 \textcolor{stringliteral}{    """}
00716     \textcolor{comment}{# c1, c2 are trimmed copies}
00717     [c1, c2] = pu.as\_series([c1, c2])
00718     z1 = \hyperlink{namespacepyneb_1_1utils_1_1chebyshev_a659e346f7cdd9fd058850b26f7e95b17}{\_cseries\_to\_zseries}(c1)
00719     z2 = \hyperlink{namespacepyneb_1_1utils_1_1chebyshev_a659e346f7cdd9fd058850b26f7e95b17}{\_cseries\_to\_zseries}(c2)
00720     prd = \hyperlink{namespacepyneb_1_1utils_1_1chebyshev_a46d958933482afac66ea41d143155dae}{\_zseries\_mul}(z1, z2)
00721     ret = \hyperlink{namespacepyneb_1_1utils_1_1chebyshev_a2a88474ce0ea12fb26f82b7116752dc1}{\_zseries\_to\_cseries}(prd)
00722     \textcolor{keywordflow}{return} pu.trimseq(ret)
00723 
00724 
\hypertarget{chebyshev_8py_source_l00725}{}\hyperlink{namespacepyneb_1_1utils_1_1chebyshev_aa84df28815f29342a7cdce9b95470f02}{00725} \textcolor{keyword}{def }\hyperlink{namespacepyneb_1_1utils_1_1chebyshev_aa84df28815f29342a7cdce9b95470f02}{chebdiv}(c1, c2):
00726     \textcolor{stringliteral}{"""}
00727 \textcolor{stringliteral}{    Divide one Chebyshev series by another.}
00728 \textcolor{stringliteral}{}
00729 \textcolor{stringliteral}{    Returns the quotient-with-remainder of two Chebyshev series}
00730 \textcolor{stringliteral}{    `c1` / `c2`.  The arguments are sequences of coefficients from lowest}
00731 \textcolor{stringliteral}{    order "term" to highest, e.g., [1,2,3] represents the series}
00732 \textcolor{stringliteral}{    ``T\_0 + 2*T\_1 + 3*T\_2``.}
00733 \textcolor{stringliteral}{}
00734 \textcolor{stringliteral}{    Parameters}
00735 \textcolor{stringliteral}{    ----------}
00736 \textcolor{stringliteral}{    c1, c2 : array\_like}
00737 \textcolor{stringliteral}{        1-d arrays of Chebyshev series coefficients ordered from low to}
00738 \textcolor{stringliteral}{        high.}
00739 \textcolor{stringliteral}{}
00740 \textcolor{stringliteral}{    Returns}
00741 \textcolor{stringliteral}{    -------}
00742 \textcolor{stringliteral}{    [quo, rem] : ndarrays}
00743 \textcolor{stringliteral}{        Of Chebyshev series coefficients representing the quotient and}
00744 \textcolor{stringliteral}{        remainder.}
00745 \textcolor{stringliteral}{}
00746 \textcolor{stringliteral}{    See Also}
00747 \textcolor{stringliteral}{    --------}
00748 \textcolor{stringliteral}{    chebadd, chebsub, chebmul, chebpow}
00749 \textcolor{stringliteral}{}
00750 \textcolor{stringliteral}{    Notes}
00751 \textcolor{stringliteral}{    -----}
00752 \textcolor{stringliteral}{    In general, the (polynomial) division of one C-series by another}
00753 \textcolor{stringliteral}{    results in quotient and remainder terms that are not in the Chebyshev}
00754 \textcolor{stringliteral}{    polynomial basis set.  Thus, to express these results as C-series, it}
00755 \textcolor{stringliteral}{    is typically necessary to "re-project" the results onto said basis}
00756 \textcolor{stringliteral}{    set, which typically produces "un-intuitive" (but correct) results;}
00757 \textcolor{stringliteral}{    see Examples section below.}
00758 \textcolor{stringliteral}{}
00759 \textcolor{stringliteral}{    Examples}
00760 \textcolor{stringliteral}{    --------}
00761 \textcolor{stringliteral}{    >>> from numpy.polynomial import chebyshev as C}
00762 \textcolor{stringliteral}{    >>> c1 = (1,2,3)}
00763 \textcolor{stringliteral}{    >>> c2 = (3,2,1)}
00764 \textcolor{stringliteral}{    >>> C.chebdiv(c1,c2) # quotient "intuitive," remainder not}
00765 \textcolor{stringliteral}{    (array([ 3.]), array([-8., -4.]))}
00766 \textcolor{stringliteral}{    >>> c2 = (0,1,2,3)}
00767 \textcolor{stringliteral}{    >>> C.chebdiv(c2,c1) # neither "intuitive"}
00768 \textcolor{stringliteral}{    (array([ 0.,  2.]), array([-2., -4.]))}
00769 \textcolor{stringliteral}{}
00770 \textcolor{stringliteral}{    """}
00771     \textcolor{comment}{# c1, c2 are trimmed copies}
00772     [c1, c2] = pu.as\_series([c1, c2])
00773     \textcolor{keywordflow}{if} c2[-1] == 0 :
00774         \textcolor{keywordflow}{raise} ZeroDivisionError()
00775 
00776     lc1 = len(c1)
00777     lc2 = len(c2)
00778     \textcolor{keywordflow}{if} lc1 < lc2 :
00779         \textcolor{keywordflow}{return} c1[:1]*0, c1
00780     \textcolor{keywordflow}{elif} lc2 == 1 :
00781         \textcolor{keywordflow}{return} c1/c2[-1], c1[:1]*0
00782     \textcolor{keywordflow}{else} :
00783         z1 = \hyperlink{namespacepyneb_1_1utils_1_1chebyshev_a659e346f7cdd9fd058850b26f7e95b17}{\_cseries\_to\_zseries}(c1)
00784         z2 = \hyperlink{namespacepyneb_1_1utils_1_1chebyshev_a659e346f7cdd9fd058850b26f7e95b17}{\_cseries\_to\_zseries}(c2)
00785         quo, rem = \hyperlink{namespacepyneb_1_1utils_1_1chebyshev_ad00f2653194746fa98c3421eb7c26f2c}{\_zseries\_div}(z1, z2)
00786         quo = pu.trimseq(\hyperlink{namespacepyneb_1_1utils_1_1chebyshev_a2a88474ce0ea12fb26f82b7116752dc1}{\_zseries\_to\_cseries}(quo))
00787         rem = pu.trimseq(\hyperlink{namespacepyneb_1_1utils_1_1chebyshev_a2a88474ce0ea12fb26f82b7116752dc1}{\_zseries\_to\_cseries}(rem))
00788         \textcolor{keywordflow}{return} quo, rem
00789 
\hypertarget{chebyshev_8py_source_l00790}{}\hyperlink{namespacepyneb_1_1utils_1_1chebyshev_ab647ff3aa7750a09b934b98cddd4c90d}{00790} \textcolor{keyword}{def }\hyperlink{namespacepyneb_1_1utils_1_1chebyshev_ab647ff3aa7750a09b934b98cddd4c90d}{chebpow}(cs, pow, maxpower=16) :
00791     \textcolor{stringliteral}{"""Raise a Chebyshev series to a power.}
00792 \textcolor{stringliteral}{}
00793 \textcolor{stringliteral}{    Returns the Chebyshev series `cs` raised to the power `pow`. The}
00794 \textcolor{stringliteral}{    arguement `cs` is a sequence of coefficients ordered from low to high.}
00795 \textcolor{stringliteral}{    i.e., [1,2,3] is the series  ``T\_0 + 2*T\_1 + 3*T\_2.``}
00796 \textcolor{stringliteral}{}
00797 \textcolor{stringliteral}{    Parameters}
00798 \textcolor{stringliteral}{    ----------}
00799 \textcolor{stringliteral}{    cs : array\_like}
00800 \textcolor{stringliteral}{        1d array of chebyshev series coefficients ordered from low to}
00801 \textcolor{stringliteral}{        high.}
00802 \textcolor{stringliteral}{    pow : integer}
00803 \textcolor{stringliteral}{        Power to which the series will be raised}
00804 \textcolor{stringliteral}{    maxpower : integer, optional}
00805 \textcolor{stringliteral}{        Maximum power allowed. This is mainly to limit growth of the series}
00806 \textcolor{stringliteral}{        to umanageable size. Default is 16}
00807 \textcolor{stringliteral}{}
00808 \textcolor{stringliteral}{    Returns}
00809 \textcolor{stringliteral}{    -------}
00810 \textcolor{stringliteral}{    coef : ndarray}
00811 \textcolor{stringliteral}{        Chebyshev series of power.}
00812 \textcolor{stringliteral}{}
00813 \textcolor{stringliteral}{    See Also}
00814 \textcolor{stringliteral}{    --------}
00815 \textcolor{stringliteral}{    chebadd, chebsub, chebmul, chebdiv}
00816 \textcolor{stringliteral}{}
00817 \textcolor{stringliteral}{    Examples}
00818 \textcolor{stringliteral}{    --------}
00819 \textcolor{stringliteral}{}
00820 \textcolor{stringliteral}{    """}
00821     \textcolor{comment}{# cs is a trimmed copy}
00822     [cs] = pu.as\_series([cs])
00823     power = int(pow)
00824     \textcolor{keywordflow}{if} power != pow \textcolor{keywordflow}{or} power < 0 :
00825         \textcolor{keywordflow}{raise} ValueError(\textcolor{stringliteral}{"Power must be a non-negative integer."})
00826     \textcolor{keywordflow}{elif} maxpower \textcolor{keywordflow}{is} \textcolor{keywordflow}{not} \textcolor{keywordtype}{None} \textcolor{keywordflow}{and} power > maxpower :
00827         \textcolor{keywordflow}{raise} ValueError(\textcolor{stringliteral}{"Power is too large"})
00828     \textcolor{keywordflow}{elif} power == 0 :
00829         \textcolor{keywordflow}{return} np.array([1], dtype=cs.dtype)
00830     \textcolor{keywordflow}{elif} power == 1 :
00831         \textcolor{keywordflow}{return} cs
00832     \textcolor{keywordflow}{else} :
00833         \textcolor{comment}{# This can be made more efficient by using powers of two}
00834         \textcolor{comment}{# in the usual way.}
00835         zs = \hyperlink{namespacepyneb_1_1utils_1_1chebyshev_a659e346f7cdd9fd058850b26f7e95b17}{\_cseries\_to\_zseries}(cs)
00836         prd = zs
00837         \textcolor{keywordflow}{for} i \textcolor{keywordflow}{in} range(2, power + 1) :
00838             prd = np.convolve(prd, zs)
00839         \textcolor{keywordflow}{return} \hyperlink{namespacepyneb_1_1utils_1_1chebyshev_a2a88474ce0ea12fb26f82b7116752dc1}{\_zseries\_to\_cseries}(prd)
00840 
\hypertarget{chebyshev_8py_source_l00841}{}\hyperlink{namespacepyneb_1_1utils_1_1chebyshev_a1ed2b9248c247449584904025ec9d460}{00841} \textcolor{keyword}{def }\hyperlink{namespacepyneb_1_1utils_1_1chebyshev_a1ed2b9248c247449584904025ec9d460}{chebder}(cs, m=1, scl=1) :
00842     \textcolor{stringliteral}{"""}
00843 \textcolor{stringliteral}{    Differentiate a Chebyshev series.}
00844 \textcolor{stringliteral}{}
00845 \textcolor{stringliteral}{    Returns the series `cs` differentiated `m` times.  At each iteration the}
00846 \textcolor{stringliteral}{    result is multiplied by `scl` (the scaling factor is for use in a linear}
00847 \textcolor{stringliteral}{    change of variable).  The argument `cs` is the sequence of coefficients}
00848 \textcolor{stringliteral}{    from lowest order "term" to highest, e.g., [1,2,3] represents the series}
00849 \textcolor{stringliteral}{    ``T\_0 + 2*T\_1 + 3*T\_2``.}
00850 \textcolor{stringliteral}{}
00851 \textcolor{stringliteral}{    Parameters}
00852 \textcolor{stringliteral}{    ----------}
00853 \textcolor{stringliteral}{    cs: array\_like}
00854 \textcolor{stringliteral}{        1-d array of Chebyshev series coefficients ordered from low to high.}
00855 \textcolor{stringliteral}{    m : int, optional}
00856 \textcolor{stringliteral}{        Number of derivatives taken, must be non-negative. (Default: 1)}
00857 \textcolor{stringliteral}{    scl : scalar, optional}
00858 \textcolor{stringliteral}{        Each differentiation is multiplied by `scl`.  The end result is}
00859 \textcolor{stringliteral}{        multiplication by ``scl**m``.  This is for use in a linear change of}
00860 \textcolor{stringliteral}{        variable. (Default: 1)}
00861 \textcolor{stringliteral}{}
00862 \textcolor{stringliteral}{    Returns}
00863 \textcolor{stringliteral}{    -------}
00864 \textcolor{stringliteral}{    der : ndarray}
00865 \textcolor{stringliteral}{        Chebyshev series of the derivative.}
00866 \textcolor{stringliteral}{}
00867 \textcolor{stringliteral}{    See Also}
00868 \textcolor{stringliteral}{    --------}
00869 \textcolor{stringliteral}{    chebint}
00870 \textcolor{stringliteral}{}
00871 \textcolor{stringliteral}{    Notes}
00872 \textcolor{stringliteral}{    -----}
00873 \textcolor{stringliteral}{    In general, the result of differentiating a C-series needs to be}
00874 \textcolor{stringliteral}{    "re-projected" onto the C-series basis set. Thus, typically, the}
00875 \textcolor{stringliteral}{    result of this function is "un-intuitive," albeit correct; see Examples}
00876 \textcolor{stringliteral}{    section below.}
00877 \textcolor{stringliteral}{}
00878 \textcolor{stringliteral}{    Examples}
00879 \textcolor{stringliteral}{    --------}
00880 \textcolor{stringliteral}{    >>> from numpy.polynomial import chebyshev as C}
00881 \textcolor{stringliteral}{    >>> cs = (1,2,3,4)}
00882 \textcolor{stringliteral}{    >>> C.chebder(cs)}
00883 \textcolor{stringliteral}{    array([ 14.,  12.,  24.])}
00884 \textcolor{stringliteral}{    >>> C.chebder(cs,3)}
00885 \textcolor{stringliteral}{    array([ 96.])}
00886 \textcolor{stringliteral}{    >>> C.chebder(cs,scl=-1)}
00887 \textcolor{stringliteral}{    array([-14., -12., -24.])}
00888 \textcolor{stringliteral}{    >>> C.chebder(cs,2,-1)}
00889 \textcolor{stringliteral}{    array([ 12.,  96.])}
00890 \textcolor{stringliteral}{}
00891 \textcolor{stringliteral}{    """}
00892     cnt = int(m)
00893 
00894     \textcolor{keywordflow}{if} cnt != m:
00895         \textcolor{keywordflow}{raise} ValueError, \textcolor{stringliteral}{"The order of derivation must be integer"}
00896     \textcolor{keywordflow}{if} cnt < 0 :
00897         \textcolor{keywordflow}{raise} ValueError, \textcolor{stringliteral}{"The order of derivation must be non-negative"}
00898 
00899     \textcolor{comment}{# cs is a trimmed copy}
00900     [cs] = pu.as\_series([cs])
00901     \textcolor{keywordflow}{if} cnt == 0:
00902         \textcolor{keywordflow}{return} cs
00903     \textcolor{keywordflow}{elif} cnt >= len(cs):
00904         \textcolor{keywordflow}{return} cs[:1]*0
00905     \textcolor{keywordflow}{else} :
00906         zs = \hyperlink{namespacepyneb_1_1utils_1_1chebyshev_a659e346f7cdd9fd058850b26f7e95b17}{\_cseries\_to\_zseries}(cs)
00907         \textcolor{keywordflow}{for} i \textcolor{keywordflow}{in} range(cnt):
00908             zs = \hyperlink{namespacepyneb_1_1utils_1_1chebyshev_adb39b6e9d3aba3bf3784a3364ec3e6ca}{\_zseries\_der}(zs)*scl
00909         \textcolor{keywordflow}{return} \hyperlink{namespacepyneb_1_1utils_1_1chebyshev_a2a88474ce0ea12fb26f82b7116752dc1}{\_zseries\_to\_cseries}(zs)
00910 
00911 
\hypertarget{chebyshev_8py_source_l00912}{}\hyperlink{namespacepyneb_1_1utils_1_1chebyshev_a233e686a07b9935c74ce1e053f0c5f5f}{00912} \textcolor{keyword}{def }\hyperlink{namespacepyneb_1_1utils_1_1chebyshev_a233e686a07b9935c74ce1e053f0c5f5f}{chebint}(cs, m=1, k=[], lbnd=0, scl=1):
00913     \textcolor{stringliteral}{"""}
00914 \textcolor{stringliteral}{    Integrate a Chebyshev series.}
00915 \textcolor{stringliteral}{}
00916 \textcolor{stringliteral}{    Returns, as a C-series, the input C-series `cs`, integrated `m` times}
00917 \textcolor{stringliteral}{    from `lbnd` to `x`.  At each iteration the resulting series is}
00918 \textcolor{stringliteral}{    **multiplied** by `scl` and an integration constant, `k`, is added.}
00919 \textcolor{stringliteral}{    The scaling factor is for use in a linear change of variable.  ("Buyer}
00920 \textcolor{stringliteral}{    beware": note that, depending on what one is doing, one may want `scl`}
00921 \textcolor{stringliteral}{    to be the reciprocal of what one might expect; for more information,}
00922 \textcolor{stringliteral}{    see the Notes section below.)  The argument `cs` is a sequence of}
00923 \textcolor{stringliteral}{    coefficients, from lowest order C-series "term" to highest, e.g.,}
00924 \textcolor{stringliteral}{    [1,2,3] represents the series :math:`T\_0(x) + 2T\_1(x) + 3T\_2(x)`.}
00925 \textcolor{stringliteral}{}
00926 \textcolor{stringliteral}{    Parameters}
00927 \textcolor{stringliteral}{    ----------}
00928 \textcolor{stringliteral}{    cs : array\_like}
00929 \textcolor{stringliteral}{        1-d array of C-series coefficients, ordered from low to high.}
00930 \textcolor{stringliteral}{    m : int, optional}
00931 \textcolor{stringliteral}{        Order of integration, must be positive. (Default: 1)}
00932 \textcolor{stringliteral}{    k : \{[], list, scalar\}, optional}
00933 \textcolor{stringliteral}{        Integration constant(s).  The value of the first integral at zero}
00934 \textcolor{stringliteral}{        is the first value in the list, the value of the second integral}
00935 \textcolor{stringliteral}{        at zero is the second value, etc.  If ``k == []`` (the default),}
00936 \textcolor{stringliteral}{        all constants are set to zero.  If ``m == 1``, a single scalar can}
00937 \textcolor{stringliteral}{        be given instead of a list.}
00938 \textcolor{stringliteral}{    lbnd : scalar, optional}
00939 \textcolor{stringliteral}{        The lower bound of the integral. (Default: 0)}
00940 \textcolor{stringliteral}{    scl : scalar, optional}
00941 \textcolor{stringliteral}{        Following each integration the result is *multiplied* by `scl`}
00942 \textcolor{stringliteral}{        before the integration constant is added. (Default: 1)}
00943 \textcolor{stringliteral}{}
00944 \textcolor{stringliteral}{    Returns}
00945 \textcolor{stringliteral}{    -------}
00946 \textcolor{stringliteral}{    S : ndarray}
00947 \textcolor{stringliteral}{        C-series coefficients of the integral.}
00948 \textcolor{stringliteral}{}
00949 \textcolor{stringliteral}{    Raises}
00950 \textcolor{stringliteral}{    ------}
00951 \textcolor{stringliteral}{    ValueError}
00952 \textcolor{stringliteral}{        If ``m < 1``, ``len(k) > m``, ``np.isscalar(lbnd) == False``, or}
00953 \textcolor{stringliteral}{        ``np.isscalar(scl) == False``.}
00954 \textcolor{stringliteral}{}
00955 \textcolor{stringliteral}{    See Also}
00956 \textcolor{stringliteral}{    --------}
00957 \textcolor{stringliteral}{    chebder}
00958 \textcolor{stringliteral}{}
00959 \textcolor{stringliteral}{    Notes}
00960 \textcolor{stringliteral}{    -----}
00961 \textcolor{stringliteral}{    Note that the result of each integration is *multiplied* by `scl`.}
00962 \textcolor{stringliteral}{    Why is this important to note?  Say one is making a linear change of}
00963 \textcolor{stringliteral}{    variable :math:`u = ax + b` in an integral relative to `x`.  Then}
00964 \textcolor{stringliteral}{    :math:`dx = du/a`, so one will need to set `scl` equal to :math:`1/a`}
00965 \textcolor{stringliteral}{    - perhaps not what one would have first thought.}
00966 \textcolor{stringliteral}{}
00967 \textcolor{stringliteral}{    Also note that, in general, the result of integrating a C-series needs}
00968 \textcolor{stringliteral}{    to be "re-projected" onto the C-series basis set.  Thus, typically,}
00969 \textcolor{stringliteral}{    the result of this function is "un-intuitive," albeit correct; see}
00970 \textcolor{stringliteral}{    Examples section below.}
00971 \textcolor{stringliteral}{}
00972 \textcolor{stringliteral}{    Examples}
00973 \textcolor{stringliteral}{    --------}
00974 \textcolor{stringliteral}{    >>> from numpy.polynomial import chebyshev as C}
00975 \textcolor{stringliteral}{    >>> cs = (1,2,3)}
00976 \textcolor{stringliteral}{    >>> C.chebint(cs)}
00977 \textcolor{stringliteral}{    array([ 0.5, -0.5,  0.5,  0.5])}
00978 \textcolor{stringliteral}{    >>> C.chebint(cs,3)}
00979 \textcolor{stringliteral}{    array([ 0.03125   , -0.1875    ,  0.04166667, -0.05208333,  0.01041667,}
00980 \textcolor{stringliteral}{            0.00625   ])}
00981 \textcolor{stringliteral}{    >>> C.chebint(cs, k=3)}
00982 \textcolor{stringliteral}{    array([ 3.5, -0.5,  0.5,  0.5])}
00983 \textcolor{stringliteral}{    >>> C.chebint(cs,lbnd=-2)}
00984 \textcolor{stringliteral}{    array([ 8.5, -0.5,  0.5,  0.5])}
00985 \textcolor{stringliteral}{    >>> C.chebint(cs,scl=-2)}
00986 \textcolor{stringliteral}{    array([-1.,  1., -1., -1.])}
00987 \textcolor{stringliteral}{}
00988 \textcolor{stringliteral}{    """}
00989     cnt = int(m)
00990     \textcolor{keywordflow}{if} \textcolor{keywordflow}{not} np.iterable(k):
00991         k = [k]
00992 
00993     \textcolor{keywordflow}{if} cnt != m:
00994         \textcolor{keywordflow}{raise} ValueError, \textcolor{stringliteral}{"The order of integration must be integer"}
00995     \textcolor{keywordflow}{if} cnt < 0 :
00996         \textcolor{keywordflow}{raise} ValueError, \textcolor{stringliteral}{"The order of integration must be non-negative"}
00997     \textcolor{keywordflow}{if} len(k) > cnt :
00998         \textcolor{keywordflow}{raise} ValueError, \textcolor{stringliteral}{"Too many integration constants"}
00999 
01000     \textcolor{comment}{# cs is a trimmed copy}
01001     [cs] = pu.as\_series([cs])
01002     \textcolor{keywordflow}{if} cnt == 0:
01003         \textcolor{keywordflow}{return} cs
01004 
01005     k = list(k) + [0]*(cnt - len(k))
01006     \textcolor{keywordflow}{for} i \textcolor{keywordflow}{in} range(cnt) :
01007         n = len(cs)
01008         cs *= scl
01009         \textcolor{keywordflow}{if} n == 1 \textcolor{keywordflow}{and} cs[0] == 0:
01010             cs[0] += k[i]
01011         \textcolor{keywordflow}{else}:
01012             zs = \hyperlink{namespacepyneb_1_1utils_1_1chebyshev_a659e346f7cdd9fd058850b26f7e95b17}{\_cseries\_to\_zseries}(cs)
01013             zs = \hyperlink{namespacepyneb_1_1utils_1_1chebyshev_a39d3ae73408fcde399d87f5cfd5e7c98}{\_zseries\_int}(zs)
01014             cs = \hyperlink{namespacepyneb_1_1utils_1_1chebyshev_a2a88474ce0ea12fb26f82b7116752dc1}{\_zseries\_to\_cseries}(zs)
01015             cs[0] += k[i] - \hyperlink{namespacepyneb_1_1utils_1_1chebyshev_a7f7442e9a3a00e1ca04a2283877b15a8}{chebval}(lbnd, cs)
01016     \textcolor{keywordflow}{return} cs
01017 
\hypertarget{chebyshev_8py_source_l01018}{}\hyperlink{namespacepyneb_1_1utils_1_1chebyshev_a7f7442e9a3a00e1ca04a2283877b15a8}{01018} \textcolor{keyword}{def }\hyperlink{namespacepyneb_1_1utils_1_1chebyshev_a7f7442e9a3a00e1ca04a2283877b15a8}{chebval}(x, cs):
01019     \textcolor{stringliteral}{"""Evaluate a Chebyshev series.}
01020 \textcolor{stringliteral}{}
01021 \textcolor{stringliteral}{    If `cs` is of length `n`, this function returns :}
01022 \textcolor{stringliteral}{}
01023 \textcolor{stringliteral}{    ``p(x) = cs[0]*T\_0(x) + cs[1]*T\_1(x) + ... + cs[n-1]*T\_\{n-1\}(x)``}
01024 \textcolor{stringliteral}{}
01025 \textcolor{stringliteral}{    If x is a sequence or array then p(x) will have the same shape as x.}
01026 \textcolor{stringliteral}{    If r is a ring\_like object that supports multiplication and addition}
01027 \textcolor{stringliteral}{    by the values in `cs`, then an object of the same type is returned.}
01028 \textcolor{stringliteral}{}
01029 \textcolor{stringliteral}{    Parameters}
01030 \textcolor{stringliteral}{    ----------}
01031 \textcolor{stringliteral}{    x : array\_like, ring\_like}
01032 \textcolor{stringliteral}{        Array of numbers or objects that support multiplication and}
01033 \textcolor{stringliteral}{        addition with themselves and with the elements of `cs`.}
01034 \textcolor{stringliteral}{    cs : array\_like}
01035 \textcolor{stringliteral}{        1-d array of Chebyshev coefficients ordered from low to high.}
01036 \textcolor{stringliteral}{}
01037 \textcolor{stringliteral}{    Returns}
01038 \textcolor{stringliteral}{    -------}
01039 \textcolor{stringliteral}{    values : ndarray, ring\_like}
01040 \textcolor{stringliteral}{        If the return is an ndarray then it has the same shape as `x`.}
01041 \textcolor{stringliteral}{}
01042 \textcolor{stringliteral}{    See Also}
01043 \textcolor{stringliteral}{    --------}
01044 \textcolor{stringliteral}{    chebfit}
01045 \textcolor{stringliteral}{}
01046 \textcolor{stringliteral}{    Examples}
01047 \textcolor{stringliteral}{    --------}
01048 \textcolor{stringliteral}{}
01049 \textcolor{stringliteral}{    Notes}
01050 \textcolor{stringliteral}{    -----}
01051 \textcolor{stringliteral}{    The evaluation uses Clenshaw recursion, aka synthetic division.}
01052 \textcolor{stringliteral}{}
01053 \textcolor{stringliteral}{    Examples}
01054 \textcolor{stringliteral}{    --------}
01055 \textcolor{stringliteral}{}
01056 \textcolor{stringliteral}{    """}
01057     \textcolor{comment}{# cs is a trimmed copy}
01058     [cs] = pu.as\_series([cs])
01059     \textcolor{keywordflow}{if} isinstance(x, tuple) \textcolor{keywordflow}{or} isinstance(x, list) :
01060         x = np.asarray(x)
01061 
01062     \textcolor{keywordflow}{if} len(cs) == 1 :
01063         c0 = cs[0]
01064         c1 = 0
01065     \textcolor{keywordflow}{elif} len(cs) == 2 :
01066         c0 = cs[0]
01067         c1 = cs[1]
01068     \textcolor{keywordflow}{else} :
01069         x2 = 2*x
01070         c0 = cs[-2]
01071         c1 = cs[-1]
01072         \textcolor{keywordflow}{for} i \textcolor{keywordflow}{in} range(3, len(cs) + 1) :
01073             tmp = c0
01074             c0 = cs[-i] - c1
01075             c1 = tmp + c1*x2
01076     \textcolor{keywordflow}{return} c0 + c1*x
01077 
\hypertarget{chebyshev_8py_source_l01078}{}\hyperlink{namespacepyneb_1_1utils_1_1chebyshev_aa9deae857486c8ee69362ded985b09d8}{01078} \textcolor{keyword}{def }\hyperlink{namespacepyneb_1_1utils_1_1chebyshev_aa9deae857486c8ee69362ded985b09d8}{chebvander}(x, deg) :
01079     \textcolor{stringliteral}{"""Vandermonde matrix of given degree.}
01080 \textcolor{stringliteral}{}
01081 \textcolor{stringliteral}{    Returns the Vandermonde matrix of degree `deg` and sample points `x`.}
01082 \textcolor{stringliteral}{    This isn't a true Vandermonde matrix because `x` can be an arbitrary}
01083 \textcolor{stringliteral}{    ndarray and the Chebyshev polynomials aren't powers. If ``V`` is the}
01084 \textcolor{stringliteral}{    returned matrix and `x` is a 2d array, then the elements of ``V`` are}
01085 \textcolor{stringliteral}{    ``V[i,j,k] = T\_k(x[i,j])``, where ``T\_k`` is the Chebyshev polynomial}
01086 \textcolor{stringliteral}{    of degree ``k``.}
01087 \textcolor{stringliteral}{}
01088 \textcolor{stringliteral}{    Parameters}
01089 \textcolor{stringliteral}{    ----------}
01090 \textcolor{stringliteral}{    x : array\_like}
01091 \textcolor{stringliteral}{        Array of points. The values are converted to double or complex}
01092 \textcolor{stringliteral}{        doubles. If x is scalar it is converted to a 1D array.}
01093 \textcolor{stringliteral}{    deg : integer}
01094 \textcolor{stringliteral}{        Degree of the resulting matrix.}
01095 \textcolor{stringliteral}{}
01096 \textcolor{stringliteral}{    Returns}
01097 \textcolor{stringliteral}{    -------}
01098 \textcolor{stringliteral}{    vander : Vandermonde matrix.}
01099 \textcolor{stringliteral}{        The shape of the returned matrix is ``x.shape + (deg+1,)``. The last}
01100 \textcolor{stringliteral}{        index is the degree.}
01101 \textcolor{stringliteral}{}
01102 \textcolor{stringliteral}{    """}
01103     ideg = int(deg)
01104     \textcolor{keywordflow}{if} ideg != deg:
01105         \textcolor{keywordflow}{raise} ValueError(\textcolor{stringliteral}{"deg must be integer"})
01106     \textcolor{keywordflow}{if} ideg < 0:
01107         \textcolor{keywordflow}{raise} ValueError(\textcolor{stringliteral}{"deg must be non-negative"})
01108 
01109     x = np.array(x, copy=0, ndmin=1) + 0.0
01110     v = np.empty((ideg + 1,) + x.shape, dtype=x.dtype)
01111     \textcolor{comment}{# Use forward recursion to generate the entries.}
01112     v[0] = x*0 + 1
01113     \textcolor{keywordflow}{if} ideg > 0 :
01114         x2 = 2*x
01115         v[1] = x
01116         \textcolor{keywordflow}{for} i \textcolor{keywordflow}{in} range(2, ideg + 1) :
01117             v[i] = v[i-1]*x2 - v[i-2]
01118     \textcolor{keywordflow}{return} np.rollaxis(v, 0, v.ndim)
01119 
01120 
\hypertarget{chebyshev_8py_source_l01121}{}\hyperlink{namespacepyneb_1_1utils_1_1chebyshev_a6bb5eb95525fd0572aea52fcd2a12c6c}{01121} \textcolor{keyword}{def }\hyperlink{namespacepyneb_1_1utils_1_1chebyshev_a6bb5eb95525fd0572aea52fcd2a12c6c}{chebfit}(x, y, deg, rcond=None, full=False, w=None):
01122     \textcolor{stringliteral}{"""}
01123 \textcolor{stringliteral}{    Least squares fit of Chebyshev series to data.}
01124 \textcolor{stringliteral}{}
01125 \textcolor{stringliteral}{    Fit a Chebyshev series ``p(x) = p[0] * T\_\{0\}(x) + ... + p[deg] *}
01126 \textcolor{stringliteral}{    T\_\{deg\}(x)`` of degree `deg` to points `(x, y)`. Returns a vector of}
01127 \textcolor{stringliteral}{    coefficients `p` that minimises the squared error.}
01128 \textcolor{stringliteral}{}
01129 \textcolor{stringliteral}{    Parameters}
01130 \textcolor{stringliteral}{    ----------}
01131 \textcolor{stringliteral}{    x : array\_like, shape (M,)}
01132 \textcolor{stringliteral}{        x-coordinates of the M sample points ``(x[i], y[i])``.}
01133 \textcolor{stringliteral}{    y : array\_like, shape (M,) or (M, K)}
01134 \textcolor{stringliteral}{        y-coordinates of the sample points. Several data sets of sample}
01135 \textcolor{stringliteral}{        points sharing the same x-coordinates can be fitted at once by}
01136 \textcolor{stringliteral}{        passing in a 2D-array that contains one dataset per column.}
01137 \textcolor{stringliteral}{    deg : int}
01138 \textcolor{stringliteral}{        Degree of the fitting polynomial}
01139 \textcolor{stringliteral}{    rcond : float, optional}
01140 \textcolor{stringliteral}{        Relative condition number of the fit. Singular values smaller than}
01141 \textcolor{stringliteral}{        this relative to the largest singular value will be ignored. The}
01142 \textcolor{stringliteral}{        default value is len(x)*eps, where eps is the relative precision of}
01143 \textcolor{stringliteral}{        the float type, about 2e-16 in most cases.}
01144 \textcolor{stringliteral}{    full : bool, optional}
01145 \textcolor{stringliteral}{        Switch determining nature of return value. When it is False (the}
01146 \textcolor{stringliteral}{        default) just the coefficients are returned, when True diagnostic}
01147 \textcolor{stringliteral}{        information from the singular value decomposition is also returned.}
01148 \textcolor{stringliteral}{    w : array\_like, shape (`M`,), optional}
01149 \textcolor{stringliteral}{        Weights. If not None, the contribution of each point}
01150 \textcolor{stringliteral}{        ``(x[i],y[i])`` to the fit is weighted by `w[i]`. Ideally the}
01151 \textcolor{stringliteral}{        weights are chosen so that the errors of the products ``w[i]*y[i]``}
01152 \textcolor{stringliteral}{        all have the same variance.  The default value is None.}
01153 \textcolor{stringliteral}{        .. versionadded:: 1.5.0}
01154 \textcolor{stringliteral}{}
01155 \textcolor{stringliteral}{    Returns}
01156 \textcolor{stringliteral}{    -------}
01157 \textcolor{stringliteral}{    coef : ndarray, shape (M,) or (M, K)}
01158 \textcolor{stringliteral}{        Chebyshev coefficients ordered from low to high. If `y` was 2-D,}
01159 \textcolor{stringliteral}{        the coefficients for the data in column k  of `y` are in column}
01160 \textcolor{stringliteral}{        `k`.}
01161 \textcolor{stringliteral}{}
01162 \textcolor{stringliteral}{    [residuals, rank, singular\_values, rcond] : present when `full` = True}
01163 \textcolor{stringliteral}{        Residuals of the least-squares fit, the effective rank of the}
01164 \textcolor{stringliteral}{        scaled Vandermonde matrix and its singular values, and the}
01165 \textcolor{stringliteral}{        specified value of `rcond`. For more details, see `linalg.lstsq`.}
01166 \textcolor{stringliteral}{}
01167 \textcolor{stringliteral}{    Warns}
01168 \textcolor{stringliteral}{    -----}
01169 \textcolor{stringliteral}{    RankWarning}
01170 \textcolor{stringliteral}{        The rank of the coefficient matrix in the least-squares fit is}
01171 \textcolor{stringliteral}{        deficient. The warning is only raised if `full` = False.  The}
01172 \textcolor{stringliteral}{        warnings can be turned off by}
01173 \textcolor{stringliteral}{}
01174 \textcolor{stringliteral}{        >>> import warnings}
01175 \textcolor{stringliteral}{        >>> warnings.simplefilter('ignore', RankWarning)}
01176 \textcolor{stringliteral}{}
01177 \textcolor{stringliteral}{    See Also}
01178 \textcolor{stringliteral}{    --------}
01179 \textcolor{stringliteral}{    chebval : Evaluates a Chebyshev series.}
01180 \textcolor{stringliteral}{    chebvander : Vandermonde matrix of Chebyshev series.}
01181 \textcolor{stringliteral}{    polyfit : least squares fit using polynomials.}
01182 \textcolor{stringliteral}{    linalg.lstsq : Computes a least-squares fit from the matrix.}
01183 \textcolor{stringliteral}{    scipy.interpolate.UnivariateSpline : Computes spline fits.}
01184 \textcolor{stringliteral}{}
01185 \textcolor{stringliteral}{    Notes}
01186 \textcolor{stringliteral}{    -----}
01187 \textcolor{stringliteral}{    The solution are the coefficients ``c[i]`` of the Chebyshev series}
01188 \textcolor{stringliteral}{    ``T(x)`` that minimizes the squared error}
01189 \textcolor{stringliteral}{}
01190 \textcolor{stringliteral}{    ``E = \(\backslash\)\(\backslash\)sum\_j |y\_j - T(x\_j)|^2``.}
01191 \textcolor{stringliteral}{}
01192 \textcolor{stringliteral}{    This problem is solved by setting up as the overdetermined matrix}
01193 \textcolor{stringliteral}{    equation}
01194 \textcolor{stringliteral}{}
01195 \textcolor{stringliteral}{    ``V(x)*c = y``,}
01196 \textcolor{stringliteral}{}
01197 \textcolor{stringliteral}{    where ``V`` is the Vandermonde matrix of `x`, the elements of ``c`` are}
01198 \textcolor{stringliteral}{    the coefficients to be solved for, and the elements of `y` are the}
01199 \textcolor{stringliteral}{    observed values.  This equation is then solved using the singular value}
01200 \textcolor{stringliteral}{    decomposition of ``V``.}
01201 \textcolor{stringliteral}{}
01202 \textcolor{stringliteral}{    If some of the singular values of ``V`` are so small that they are}
01203 \textcolor{stringliteral}{    neglected, then a `RankWarning` will be issued. This means that the}
01204 \textcolor{stringliteral}{    coeficient values may be poorly determined. Using a lower order fit}
01205 \textcolor{stringliteral}{    will usually get rid of the warning.  The `rcond` parameter can also be}
01206 \textcolor{stringliteral}{    set to a value smaller than its default, but the resulting fit may be}
01207 \textcolor{stringliteral}{    spurious and have large contributions from roundoff error.}
01208 \textcolor{stringliteral}{}
01209 \textcolor{stringliteral}{    Fits using Chebyshev series are usually better conditioned than fits}
01210 \textcolor{stringliteral}{    using power series, but much can depend on the distribution of the}
01211 \textcolor{stringliteral}{    sample points and the smoothness of the data. If the quality of the fit}
01212 \textcolor{stringliteral}{    is inadequate splines may be a good alternative.}
01213 \textcolor{stringliteral}{}
01214 \textcolor{stringliteral}{    References}
01215 \textcolor{stringliteral}{    ----------}
01216 \textcolor{stringliteral}{    .. [1] Wikipedia, "Curve fitting",}
01217 \textcolor{stringliteral}{           http://en.wikipedia.org/wiki/Curve\_fitting}
01218 \textcolor{stringliteral}{}
01219 \textcolor{stringliteral}{    Examples}
01220 \textcolor{stringliteral}{    --------}
01221 \textcolor{stringliteral}{}
01222 \textcolor{stringliteral}{    """}
01223     order = int(deg) + 1
01224     x = np.asarray(x) + 0.0
01225     y = np.asarray(y) + 0.0
01226 
01227     \textcolor{comment}{# check arguments.}
01228     \textcolor{keywordflow}{if} deg < 0 :
01229         \textcolor{keywordflow}{raise} ValueError, \textcolor{stringliteral}{"expected deg >= 0"}
01230     \textcolor{keywordflow}{if} x.ndim != 1:
01231         \textcolor{keywordflow}{raise} TypeError, \textcolor{stringliteral}{"expected 1D vector for x"}
01232     \textcolor{keywordflow}{if} x.size == 0:
01233         \textcolor{keywordflow}{raise} TypeError, \textcolor{stringliteral}{"expected non-empty vector for x"}
01234     \textcolor{keywordflow}{if} y.ndim < 1 \textcolor{keywordflow}{or} y.ndim > 2 :
01235         \textcolor{keywordflow}{raise} TypeError, \textcolor{stringliteral}{"expected 1D or 2D array for y"}
01236     \textcolor{keywordflow}{if} len(x) != len(y):
01237         \textcolor{keywordflow}{raise} TypeError, \textcolor{stringliteral}{"expected x and y to have same length"}
01238 
01239     \textcolor{comment}{# set up the least squares matrices}
01240     lhs = \hyperlink{namespacepyneb_1_1utils_1_1chebyshev_aa9deae857486c8ee69362ded985b09d8}{chebvander}(x, deg)
01241     rhs = y
01242     \textcolor{keywordflow}{if} w \textcolor{keywordflow}{is} \textcolor{keywordflow}{not} \textcolor{keywordtype}{None}:
01243         w = np.asarray(w) + 0.0
01244         \textcolor{keywordflow}{if} w.ndim != 1:
01245             \textcolor{keywordflow}{raise} TypeError, \textcolor{stringliteral}{"expected 1D vector for w"}
01246         \textcolor{keywordflow}{if} len(x) != len(w):
01247             \textcolor{keywordflow}{raise} TypeError, \textcolor{stringliteral}{"expected x and w to have same length"}
01248         \textcolor{comment}{# apply weights}
01249         \textcolor{keywordflow}{if} rhs.ndim == 2:
01250             lhs *= w[:, np.newaxis]
01251             rhs *= w[:, np.newaxis]
01252         \textcolor{keywordflow}{else}:
01253             lhs *= w[:, np.newaxis]
01254             rhs *= w
01255 
01256     \textcolor{comment}{# set rcond}
01257     \textcolor{keywordflow}{if} rcond \textcolor{keywordflow}{is} \textcolor{keywordtype}{None} :
01258         rcond = len(x)*np.finfo(x.dtype).eps
01259 
01260     \textcolor{comment}{# scale the design matrix and solve the least squares equation}
01261     scl = np.sqrt((lhs*lhs).sum(0))
01262     c, resids, rank, s = la.lstsq(lhs/scl, rhs, rcond)
01263     c = (c.T/scl).T
01264 
01265     \textcolor{comment}{# warn on rank reduction}
01266     \textcolor{keywordflow}{if} rank != order \textcolor{keywordflow}{and} \textcolor{keywordflow}{not} full:
01267         msg = \textcolor{stringliteral}{"The fit may be poorly conditioned"}
01268         warnings.warn(msg, pu.RankWarning)
01269 
01270     \textcolor{keywordflow}{if} full :
01271         \textcolor{keywordflow}{return} c, [resids, rank, s, rcond]
01272     \textcolor{keywordflow}{else} :
01273         \textcolor{keywordflow}{return} c
01274 
01275 
\hypertarget{chebyshev_8py_source_l01276}{}\hyperlink{namespacepyneb_1_1utils_1_1chebyshev_a2f01a14e36b415a2f3ebc67c95cceb26}{01276} \textcolor{keyword}{def }\hyperlink{namespacepyneb_1_1utils_1_1chebyshev_a2f01a14e36b415a2f3ebc67c95cceb26}{chebroots}(cs):
01277     \textcolor{stringliteral}{"""}
01278 \textcolor{stringliteral}{    Compute the roots of a Chebyshev series.}
01279 \textcolor{stringliteral}{}
01280 \textcolor{stringliteral}{    Return the roots (a.k.a "zeros") of the C-series represented by `cs`,}
01281 \textcolor{stringliteral}{    which is the sequence of the C-series' coefficients from lowest order}
01282 \textcolor{stringliteral}{    "term" to highest, e.g., [1,2,3] represents the C-series}
01283 \textcolor{stringliteral}{    ``T\_0 + 2*T\_1 + 3*T\_2``.}
01284 \textcolor{stringliteral}{}
01285 \textcolor{stringliteral}{    Parameters}
01286 \textcolor{stringliteral}{    ----------}
01287 \textcolor{stringliteral}{    cs : array\_like}
01288 \textcolor{stringliteral}{        1-d array of C-series coefficients ordered from low to high.}
01289 \textcolor{stringliteral}{}
01290 \textcolor{stringliteral}{    Returns}
01291 \textcolor{stringliteral}{    -------}
01292 \textcolor{stringliteral}{    out : ndarray}
01293 \textcolor{stringliteral}{        Array of the roots.  If all the roots are real, then so is the}
01294 \textcolor{stringliteral}{        dtype of ``out``; otherwise, ``out``'s dtype is complex.}
01295 \textcolor{stringliteral}{}
01296 \textcolor{stringliteral}{    See Also}
01297 \textcolor{stringliteral}{    --------}
01298 \textcolor{stringliteral}{    polyroots}
01299 \textcolor{stringliteral}{}
01300 \textcolor{stringliteral}{    Notes}
01301 \textcolor{stringliteral}{    -----}
01302 \textcolor{stringliteral}{    Algorithm(s) used:}
01303 \textcolor{stringliteral}{}
01304 \textcolor{stringliteral}{    Remember: because the C-series basis set is different from the}
01305 \textcolor{stringliteral}{    "standard" basis set, the results of this function *may* not be what}
01306 \textcolor{stringliteral}{    one is expecting.}
01307 \textcolor{stringliteral}{}
01308 \textcolor{stringliteral}{    Examples}
01309 \textcolor{stringliteral}{    --------}
01310 \textcolor{stringliteral}{    >>> import numpy.polynomial as P}
01311 \textcolor{stringliteral}{    >>> import numpy.polynomial.chebyshev as C}
01312 \textcolor{stringliteral}{    >>> P.polyroots((-1,1,-1,1)) # x^3 - x^2 + x - 1 has two complex roots}
01313 \textcolor{stringliteral}{    array([ -4.99600361e-16-1.j,  -4.99600361e-16+1.j,   1.00000e+00+0.j])}
01314 \textcolor{stringliteral}{    >>> C.chebroots((-1,1,-1,1)) # T3 - T2 + T1 - T0 has only real roots}
01315 \textcolor{stringliteral}{    array([ -5.00000000e-01,   2.60860684e-17,   1.00000000e+00])}
01316 \textcolor{stringliteral}{}
01317 \textcolor{stringliteral}{    """}
01318     \textcolor{comment}{# cs is a trimmed copy}
01319     [cs] = pu.as\_series([cs])
01320     \textcolor{keywordflow}{if} len(cs) <= 1 :
01321         \textcolor{keywordflow}{return} np.array([], dtype=cs.dtype)
01322     \textcolor{keywordflow}{if} len(cs) == 2 :
01323         \textcolor{keywordflow}{return} np.array([-cs[0]/cs[1]])
01324 
01325     n = len(cs) - 1
01326     cs /= cs[-1]
01327     cmat = np.zeros((n,n), dtype=cs.dtype)
01328     cmat[1, 0] = 1
01329     \textcolor{keywordflow}{for} i \textcolor{keywordflow}{in} range(1, n):
01330         cmat[i - 1, i] = .5
01331         \textcolor{keywordflow}{if} i != n - 1:
01332             cmat[i + 1, i] = .5
01333         \textcolor{keywordflow}{else}:
01334             cmat[:, i] -= cs[:-1]*.5
01335     roots = la.eigvals(cmat)
01336     roots.sort()
01337     \textcolor{keywordflow}{return} roots
01338 
01339 
\hypertarget{chebyshev_8py_source_l01340}{}\hyperlink{namespacepyneb_1_1utils_1_1chebyshev_a5ac57e95809a1f627400a7c3a34606f7}{01340} \textcolor{keyword}{def }\hyperlink{namespacepyneb_1_1utils_1_1chebyshev_a5ac57e95809a1f627400a7c3a34606f7}{chebpts1}(npts):
01341     \textcolor{stringliteral}{"""Chebyshev points of the first kind.}
01342 \textcolor{stringliteral}{}
01343 \textcolor{stringliteral}{    Chebyshev points of the first kind are the set ``\{cos(x\_k)\}``,}
01344 \textcolor{stringliteral}{    where ``x\_k = pi*(k + .5)/npts`` for k in ``range(npts\}``.}
01345 \textcolor{stringliteral}{}
01346 \textcolor{stringliteral}{    Parameters}
01347 \textcolor{stringliteral}{    ----------}
01348 \textcolor{stringliteral}{    npts : int}
01349 \textcolor{stringliteral}{        Number of sample points desired.}
01350 \textcolor{stringliteral}{}
01351 \textcolor{stringliteral}{    Returns}
01352 \textcolor{stringliteral}{    -------}
01353 \textcolor{stringliteral}{    pts : ndarray}
01354 \textcolor{stringliteral}{        The Chebyshev points of the second kind.}
01355 \textcolor{stringliteral}{}
01356 \textcolor{stringliteral}{    Notes}
01357 \textcolor{stringliteral}{    -----}
01358 \textcolor{stringliteral}{    .. versionadded:: 1.5.0}
01359 \textcolor{stringliteral}{}
01360 \textcolor{stringliteral}{    """}
01361     \_npts = int(npts)
01362     \textcolor{keywordflow}{if} \_npts != npts:
01363         \textcolor{keywordflow}{raise} ValueError(\textcolor{stringliteral}{"npts must be integer"})
01364     \textcolor{keywordflow}{if} \_npts < 1:
01365         \textcolor{keywordflow}{raise} ValueError(\textcolor{stringliteral}{"npts must be >= 1"})
01366 
01367     x = np.linspace(-np.pi, 0, \_npts, endpoint=\textcolor{keyword}{False}) + np.pi/(2*\_npts)
01368     \textcolor{keywordflow}{return} np.cos(x)
01369 
01370 
\hypertarget{chebyshev_8py_source_l01371}{}\hyperlink{namespacepyneb_1_1utils_1_1chebyshev_a224bc370de0cfa204c3ac15d626d13e3}{01371} \textcolor{keyword}{def }\hyperlink{namespacepyneb_1_1utils_1_1chebyshev_a224bc370de0cfa204c3ac15d626d13e3}{chebpts2}(npts):
01372     \textcolor{stringliteral}{"""Chebyshev points of the second kind.}
01373 \textcolor{stringliteral}{}
01374 \textcolor{stringliteral}{    Chebyshev points of the second kind are the set ``\{cos(x\_k)\}``,}
01375 \textcolor{stringliteral}{    where ``x\_k = pi*/(npts - 1)`` for k in ``range(npts\}``.}
01376 \textcolor{stringliteral}{}
01377 \textcolor{stringliteral}{    Parameters}
01378 \textcolor{stringliteral}{    ----------}
01379 \textcolor{stringliteral}{    npts : int}
01380 \textcolor{stringliteral}{        Number of sample points desired.}
01381 \textcolor{stringliteral}{}
01382 \textcolor{stringliteral}{    Returns}
01383 \textcolor{stringliteral}{    -------}
01384 \textcolor{stringliteral}{    pts : ndarray}
01385 \textcolor{stringliteral}{        The Chebyshev points of the second kind.}
01386 \textcolor{stringliteral}{}
01387 \textcolor{stringliteral}{    Notes}
01388 \textcolor{stringliteral}{    -----}
01389 \textcolor{stringliteral}{    .. versionadded:: 1.5.0}
01390 \textcolor{stringliteral}{}
01391 \textcolor{stringliteral}{    """}
01392     \_npts = int(npts)
01393     \textcolor{keywordflow}{if} \_npts != npts:
01394         \textcolor{keywordflow}{raise} ValueError(\textcolor{stringliteral}{"npts must be integer"})
01395     \textcolor{keywordflow}{if} \_npts < 2:
01396         \textcolor{keywordflow}{raise} ValueError(\textcolor{stringliteral}{"npts must be >= 2"})
01397 
01398     x = np.linspace(-np.pi, 0, \_npts)
01399     \textcolor{keywordflow}{return} np.cos(x)
01400 
01401 
01402 \textcolor{comment}{#}
01403 \textcolor{comment}{# Chebyshev series class}
01404 \textcolor{comment}{#}
01405 
01406 exec polytemplate.substitute(name=\textcolor{stringliteral}{'Chebyshev'}, nick=\textcolor{stringliteral}{'cheb'}, domain=\textcolor{stringliteral}{'[-1,1]'})
\end{DoxyCode}
