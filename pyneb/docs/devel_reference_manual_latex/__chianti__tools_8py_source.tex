\hypertarget{__chianti__tools_8py_source}{}\section{\+\_\+chianti\+\_\+tools.\+py}
\label{__chianti__tools_8py_source}\index{pyneb/utils/\+\_\+chianti\+\_\+tools.\+py@{pyneb/utils/\+\_\+chianti\+\_\+tools.\+py}}

\begin{DoxyCode}
\hypertarget{__chianti__tools_8py_source_l00001}{}\hyperlink{namespacepyneb_1_1utils_1_1__chianti__tools}{00001} \textcolor{stringliteral}{'''Utility functions, many for reading the CHIANTI database files.}
00002 \textcolor{stringliteral}{}
00003 \textcolor{stringliteral}{Copyright 2009, 2010 Kenneth P. Dere}
00004 \textcolor{stringliteral}{}
00005 \textcolor{stringliteral}{This software is distributed under the terms of the GNU General Public License}
00006 \textcolor{stringliteral}{that is found in the LICENSE file}
00007 \textcolor{stringliteral}{}
00008 \textcolor{stringliteral}{}
00009 \textcolor{stringliteral}{'''}
00010 \textcolor{keyword}{import} os, fnmatch
00011 \textcolor{keyword}{from} types \textcolor{keyword}{import} *
00012 \textcolor{keyword}{from} ConfigParser \textcolor{keyword}{import} *
00013 \textcolor{keyword}{import} pickle
00014 \textcolor{keyword}{from} datetime \textcolor{keyword}{import} date
00015 \textcolor{keyword}{import} numpy \textcolor{keyword}{as} np
00016 \textcolor{keyword}{from} scipy \textcolor{keyword}{import} interpolate
00017 \textcolor{keyword}{from} FortranFormat \textcolor{keyword}{import} *
00018 \textcolor{keyword}{import} \_chianti\_constants \textcolor{keyword}{as} const
00019 \textcolor{comment}{#}
00020 \textcolor{comment}{#}
\hypertarget{__chianti__tools_8py_source_l00021}{}\hyperlink{namespacepyneb_1_1utils_1_1__chianti__tools_a46b8e5a174144ca33b476e2ea883403a}{00021} \textcolor{keyword}{def }\hyperlink{namespacepyneb_1_1utils_1_1__chianti__tools_a46b8e5a174144ca33b476e2ea883403a}{between}(array,limits):
00022     \textcolor{stringliteral}{'''returns an index array of elements of array where the values lie}
00023 \textcolor{stringliteral}{    between the limits given as a 2 element list or tuple'''}
00024     array=np.asarray(array)
00025     nlines=len(array)
00026     hi=np.where(array >= limits[0],range(1,nlines+1),0)
00027     lo=np.where(array <= limits[1],range(1,nlines+1),0)
00028     hilo=hi&lo
00029     out=[a -1  \textcolor{keywordflow}{for} a \textcolor{keywordflow}{in} hilo \textcolor{keywordflow}{if} a > 0]
00030     \textcolor{keywordflow}{return} out
00031     \textcolor{comment}{# -------------------------------------------------------------------------------------}
00032     \textcolor{comment}{#}
\hypertarget{__chianti__tools_8py_source_l00033}{}\hyperlink{namespacepyneb_1_1utils_1_1__chianti__tools_addc139edf6069387aed0049344002b13}{00033} \textcolor{keyword}{def }\hyperlink{namespacepyneb_1_1utils_1_1__chianti__tools_addc139edf6069387aed0049344002b13}{ipRead}(verbose=False):
00034     \textcolor{stringliteral}{"""}
00035 \textcolor{stringliteral}{    reads the ionization potential file, returns ip array in eV}
00036 \textcolor{stringliteral}{    """}
00037     topdir=os.environ[\textcolor{stringliteral}{"XUVTOP"}]
00038     ipname=os.path.join(topdir, \textcolor{stringliteral}{'ip'},\textcolor{stringliteral}{'chianti.ip'})
00039     ipfile=open(ipname)
00040     data=ipfile.readlines()
00041     ipfile.close()
00042     nip=0
00043     ndata=2
00044     maxz=0
00045     \textcolor{keywordflow}{while} ndata > 1:
00046         s1=data[nip]
00047         s2=s1.split()
00048         ndata=len(s2)
00049         nip=nip+1
00050         \textcolor{keywordflow}{if} int(s2[0]) > maxz:
00051             maxz=int(s2[0])
00052     \textcolor{keywordflow}{if} verbose:
00053         print((\textcolor{stringliteral}{' maxz = %5i'}%(maxz)))
00054     nip=nip-1
00055     ip=np.zeros((maxz, maxz), \textcolor{stringliteral}{'Float64'})
00056     \textcolor{keywordflow}{for} aline \textcolor{keywordflow}{in} data[0:nip]:
00057         s2=aline.split()
00058         iz=int(s2[0])
00059         ion=int(s2[1])
00060         ip[iz-1, ion-1]=float(s2[2])
00061     \textcolor{keywordflow}{return} ip*const.invCm2Ev
00062     \textcolor{comment}{#}
00063     \textcolor{comment}{# -------------------------------------------------------------------------------------}
00064     \textcolor{comment}{#}
\hypertarget{__chianti__tools_8py_source_l00065}{}\hyperlink{namespacepyneb_1_1utils_1_1__chianti__tools_a5677fc0d50bd22a853c14b4b2606b89e}{00065} \textcolor{keyword}{def }\hyperlink{namespacepyneb_1_1utils_1_1__chianti__tools_a5677fc0d50bd22a853c14b4b2606b89e}{masterListInfo}(force=0):
00066     \textcolor{stringliteral}{""" returns information about ions in masterlist}
00067 \textcolor{stringliteral}{    the reason for this file is to speed up multi-ion spectral calculations}
00068 \textcolor{stringliteral}{    the information is stored in a pickled file 'masterlist\_ions.pkl'}
00069 \textcolor{stringliteral}{    if the file is not found, one will be created and the following information}
00070 \textcolor{stringliteral}{    returned for each ion}
00071 \textcolor{stringliteral}{    wmin, wmax :  the minimum and maximum wavelengths in the wgfa file}
00072 \textcolor{stringliteral}{    tmin, tmax :  the minimum and maximum temperatures for which the ionization balance is nonzero"""}
00073     dir=os.environ[\textcolor{stringliteral}{"XUVTOP"}]
00074     infoPath = os.path.join(dir, \textcolor{stringliteral}{'masterlist'})
00075     infoName=os.path.join(dir,\textcolor{stringliteral}{'masterlist'},\textcolor{stringliteral}{'masterlist\_ions.pkl'})
00076     masterName=os.path.join(dir,\textcolor{stringliteral}{'masterlist'},\textcolor{stringliteral}{'masterlist.ions'})
00077     \textcolor{comment}{#}
00078     makeNew = force == 1 \textcolor{keywordflow}{or} \textcolor{keywordflow}{not} os.path.isfile(infoName)
00079 \textcolor{comment}{#    if os.path.isfile(infoName):}
00080     \textcolor{keywordflow}{if} \textcolor{keywordflow}{not} makeNew:
00081 \textcolor{comment}{#       print ' file exists - ',  infoName}
00082         pfile = open(infoName, \textcolor{stringliteral}{'}\textcolor{stringliteral}{r')}
00083 \textcolor{stringliteral}{        masterListInfo = pickle.load(pfile)}
00084 \textcolor{stringliteral}{        pfile.close}
00085 \textcolor{stringliteral}{    }\textcolor{keywordflow}{elif} os.access(infoPath, os.W\_OK):
00086         \textcolor{comment}{# the file does not exist but we have write access and will create it}
00087         defaults = \hyperlink{namespacepyneb_1_1utils_1_1__chianti__tools_a235ada2c4e384f436dbade0913107585}{defaultsRead}()
00088         print((\textcolor{stringliteral}{' defaults = %s'}%(str(defaults))))
00089         ioneqName = defaults[\textcolor{stringliteral}{'ioneqfile'}]
00090         ioneq = \hyperlink{namespacepyneb_1_1utils_1_1__chianti__tools_a8b6257cfe133ac906966b20c8721f82a}{ioneqRead}(ioneqname = ioneqName)
00091         masterList = \hyperlink{namespacepyneb_1_1utils_1_1__chianti__tools_a1a4447320a9d614f994bc538890c32cd}{masterListRead}()
00092         masterListInfo = \{\}
00093         haveZ = [0]*31
00094         haveStage = np.zeros((31, 31), \textcolor{stringliteral}{'Int32'})
00095         haveDielectronic = np.zeros((31, 31), \textcolor{stringliteral}{'Int32'})
00096         \textcolor{keywordflow}{for} one \textcolor{keywordflow}{in} masterList:
00097             ionInfo = \hyperlink{namespacepyneb_1_1utils_1_1__chianti__tools_a92cf299ad3407ee8923739e2761ab13f}{convertName}(one)
00098             z = ionInfo[\textcolor{stringliteral}{'Z'}]
00099             stage = ionInfo[\textcolor{stringliteral}{'Ion'}]
00100             haveZ[z] = 1
00101             dielectronic = ionInfo[\textcolor{stringliteral}{'Dielectronic'}]
00102             \textcolor{keywordflow}{if} dielectronic:
00103                 haveDielectronic[z, stage] = 1
00104             \textcolor{keywordflow}{else}:
00105                 haveStage[z, stage] = 1
00106             thisIoneq = ioneq[\textcolor{stringliteral}{'ioneqAll'}][z- 1, stage - 1 + dielectronic]
00107             good = thisIoneq > 0.
00108             goodTemp = ioneq[\textcolor{stringliteral}{'ioneqTemperature'}][good]
00109             tmin = goodTemp.min()
00110             tmax = goodTemp.max()
00111             vgood = thisIoneq == thisIoneq.max()
00112             vgoodTemp = ioneq[\textcolor{stringliteral}{'ioneqTemperature'}][vgood][0]
00113             wgfa = \hyperlink{namespacepyneb_1_1utils_1_1__chianti__tools_a3ef36a1d0a4df4cab94a392d2a3da980}{wgfaRead}(one)
00114             nZeros = wgfa[\textcolor{stringliteral}{'wvl'}].count(0.)
00115             \textcolor{comment}{# two-photon transitions are denoted by a wavelength of zero (0.)}
00116             \textcolor{keywordflow}{while} nZeros > 0:
00117                 wgfa[\textcolor{stringliteral}{'wvl'}].remove(0.)
00118                 nZeros = wgfa[\textcolor{stringliteral}{'wvl'}].count(0.)
00119             \textcolor{comment}{# unobserved lines are denoted with a negative wavelength}
00120             wvl = np.abs(np.asarray(wgfa[\textcolor{stringliteral}{'wvl'}], \textcolor{stringliteral}{'float64'}))
00121             wmin = wvl.min()
00122             wmax = wvl.max()
00123             masterListInfo[one] = \{\textcolor{stringliteral}{'wmin'}:wmin, \textcolor{stringliteral}{'wmax'}:wmax, \textcolor{stringliteral}{'tmin'}:tmin, \textcolor{stringliteral}{'tmax'}:tmax, \textcolor{stringliteral}{'tIoneqMax'}:
      vgoodTemp\}
00124         masterListInfo[\textcolor{stringliteral}{'haveZ'}] = haveZ
00125         masterListInfo[\textcolor{stringliteral}{'haveStage'}] = haveStage
00126         masterListInfo[\textcolor{stringliteral}{'haveDielectronic'}] = haveDielectronic
00127         \textcolor{comment}{#  now do the bare ions from H thru Zn}
00128         \textcolor{comment}{#  these are only involved in the continuum}
00129         \textcolor{keywordflow}{for} iz \textcolor{keywordflow}{in} range(1, 31):
00130             ions = \hyperlink{namespacepyneb_1_1utils_1_1__chianti__tools_a7d6debb5f68b52f64c311ca1fe99945c}{zion2name}(iz, iz+1)
00131             thisIoneq = ioneq[\textcolor{stringliteral}{'ioneqAll'}][iz-1, iz]
00132             good = thisIoneq > 0.
00133             goodTemp = ioneq[\textcolor{stringliteral}{'ioneqTemperature'}][good]
00134             tmin = goodTemp.min()
00135             tmax = goodTemp.max()
00136             wmin=0.
00137             wmax = 1.e+30
00138             masterListInfo[ions] = \{\textcolor{stringliteral}{'wmin'}:wmin, \textcolor{stringliteral}{'wmax'}:wmax, \textcolor{stringliteral}{'tmin'}:tmin, \textcolor{stringliteral}{'tmax'}:tmax\}
00139         pfile = open(infoName, \textcolor{stringliteral}{'w'})
00140         pickle.dump(masterListInfo, pfile)
00141         pfile.close
00142     \textcolor{keywordflow}{else}:
00143         \textcolor{comment}{# the file does not exist and we do NOT have write access to creat it}
00144         \textcolor{comment}{# will just make an inefficient, useless version}
00145         masterListInfo = \{\}
00146         \textcolor{keywordflow}{for} one \textcolor{keywordflow}{in} masterList:
00147             ionInfo = \hyperlink{namespacepyneb_1_1utils_1_1__chianti__tools_a92cf299ad3407ee8923739e2761ab13f}{convertName}(one)
00148             z = ionInfo[\textcolor{stringliteral}{'Z'}]
00149             stage = ionInfo[\textcolor{stringliteral}{'Ion'}]
00150             dielectronic = ionInfo[\textcolor{stringliteral}{'Dielectronic'}]
00151             wmin=0.
00152             wmax = 1.e+30
00153             masterListInfo[one] = \{\textcolor{stringliteral}{'wmin'}:wmin, \textcolor{stringliteral}{'wmax'}:wmax, \textcolor{stringliteral}{'tmin'}:1.e+4, \textcolor{stringliteral}{'tmax'}:1.e+9\}
00154         \textcolor{comment}{#  now do the bare ions from H thru Zn}
00155         \textcolor{comment}{#  these are only involved in the continuum}
00156         \textcolor{keywordflow}{for} iz \textcolor{keywordflow}{in} range(1, 31):
00157             ions = \hyperlink{namespacepyneb_1_1utils_1_1__chianti__tools_a7d6debb5f68b52f64c311ca1fe99945c}{zion2name}(iz, iz+1)
00158             wmin=0.
00159             wmax = 1.e+30
00160             masterListInfo[ions] = \{\textcolor{stringliteral}{'wmin'}:wmin, \textcolor{stringliteral}{'wmax'}:wmax, \textcolor{stringliteral}{'tmin'}:1.e+4, \textcolor{stringliteral}{'tmax'}:1.e+9\}
00161         pfile = open(infoName, \textcolor{stringliteral}{'w'})
00162         pickle.dump(masterListInfo, pfile)
00163         pfile.close
00164         masterListInfo = \{\textcolor{stringliteral}{'noInfo'}:\textcolor{stringliteral}{'none'}\}
00165     \textcolor{keywordflow}{return} masterListInfo
00166     \textcolor{comment}{#}
00167     \textcolor{comment}{# -------------------------------------------------------------------------------------}
00168     \textcolor{comment}{#}
\hypertarget{__chianti__tools_8py_source_l00169}{}\hyperlink{namespacepyneb_1_1utils_1_1__chianti__tools_a1a4447320a9d614f994bc538890c32cd}{00169} \textcolor{keyword}{def }\hyperlink{namespacepyneb_1_1utils_1_1__chianti__tools_a1a4447320a9d614f994bc538890c32cd}{masterListRead}():
00170     \textcolor{stringliteral}{""" read a chianti masterlist file and return a list of files"""}
00171     dir=os.environ[\textcolor{stringliteral}{"XUVTOP"}]
00172     fname=os.path.join(dir,\textcolor{stringliteral}{'masterlist'},\textcolor{stringliteral}{'masterlist.ions'})
00173     input=open(fname,\textcolor{stringliteral}{'}\textcolor{stringliteral}{r')}
00174 \textcolor{stringliteral}{    s1=input.readlines()}
00175 \textcolor{stringliteral}{    dum=input.close()}
00176 \textcolor{stringliteral}{    masterlist=[]}
00177 \textcolor{stringliteral}{    }\textcolor{keywordflow}{for} i \textcolor{keywordflow}{in} range(0,len(s1)):
00178         s1a=s1[i][:-1]
00179         s2=s1a.split(\textcolor{stringliteral}{';'})
00180         masterlist.append(s2[0].strip())
00181     \textcolor{keywordflow}{return} masterlist
00182     \textcolor{comment}{#}
00183     \textcolor{comment}{# -------------------------------------------------------------------------------------}
00184     \textcolor{comment}{#}
\hypertarget{__chianti__tools_8py_source_l00185}{}\hyperlink{namespacepyneb_1_1utils_1_1__chianti__tools_ac5c830808b3221d5f11b3e2d83b6140d}{00185} \textcolor{keyword}{def }\hyperlink{namespacepyneb_1_1utils_1_1__chianti__tools_ac5c830808b3221d5f11b3e2d83b6140d}{photoxRead}(ions):
00186     \textcolor{stringliteral}{"""read chianti photoionization .photox files and return}
00187 \textcolor{stringliteral}{        \{"energy", "cross"\} where energy is in Rydbergs and the}
00188 \textcolor{stringliteral}{        cross section is in cm^2  """}
00189     \textcolor{comment}{#}
00190     zion=\hyperlink{namespacepyneb_1_1utils_1_1__chianti__tools_a92cf299ad3407ee8923739e2761ab13f}{convertName}(ions)
00191     \textcolor{keywordflow}{if} zion[\textcolor{stringliteral}{'Z'}] < zion[\textcolor{stringliteral}{'Ion'}]:
00192         print((\textcolor{stringliteral}{' this is a bare nucleus that has no ionization rate'}))
00193         \textcolor{keywordflow}{return}
00194     \textcolor{comment}{#}
00195     fname=\hyperlink{namespacepyneb_1_1utils_1_1__chianti__tools_ad4bc7b577fd4c3819ceb00b0a444351b}{ion2filename}(ions)
00196     paramname=fname+\textcolor{stringliteral}{'.photox'}
00197     input=open(paramname,\textcolor{stringliteral}{'}\textcolor{stringliteral}{r')}
00198 \textcolor{stringliteral}{    lines = input.readlines()}
00199 \textcolor{stringliteral}{    input.close}
00200 \textcolor{stringliteral}{    }\textcolor{comment}{# get number of energies}
00201 \textcolor{comment}{#    neng = int(lines[0][0:6])}
00202     dataEnd = 0
00203     lvl1 = []
00204     lvl2 = []
00205     energy = []
00206     cross = []
00207     icounter = 0
00208     \textcolor{keywordflow}{while} \textcolor{keywordflow}{not} dataEnd:
00209         lvl11 = int(lines[icounter][:8])
00210         lvl21 = int(lines[icounter][8:15])
00211         ener = lines[icounter][15:].split()
00212         energy1 = np.asarray(ener, \textcolor{stringliteral}{'float64'})
00213         \textcolor{comment}{#}
00214         icounter += 1
00215         irsl = int(lines[icounter][:8])
00216         ind0 = int(lines[icounter][8:15])
00217         \textcolor{keywordflow}{if} irsl != lvl11 \textcolor{keywordflow}{or} ind0 != lvl21:
00218             \textcolor{comment}{# this only happens if the file was written incorrectly}
00219             print((\textcolor{stringliteral}{' lvl1, lvl2 = %7i %7i'}%(lvl11, lvl21)))
00220             print((\textcolor{stringliteral}{' irsl, indo = %7i %7i'}%(irsl,  ind0)))
00221             \textcolor{keywordflow}{return}
00222         crs = lines[icounter][15:].split()
00223         cross1 = np.asarray(crs, \textcolor{stringliteral}{'float64'})
00224         lvl1.append(lvl11)
00225         lvl2.append(lvl21)
00226         energy.append(energy1)
00227         cross.append(cross1)
00228         icounter += 1
00229         dataEnd = lines[icounter].count(\textcolor{stringliteral}{'-1'})
00230     ref = lines[icounter+1:-1]
00231     cross = np.asarray(cross, \textcolor{stringliteral}{'float64'})
00232     energy = np.asarray(energy, \textcolor{stringliteral}{'float64'})
00233     \textcolor{keywordflow}{return} \{\textcolor{stringliteral}{'lvl1'}:lvl1, \textcolor{stringliteral}{'lvl2'}:lvl2,\textcolor{stringliteral}{'energy'}:energy, \textcolor{stringliteral}{'cross'}:cross,  \textcolor{stringliteral}{'ref'}:ref\}
00234     \textcolor{comment}{#}
00235     \textcolor{comment}{# -------------------------------------------------------------------------------------}
00236     \textcolor{comment}{#}
\hypertarget{__chianti__tools_8py_source_l00237}{}\hyperlink{namespacepyneb_1_1utils_1_1__chianti__tools_a552566dde90c6b3d2d8e934914d3ae76}{00237} \textcolor{keyword}{def }\hyperlink{namespacepyneb_1_1utils_1_1__chianti__tools_a552566dde90c6b3d2d8e934914d3ae76}{ionrecdatRead}(filename):
00238     \textcolor{stringliteral}{""" read chianti ionxdat, ionizdat, recombdat files and return}
00239 \textcolor{stringliteral}{    \{"ev":ev,"cross":cross,"crosserr":crosserr,"ref":ref\}  not tested """}
00240     \textcolor{comment}{#}
00241     input=open(filename,\textcolor{stringliteral}{'}\textcolor{stringliteral}{r')}
00242 \textcolor{stringliteral}{    ionrec=input.readlines()}
00243 \textcolor{stringliteral}{    dum=input.close()}
00244 \textcolor{stringliteral}{    }\textcolor{comment}{#}
00245     \textcolor{comment}{# first get the number of data lines}
00246     ndata=2
00247     iline=0
00248     \textcolor{keywordflow}{while} ndata > 1:
00249         s2=ionrec[iline].split()
00250         ndata=len(s2)
00251         iline=iline+1
00252     nline=iline-1
00253     \textcolor{comment}{#}
00254     x=np.zeros(nline,\textcolor{stringliteral}{'Float64'})
00255     y=np.zeros(nline,\textcolor{stringliteral}{'Float64'})
00256     yerr=np.zeros(nline,\textcolor{stringliteral}{'Float64'})
00257 \textcolor{comment}{#}
00258     \textcolor{keywordflow}{for} iline \textcolor{keywordflow}{in} range(0,nline):
00259         ndata=len
00260         s2=ionrec[iline].split()
00261         ndata=len(s2)
00262         \textcolor{keywordflow}{if} ndata == 2:
00263             x[iline]=float(s2[0])
00264             y[iline]=float(s2[1])
00265             yerr[iline]=float(0.)
00266         \textcolor{keywordflow}{else}:
00267             x[iline]=float(s2[0])
00268             y[iline]=float(s2[1])
00269             yerr[iline]=float(s2[2])
00270     \textcolor{comment}{#}
00271     ref=[]
00272     \textcolor{keywordflow}{for} iline \textcolor{keywordflow}{in} range(nline+1,len(ionrec)-1):
00273         s1a=ionrec[iline][:-1]
00274         ref.append(s1a.strip())
00275 
00276 
00277     ionrecdat=\{\textcolor{stringliteral}{"x"}:x,\textcolor{stringliteral}{"y"}:y,\textcolor{stringliteral}{"yerr"}:yerr,\textcolor{stringliteral}{"ref"}:ref\}
00278     \textcolor{keywordflow}{return} ionrecdat
00279     \textcolor{comment}{#}
00280     \textcolor{comment}{# --------------------------------------------------}
00281     \textcolor{comment}{#}
\hypertarget{__chianti__tools_8py_source_l00282}{}\hyperlink{namespacepyneb_1_1utils_1_1__chianti__tools_a0e092ff402fc0287bcaaed67171934ea}{00282} \textcolor{keyword}{def }\hyperlink{namespacepyneb_1_1utils_1_1__chianti__tools_a0e092ff402fc0287bcaaed67171934ea}{z2element}(z):
00283     \textcolor{stringliteral}{""" convert Z to element string """}
00284     \textcolor{keywordflow}{if} z-1 < len(const.El):
00285         thisel=const.El[z-1]
00286     \textcolor{keywordflow}{else}:
00287         thisel=\textcolor{stringliteral}{''}
00288     \textcolor{keywordflow}{return} thisel
00289     \textcolor{comment}{#}
00290     \textcolor{comment}{# -------------------------------------------------------------------------------------}
00291     \textcolor{comment}{#}
\hypertarget{__chianti__tools_8py_source_l00292}{}\hyperlink{namespacepyneb_1_1utils_1_1__chianti__tools_a7d6debb5f68b52f64c311ca1fe99945c}{00292} \textcolor{keyword}{def }\hyperlink{namespacepyneb_1_1utils_1_1__chianti__tools_a7d6debb5f68b52f64c311ca1fe99945c}{zion2name}(z,ion, dielectronic=False):
00293     \textcolor{stringliteral}{""" convert Z, ion to generic name  26, 13 -> fe\_13 """}
00294     \textcolor{keywordflow}{if} (z-1 < len(const.El)) \textcolor{keywordflow}{and} (ion <= z+1):
00295         thisone=const.El[z-1]+\textcolor{stringliteral}{'\_'}+str(ion)
00296         \textcolor{keywordflow}{if} dielectronic:
00297             thisone+=\textcolor{stringliteral}{'d'}
00298     \textcolor{keywordflow}{else}:
00299         thisone=\textcolor{stringliteral}{''}
00300     \textcolor{keywordflow}{return} thisone
00301     \textcolor{comment}{#}
00302     \textcolor{comment}{# -------------------------------------------------------------------------------------}
00303     \textcolor{comment}{#}
\hypertarget{__chianti__tools_8py_source_l00304}{}\hyperlink{namespacepyneb_1_1utils_1_1__chianti__tools_a46370e972711f8737755fa9c8fb57ed8}{00304} \textcolor{keyword}{def }\hyperlink{namespacepyneb_1_1utils_1_1__chianti__tools_a46370e972711f8737755fa9c8fb57ed8}{zion2filename}(z,ion, dielectronic=False):
00305     \textcolor{stringliteral}{""" convert Z to generic file name string """}
00306     dir=os.environ[\textcolor{stringliteral}{"XUVTOP"}]
00307     \textcolor{keywordflow}{if} (z-1 < len(const.El)) \textcolor{keywordflow}{and} (ion <= z+1):
00308         thisel=const.El[z-1]
00309     \textcolor{keywordflow}{else}:
00310         thisel=\textcolor{stringliteral}{''}
00311     \textcolor{keywordflow}{if} z-1 < len(const.El):
00312         thisone=const.El[z-1]+\textcolor{stringliteral}{'\_'}+str(ion)
00313         \textcolor{keywordflow}{if} dielectronic:
00314             thisone+=\textcolor{stringliteral}{'d'}
00315     \textcolor{keywordflow}{else}:
00316         thisone=\textcolor{stringliteral}{''}
00317     \textcolor{keywordflow}{if} thisel != \textcolor{stringliteral}{''} :
00318         fname=os.path.join(dir,thisel,thisone,thisone)
00319     \textcolor{keywordflow}{return} fname
00320     \textcolor{comment}{#}
00321     \textcolor{comment}{# -------------------------------------------------------------------------------------}
00322     \textcolor{comment}{#}
\hypertarget{__chianti__tools_8py_source_l00323}{}\hyperlink{namespacepyneb_1_1utils_1_1__chianti__tools_aa1c272dcccda8670d22cf866cd0e9b5f}{00323} \textcolor{keyword}{def }\hyperlink{namespacepyneb_1_1utils_1_1__chianti__tools_aa1c272dcccda8670d22cf866cd0e9b5f}{zion2localFilename}(z,ion, dielectronic=False):
00324     \textcolor{stringliteral}{""" convert Z to generic file name string with current directory at top"""}
00325     dir=\textcolor{stringliteral}{'.'}
00326     \textcolor{keywordflow}{if} (z-1 < len(const.El)) \textcolor{keywordflow}{and} (ion <= z+1):
00327         thisel=const.El[z-1]
00328     \textcolor{keywordflow}{else}:
00329         thisel=\textcolor{stringliteral}{''}
00330     \textcolor{keywordflow}{if} z-1 < len(const.El):
00331         thisone=const.El[z-1]+\textcolor{stringliteral}{'\_'}+str(ion)
00332         \textcolor{keywordflow}{if} dielectronic:
00333             thisone+=\textcolor{stringliteral}{'d'}
00334     \textcolor{keywordflow}{else}:
00335         thisone=\textcolor{stringliteral}{''}
00336     \textcolor{keywordflow}{if} thisel != \textcolor{stringliteral}{''} :
00337         fname=os.path.join(dir,thisel,thisone,thisone)
00338     \textcolor{keywordflow}{return} fname
00339     \textcolor{comment}{#}
00340     \textcolor{comment}{# -------------------------------------------------------------------------------------}
00341     \textcolor{comment}{#}
\hypertarget{__chianti__tools_8py_source_l00342}{}\hyperlink{namespacepyneb_1_1utils_1_1__chianti__tools_a4c5a7ace6a222c4cfad8f00cf2b68554}{00342} \textcolor{keyword}{def }\hyperlink{namespacepyneb_1_1utils_1_1__chianti__tools_a4c5a7ace6a222c4cfad8f00cf2b68554}{zion2spectroscopic}(z,ion, dielectronic=False):
00343     \textcolor{stringliteral}{""" convert Z and ion to spectroscopic notation string """}
00344     \textcolor{keywordflow}{if} (z-1 < len(const.El)) \textcolor{keywordflow}{and} (ion <= z+1):
00345         spect=const.El[z-1].capitalize()+\textcolor{stringliteral}{' '}+const.Ionstage[ion-1]
00346         \textcolor{keywordflow}{if} dielectronic:
00347             spect+=\textcolor{stringliteral}{' d'}
00348     \textcolor{keywordflow}{else}:  spect = \textcolor{stringliteral}{''}
00349     \textcolor{keywordflow}{return} spect
00350     \textcolor{comment}{#}
00351     \textcolor{comment}{# -------------------------------------------------------------------------------------}
00352     \textcolor{comment}{#}
\hypertarget{__chianti__tools_8py_source_l00353}{}\hyperlink{namespacepyneb_1_1utils_1_1__chianti__tools_a92cf299ad3407ee8923739e2761ab13f}{00353} \textcolor{keyword}{def }\hyperlink{namespacepyneb_1_1utils_1_1__chianti__tools_a92cf299ad3407ee8923739e2761ab13f}{convertName}(name):
00354     \textcolor{stringliteral}{""" convert ion name string to Z and Ion """}
00355     s2=name.split(\textcolor{stringliteral}{'\_'})
00356     els=s2[0].strip()
00357     i1=const.El.index(els)+1
00358     ions=s2[1].strip()
00359     d=ions.find(\textcolor{stringliteral}{'d'})
00360     \textcolor{keywordflow}{if} d >0 :
00361         dielectronic=\textcolor{keyword}{True}
00362         ions=ions.replace(\textcolor{stringliteral}{'d'},\textcolor{stringliteral}{''})
00363     \textcolor{keywordflow}{else}: dielectronic=\textcolor{keyword}{False}
00364     \textcolor{keywordflow}{return} \{\textcolor{stringliteral}{'Z'}:int(i1),\textcolor{stringliteral}{'Ion'}:int(ions),\textcolor{stringliteral}{'Dielectronic'}:dielectronic, \textcolor{stringliteral}{'Element'}:els\}
00365     \textcolor{comment}{#}
00366     \textcolor{comment}{# -------------------------------------------------------------------------------------}
00367     \textcolor{comment}{#}
\hypertarget{__chianti__tools_8py_source_l00368}{}\hyperlink{namespacepyneb_1_1utils_1_1__chianti__tools_a235ada2c4e384f436dbade0913107585}{00368} \textcolor{keyword}{def }\hyperlink{namespacepyneb_1_1utils_1_1__chianti__tools_a235ada2c4e384f436dbade0913107585}{defaultsRead}(verbose=0):
00369     \textcolor{comment}{#}
00370     \textcolor{comment}{#possibleDefaults = \{'wavelength':['angstrom', 'kev', 'nm']\}}
00371     \textcolor{comment}{#symbolDefaults = \{'wavelength':['A', 'keV', 'nm']\}}
00372     initDefaults=\{\textcolor{stringliteral}{'abundfile'}: \textcolor{stringliteral}{'sun\_photospheric\_1998\_grevesse'},\textcolor{stringliteral}{'ioneqfile'}: \textcolor{stringliteral}{'chianti'}, \textcolor{stringliteral}{'wavelength'}: \textcolor{stringliteral}{'
      angstrom'}, \textcolor{stringliteral}{'flux'}: \textcolor{stringliteral}{'energy'},\textcolor{stringliteral}{'gui'}:\textcolor{keyword}{False}\}
00373     rcfile=os.path.join(os.environ[\textcolor{stringliteral}{'HOME'}],\textcolor{stringliteral}{'.chianti/chiantirc'})
00374     \textcolor{keywordflow}{if} os.path.isfile(rcfile):
00375         print((\textcolor{stringliteral}{' reading chiantirc file'}))
00376         config = RawConfigParser(initDefaults)
00377         config.read(rcfile)
00378         defaults = \{\}
00379         \textcolor{keywordflow}{for} anitem \textcolor{keywordflow}{in} config.items(\textcolor{stringliteral}{'chianti'}):
00380             defaults[anitem[0]] = anitem[1]
00381         \textcolor{keywordflow}{if} defaults[\textcolor{stringliteral}{'gui'}].lower() \textcolor{keywordflow}{in} (\textcolor{stringliteral}{'t'}, \textcolor{stringliteral}{'y'}, \textcolor{stringliteral}{'yes'}, \textcolor{stringliteral}{'on'}, \textcolor{stringliteral}{'true'}, \textcolor{stringliteral}{'1'}, 1, \textcolor{keyword}{True}):
00382             defaults[\textcolor{stringliteral}{'gui'}] = \textcolor{keyword}{True}
00383         \textcolor{keywordflow}{elif} defaults[\textcolor{stringliteral}{'gui'}].lower() \textcolor{keywordflow}{in} (\textcolor{stringliteral}{'f'}, \textcolor{stringliteral}{'n'}, \textcolor{stringliteral}{'no'}, \textcolor{stringliteral}{'off'}, \textcolor{stringliteral}{'false'}, \textcolor{stringliteral}{'0'}, 0, \textcolor{keyword}{False}):
00384             defaults[\textcolor{stringliteral}{'gui'}] = \textcolor{keyword}{False}
00385     \textcolor{keywordflow}{else}:
00386         defaults = initDefaults
00387         \textcolor{keywordflow}{if} verbose:
00388             print((\textcolor{stringliteral}{' chiantirc file (/HOME/.chianti/chiantirc) does not exist'}))
00389             print((\textcolor{stringliteral}{' using the following defaults'}))
00390             \textcolor{keywordflow}{for} akey \textcolor{keywordflow}{in} list(defaults.keys()):
00391                 print((\textcolor{stringliteral}{' %s = %s'}%(akey, defaults[akey])))
00392     \textcolor{keywordflow}{return} defaults
00393     \textcolor{comment}{#}
00394     \textcolor{comment}{# -------------------------------------------------------------------------------------}
00395     \textcolor{comment}{#}
\hypertarget{__chianti__tools_8py_source_l00396}{}\hyperlink{namespacepyneb_1_1utils_1_1__chianti__tools_ad4bc7b577fd4c3819ceb00b0a444351b}{00396} \textcolor{keyword}{def }\hyperlink{namespacepyneb_1_1utils_1_1__chianti__tools_ad4bc7b577fd4c3819ceb00b0a444351b}{ion2filename}(ions):
00397     \textcolor{stringliteral}{""" convert ion string to generic file name string """}
00398     dir=os.environ[\textcolor{stringliteral}{"XUVTOP"}]
00399     zion=\hyperlink{namespacepyneb_1_1utils_1_1__chianti__tools_a92cf299ad3407ee8923739e2761ab13f}{convertName}(ions)
00400     el=\hyperlink{namespacepyneb_1_1utils_1_1__chianti__tools_a0e092ff402fc0287bcaaed67171934ea}{z2element}(zion[\textcolor{stringliteral}{'Z'}])
00401     fname=os.path.join(dir,el,ions,ions)
00402     \textcolor{keywordflow}{return} fname
00403     \textcolor{comment}{#}
00404     \textcolor{comment}{# -------------------------------------------------------------------------------------}
00405     \textcolor{comment}{#}
\hypertarget{__chianti__tools_8py_source_l00406}{}\hyperlink{namespacepyneb_1_1utils_1_1__chianti__tools_a32ffd2ba65d19385acc6921d77da7976}{00406} \textcolor{keyword}{def }\hyperlink{namespacepyneb_1_1utils_1_1__chianti__tools_a32ffd2ba65d19385acc6921d77da7976}{el2z}(els):
00407     \textcolor{stringliteral}{""" from an the name of the element (1-2 letter) return Z"""}
00408     z=const.El.index(els.lower())+1
00409     \textcolor{keywordflow}{return} z
00410     \textcolor{comment}{#}
00411     \textcolor{comment}{# -------------------------------------------------------------------------------------}
00412     \textcolor{comment}{#}
\hypertarget{__chianti__tools_8py_source_l00413}{}\hyperlink{namespacepyneb_1_1utils_1_1__chianti__tools_a25c22edea4645bdc590eed4a1c86fea8}{00413} \textcolor{keyword}{def }\hyperlink{namespacepyneb_1_1utils_1_1__chianti__tools_a25c22edea4645bdc590eed4a1c86fea8}{abundanceRead}(abundancename=''):
00414     \textcolor{stringliteral}{""" read an abundanc file and returns the abundance values relative to hydrogen"""}
00415     \textcolor{keywordflow}{pass}
00416 
00417     \textcolor{comment}{#}
00418     \textcolor{comment}{# -------------------------------------------------------------------------------------}
00419     \textcolor{comment}{#}
\hypertarget{__chianti__tools_8py_source_l00420}{}\hyperlink{namespacepyneb_1_1utils_1_1__chianti__tools_af9318412aecf0f6ac826b43796d594db}{00420} \textcolor{keyword}{def }\hyperlink{namespacepyneb_1_1utils_1_1__chianti__tools_af9318412aecf0f6ac826b43796d594db}{qrp}(z,u):
00421     \textcolor{stringliteral}{''' qrp(Z,u)  u = E/IP}
00422 \textcolor{stringliteral}{    calculate Qr-prime (equ. 2.12) of Fontes, Sampson and Zhang 1999'''}
00423     \textcolor{comment}{#}
00424     aa=1.13  \textcolor{comment}{# aa stands for A in equ 2.12}
00425     \textcolor{comment}{#}
00426     \textcolor{keywordflow}{if} z >= 16 :
00427         \textcolor{comment}{# use Fontes Z=20, N=1 parameters}
00428         dd=3.70590
00429         c=-0.28394
00430         d=1.95270
00431         cc=0.20594
00432     \textcolor{keywordflow}{else}:
00433     \textcolor{comment}{# use Fontes Z=10, N=2 parameters}
00434         dd=3.82652
00435         c=-0.80414
00436         d=2.32431
00437         cc=0.14424
00438     \textcolor{comment}{#}
00439     \textcolor{keywordflow}{if} z > 20:
00440         cc+=((z-20.)/50.5)**1.11
00441     \textcolor{comment}{#}
00442     bu=u <= 1.
00443     q=np.ma.array(u, \textcolor{stringliteral}{'Float64'}, mask=bu, fill\_value=0.)
00444     \textcolor{comment}{#}
00445     \textcolor{comment}{#}
00446     q=(aa*np.ma.log(u) + dd*(1.-1./u)**2 + cc*u*(1.-1./u)**4 + (c/u+d/u**2)*(1.-1/u))/u
00447     \textcolor{comment}{#}
00448     q.set\_fill\_value(0.)  \textcolor{comment}{# I don't know why this is necessary}
00449     \textcolor{keywordflow}{return} q  \textcolor{comment}{#  .set\_fill\_value(0.)}
00450     \textcolor{comment}{#}
00451     \textcolor{comment}{# -------------------------------------------------------------------------------------}
00452     \textcolor{comment}{#}
\hypertarget{__chianti__tools_8py_source_l00453}{}\hyperlink{namespacepyneb_1_1utils_1_1__chianti__tools_ac848d0b5ea14bf4adf6e8cd5d46fb639}{00453} \textcolor{keyword}{def }\hyperlink{namespacepyneb_1_1utils_1_1__chianti__tools_ac848d0b5ea14bf4adf6e8cd5d46fb639}{elvlcRead}(ions, filename = None, verbose=0,  useTh=1):
00454     \textcolor{stringliteral}{"""}
00455 \textcolor{stringliteral}{    read a chianti energy level file and returns}
00456 \textcolor{stringliteral}{    \{"lvl":lvl,"conf":conf,"term":term,"spin":spin,"l":l,"spd":spd,"j":j}
00457 \textcolor{stringliteral}{    ,"mult":mult,"ecm":ecm,"eryd":eryd,"ecmth":ecmth,"erydth":erydth,"ref":ref,"pretty":pretty, 'ionS'
      :ions\}}
00458 \textcolor{stringliteral}{    if a energy value for ecm or eryd is zero(=unknown), the theoretical values}
00459 \textcolor{stringliteral}{    (ecmth and erydth) are inserted}
00460 \textcolor{stringliteral}{    """}
00461     \textcolor{comment}{#}
00462     fstring=\textcolor{stringliteral}{'i3,i6,a15,i3,i3,a3,f4.1,i3,4f15.2'}
00463     elvlcFormat=\hyperlink{classpyneb_1_1utils_1_1_fortran_format_1_1_fortran_format}{FortranFormat}(fstring)
00464     \textcolor{comment}{#}
00465     \textcolor{keywordflow}{if} type(filename) == NoneType:
00466         fname=\hyperlink{namespacepyneb_1_1utils_1_1__chianti__tools_ad4bc7b577fd4c3819ceb00b0a444351b}{ion2filename}(ions)
00467         elvlname=fname+\textcolor{stringliteral}{'.elvlc'}
00468     \textcolor{keywordflow}{else}:
00469         elvlname = filename
00470         bname = os.path.basename(filename)
00471         ions = bname.split(\textcolor{stringliteral}{'.'})[0]
00472     \textcolor{keywordflow}{if} \textcolor{keywordflow}{not} os.path.isfile(elvlname):
00473         \textcolor{keywordflow}{print} \textcolor{stringliteral}{' elvlc file does not exist:  '},elvlname
00474         \textcolor{keywordflow}{return} \{\textcolor{stringliteral}{'status'}:0\}
00475     status = 1
00476     input=open(elvlname,\textcolor{stringliteral}{'}\textcolor{stringliteral}{r')}
00477 \textcolor{stringliteral}{    s1=input.readlines()}
00478 \textcolor{stringliteral}{    input.close()}
00479 \textcolor{stringliteral}{    nlvls=0}
00480 \textcolor{stringliteral}{    ndata=2}
00481 \textcolor{stringliteral}{    }\textcolor{keywordflow}{while} ndata > 1:
00482         s1a=s1[nlvls][:-1]
00483         s2=s1a.split()
00484         ndata=len(s2)
00485         nlvls=nlvls+1
00486     nlvls-=1
00487     \textcolor{keywordflow}{if} verbose:
00488         \textcolor{keywordflow}{print} \textcolor{stringliteral}{' nlvls = '}, nlvls
00489     lvl=[0]*nlvls
00490     conf=[0]*nlvls
00491     term=[0]*nlvls
00492     spin=[0]*nlvls
00493     l=[0]*nlvls
00494     spd=[0]*nlvls
00495     j=[0]*nlvls
00496     mult=[0]*nlvls
00497     ecm=[0]*nlvls
00498     eryd=[0]*nlvls
00499     ecmth=[0]*nlvls
00500     erydth=[0]*nlvls
00501     pretty=[0]*nlvls
00502     \textcolor{keywordflow}{for} i \textcolor{keywordflow}{in} range(0,nlvls):
00503         \textcolor{keywordflow}{if} verbose:
00504             \textcolor{keywordflow}{print} s1[i][0:115]
00505         inpt=\hyperlink{classpyneb_1_1utils_1_1_fortran_format_1_1_fortran_line}{FortranLine}(s1[i][0:115],elvlcFormat)
00506         lvl[i]=inpt[0]
00507         conf[i]=inpt[1]
00508         term[i]=inpt[2].strip()
00509         spin[i]=inpt[3]
00510         l[i]=inpt[4]
00511         spd[i]=inpt[5].strip()
00512         j[i]=inpt[6]
00513         mult[i]=inpt[7]
00514         ecm[i]=inpt[8]
00515         eryd[i]=inpt[9]
00516         ecmth[i]=inpt[10]
00517         erydth[i]=inpt[11]
00518         \textcolor{keywordflow}{if} ecm[i] == 0.:
00519             \textcolor{keywordflow}{if} useTh:
00520                 ecm[i] = ecmth[i]
00521                 eryd[i] = erydth[i]
00522         stuff = term[i].strip() + \textcolor{stringliteral}{' %1i%1s%3.1f'}%( spin[i], spd[i], j[i])
00523         pretty[i] = stuff.strip()
00524     ref=[]
00525     \textcolor{keywordflow}{for} i \textcolor{keywordflow}{in} range(nlvls+1,len(s1)-1):
00526         s1a=s1[i][:-1]
00527         ref.append(s1a.strip())
00528 \textcolor{comment}{#    self.const.Elvlc=\{"lvl":lvl,"conf":conf,"term":term,"spin":spin,"l":l,"spd":spd,"j":j}
00529 \textcolor{comment}{#            ,"mult":mult,"ecm":ecm,"eryd":eryd,"ecmth":ecmth,"erydth":erydth,"ref":ref\}}
00530     \textcolor{keywordflow}{return} \{\textcolor{stringliteral}{"lvl"}:lvl,\textcolor{stringliteral}{"conf"}:conf,\textcolor{stringliteral}{"term"}:term,\textcolor{stringliteral}{"spin"}:spin,\textcolor{stringliteral}{"l"}:l,\textcolor{stringliteral}{"spd"}:spd,\textcolor{stringliteral}{"j"}:j
00531             ,\textcolor{stringliteral}{"mult"}:mult,\textcolor{stringliteral}{"ecm"}:ecm,\textcolor{stringliteral}{"eryd"}:eryd,\textcolor{stringliteral}{"ecmth"}:ecmth,\textcolor{stringliteral}{"erydth"}:erydth,\textcolor{stringliteral}{"ref"}:ref,\textcolor{stringliteral}{"pretty"}:pretty, \textcolor{stringliteral}{'
      ionS'}:ions, \textcolor{stringliteral}{'status'}:status\}
00532     \textcolor{comment}{#}
00533     \textcolor{comment}{# -------------------------------------------------------------------------------------}
00534     \textcolor{comment}{#}
\hypertarget{__chianti__tools_8py_source_l00535}{}\hyperlink{namespacepyneb_1_1utils_1_1__chianti__tools_a279f0911664046ffd65c19571be8a84d}{00535} \textcolor{keyword}{def }\hyperlink{namespacepyneb_1_1utils_1_1__chianti__tools_a279f0911664046ffd65c19571be8a84d}{elvlcWrite}(info):
00536     \textcolor{stringliteral}{''' creates a .elvlc in the current directory}
00537 \textcolor{stringliteral}{    info is a dictionary that must contain the following keys}
00538 \textcolor{stringliteral}{    ionS, the Chianti style name of the ion such as c\_4}
00539 \textcolor{stringliteral}{    conf, an integer denoting the configuration - not too essential}
00540 \textcolor{stringliteral}{    term, a string showing the configuration}
00541 \textcolor{stringliteral}{    spin, an integer of the spin of the state in LS coupling}
00542 \textcolor{stringliteral}{    l, an integer of the angular momentum quantum number}
00543 \textcolor{stringliteral}{    spd, an string for the alphabetic symbol of the angular momemtum, S, P, D, etc}
00544 \textcolor{stringliteral}{    j, a floating point number, the total angular momentum}
00545 \textcolor{stringliteral}{    ecm, the observed energy in inverse cm, if unknown, the value is 0.}
00546 \textcolor{stringliteral}{    eryd, the observed energy in Rydbergs, if unknown, the value is 0.}
00547 \textcolor{stringliteral}{    ecmth, the calculated energy from the scattering calculation, in inverse cm}
00548 \textcolor{stringliteral}{    erydth, the calculated energy from the scattering calculation in Rydbergs}
00549 \textcolor{stringliteral}{    ref, the references in the literature to the data in the input info'''}
00550     gname = info[\textcolor{stringliteral}{'ionS'}]
00551     elvlcname = gname + \textcolor{stringliteral}{'.elvlc'}
00552     \textcolor{keywordflow}{print} \textcolor{stringliteral}{' elvlc file name = '}, elvlcname
00553     out = open(elvlcname, \textcolor{stringliteral}{'w'})
00554     \textcolor{keywordflow}{for} i,  conf \textcolor{keywordflow}{in} enumerate(info[\textcolor{stringliteral}{'conf'}]):
00555         mult = int(2.*info[\textcolor{stringliteral}{'j'}][i]+1.)
00556         pstring = \textcolor{stringliteral}{'%3i%6s%15s%3i%3i%2s%5.1f%3i%15.3f%15.6f%15.3f%15.6f \(\backslash\)n'}%(i+1, conf, info[\textcolor{stringliteral}{'term'}][i], 
      info[\textcolor{stringliteral}{'spin'}][i], info[\textcolor{stringliteral}{'l'}][i], info[\textcolor{stringliteral}{'spd'}][i], info[\textcolor{stringliteral}{'j'}][i], mult, info[\textcolor{stringliteral}{'ecm'}][i], info[\textcolor{stringliteral}{'eryd'}][i], info[\textcolor{stringliteral}{'ecmth
      '}][i], info[\textcolor{stringliteral}{'erydth'}][i])
00557     \textcolor{comment}{#i3,a6,a15,2i3,a2,f5.1,i3,f15.3,f15.6,f15.3,f15.6}
00558         out.write(pstring)
00559     out.write(\textcolor{stringliteral}{' -1\(\backslash\)n'})
00560     out.write(\textcolor{stringliteral}{'%filename:  '} + elvlcname + \textcolor{stringliteral}{'\(\backslash\)n'})
00561     info[\textcolor{stringliteral}{'ref'}].append(\textcolor{stringliteral}{' produced as a part of the George Mason University, University of Cambridge,
       University of Michigan \(\backslash\)'CHIANTI\(\backslash\)' atomic database for astrophysical spectroscopy consortium'})
00562     today = date.today()
00563     info[\textcolor{stringliteral}{'ref'}].append(\textcolor{stringliteral}{' K. Dere (GMU) - '} + today.strftime(\textcolor{stringliteral}{'%Y %B %d'}))
00564     \textcolor{keywordflow}{for} one \textcolor{keywordflow}{in} info[\textcolor{stringliteral}{'ref'}]:
00565         out.write(one+\textcolor{stringliteral}{'\(\backslash\)n'})
00566     out.write(\textcolor{stringliteral}{' -1\(\backslash\)n'})
00567     out.close()
00568     \textcolor{keywordflow}{return}
00569     \textcolor{comment}{#}
00570     \textcolor{comment}{# -------------------------------------------------------------------------------------}
00571     \textcolor{comment}{#}
\hypertarget{__chianti__tools_8py_source_l00572}{}\hyperlink{namespacepyneb_1_1utils_1_1__chianti__tools_a3ef36a1d0a4df4cab94a392d2a3da980}{00572} \textcolor{keyword}{def }\hyperlink{namespacepyneb_1_1utils_1_1__chianti__tools_a3ef36a1d0a4df4cab94a392d2a3da980}{wgfaRead}(ions, filename=0, elvlcname=0, total=0):
00573     \textcolor{stringliteral}{"""}
00574 \textcolor{stringliteral}{    reads chianti wgfa file and returns}
00575 \textcolor{stringliteral}{    \{"lvl1":lvl1,"lvl2":lvl2,"wvl":wvl,"gf":gf,"avalue":avalue,"ref":ref\}}
00576 \textcolor{stringliteral}{    if elvlcname is specified, the lsj term labels are returned as 'pretty1' and 'pretty2'}
00577 \textcolor{stringliteral}{    """}
00578     \textcolor{comment}{#}
00579     \textcolor{keywordflow}{if} filename:
00580         wgfaname = filename
00581         \textcolor{keywordflow}{if} \textcolor{keywordflow}{not} filename:
00582             elvlcname = os.path.splitext(wgfaname)[0] + \textcolor{stringliteral}{'.elvlc'}
00583     \textcolor{keywordflow}{else}:
00584         fname=\hyperlink{namespacepyneb_1_1utils_1_1__chianti__tools_ad4bc7b577fd4c3819ceb00b0a444351b}{ion2filename}(ions)
00585         wgfaname=fname+\textcolor{stringliteral}{'.wgfa'}
00586         elvlcname = fname + \textcolor{stringliteral}{'.elvlc'}
00587     \textcolor{comment}{#}
00588     \textcolor{keywordflow}{if} os.path.isfile(elvlcname):
00589         elvlc = \hyperlink{namespacepyneb_1_1utils_1_1__chianti__tools_ac848d0b5ea14bf4adf6e8cd5d46fb639}{elvlcRead}(\textcolor{stringliteral}{''}, elvlcname)
00590     \textcolor{keywordflow}{else}:
00591         elvlc = 0
00592     input=open(wgfaname,\textcolor{stringliteral}{'}\textcolor{stringliteral}{r')}
00593 \textcolor{stringliteral}{    s1=input.readlines()}
00594 \textcolor{stringliteral}{    dum=input.close()}
00595 \textcolor{stringliteral}{    nwvl=0}
00596 \textcolor{stringliteral}{    ndata=2}
00597 \textcolor{stringliteral}{    }\textcolor{keywordflow}{while} ndata > 1:
00598         s1a=s1[nwvl][:-1]
00599         s2=s1a.split()
00600         ndata=len(s2)
00601         nwvl=nwvl+1
00602     nwvl=nwvl-1
00603     lvl1=[0]*nwvl
00604     lvl2=[0]*nwvl
00605     wvl=[0.]*nwvl
00606     gf=[0.]*nwvl
00607     avalue=[0.]*nwvl
00608     \textcolor{keywordflow}{if} elvlcname:
00609         pretty1 = [\textcolor{stringliteral}{''}]*nwvl
00610         pretty2 = [\textcolor{stringliteral}{''}]*nwvl
00611     \textcolor{comment}{#}
00612     wgfaFormat=\textcolor{stringliteral}{'(2i5,f15.3,2e15.3)'}
00613     \textcolor{keywordflow}{for} i \textcolor{keywordflow}{in} range(nwvl):
00614         inpt=\hyperlink{classpyneb_1_1utils_1_1_fortran_format_1_1_fortran_line}{FortranLine}(s1[i],wgfaFormat)
00615         lvl1[i]=inpt[0]
00616         lvl2[i]=inpt[1]
00617         wvl[i]=inpt[2]
00618         gf[i]=inpt[3]
00619         avalue[i]=inpt[4]
00620         \textcolor{keywordflow}{if} elvlcname:
00621             pretty1[i] = elvlc[\textcolor{stringliteral}{'pretty'}][inpt[0] - 1]
00622             pretty2[i] = elvlc[\textcolor{stringliteral}{'pretty'}][inpt[1] - 1]
00623 
00624     ref=[]
00625     \textcolor{keywordflow}{for} i \textcolor{keywordflow}{in} range(nwvl+1,len(s1)-1):
00626         s1a=s1[i][:-1]
00627         ref.append(s1a.strip())
00628     Wgfa=\{\textcolor{stringliteral}{"lvl1"}:lvl1,\textcolor{stringliteral}{"lvl2"}:lvl2,\textcolor{stringliteral}{"wvl"}:wvl,\textcolor{stringliteral}{"gf"}:gf,\textcolor{stringliteral}{"avalue"}:avalue,\textcolor{stringliteral}{"ref"}:ref, \textcolor{stringliteral}{'ionS'}:ions, \textcolor{stringliteral}{'filename'}:
      wgfaname\}
00629     \textcolor{keywordflow}{if} total:
00630         avalueLvl = [0.]*max(lvl2)
00631         \textcolor{keywordflow}{for} iwvl \textcolor{keywordflow}{in} range(nwvl):
00632             avalueLvl[lvl2[iwvl] -1] += avalue[iwvl]
00633         Wgfa[\textcolor{stringliteral}{'avalueLvl'}] = avalueLvl
00634 
00635     \textcolor{keywordflow}{if} elvlcname:
00636         Wgfa[\textcolor{stringliteral}{'pretty1'}] = pretty1
00637         Wgfa[\textcolor{stringliteral}{'pretty2'}] = pretty2
00638     \textcolor{keywordflow}{return} Wgfa
00639     \textcolor{comment}{#}
00640     \textcolor{comment}{# --------------------------------------}
00641     \textcolor{comment}{#}
\hypertarget{__chianti__tools_8py_source_l00642}{}\hyperlink{namespacepyneb_1_1utils_1_1__chianti__tools_a300c13ee6815450bc20d25a2c6dcb8a8}{00642} \textcolor{keyword}{def }\hyperlink{namespacepyneb_1_1utils_1_1__chianti__tools_a300c13ee6815450bc20d25a2c6dcb8a8}{wgfaWrite}(info, outfile = 0, minBranch = 0.):
00643     \textcolor{stringliteral}{'''}
00644 \textcolor{stringliteral}{    to write a wgfa file}
00645 \textcolor{stringliteral}{    info is a dictionary the contains the following elements}
00646 \textcolor{stringliteral}{    ionS, the Chianti style name of the ion such as c\_4 for C IV}
00647 \textcolor{stringliteral}{    lvl1 - the lower level, the ground level is 1}
00648 \textcolor{stringliteral}{    lvl2 - the upper level}
00649 \textcolor{stringliteral}{    wvl - the wavelength in Angstroms}
00650 \textcolor{stringliteral}{    gf - the weighted oscillator strength}
00651 \textcolor{stringliteral}{    avalue - the A value}
00652 \textcolor{stringliteral}{    pretty1 - descriptive text of the lower level (optional)}
00653 \textcolor{stringliteral}{    pretty2 - descriptive text of the upper level (optiona)}
00654 \textcolor{stringliteral}{    ref - reference text, a list of strings}
00655 \textcolor{stringliteral}{    minBranch:  the transition must have a branching ratio greater than the specified to be written to the
       file}
00656 \textcolor{stringliteral}{    '''}
00657     \textcolor{comment}{#}
00658 \textcolor{comment}{#    gname = info['ionS']}
00659     \textcolor{keywordflow}{if} outfile:
00660         wgfaname = outfile
00661     \textcolor{keywordflow}{else}:
00662         wgfaname = gname + \textcolor{stringliteral}{'.wgfa'}
00663     print((\textcolor{stringliteral}{' wgfa file name = '}, wgfaname))
00664     \textcolor{keywordflow}{if} minBranch > 0.:
00665         info[\textcolor{stringliteral}{'ref'}].append(\textcolor{stringliteral}{' minimum branching ratio = %10.2e'}%(minBranch))
00666     out = open(wgfaname, \textcolor{stringliteral}{'w'})
00667     ntrans = len(info[\textcolor{stringliteral}{'lvl1'}])
00668     nlvl = max(info[\textcolor{stringliteral}{'lvl2'}])
00669     totalAvalue = np.zeros(nlvl, \textcolor{stringliteral}{'float64'})
00670     \textcolor{keywordflow}{if} \textcolor{stringliteral}{'pretty1'} \textcolor{keywordflow}{in} info:
00671         pformat = \textcolor{stringliteral}{'%5i%5i%15.4f%15.3e%15.3e%30s - %30s'}
00672     \textcolor{keywordflow}{else}:
00673         pformat = \textcolor{stringliteral}{'%5i%5i%15.4f%15.3e%15.3e'}
00674     \textcolor{keywordflow}{for} itrans, avalue \textcolor{keywordflow}{in} enumerate(info[\textcolor{stringliteral}{'avalue'}]):
00675         \textcolor{comment}{# for autoionization transitions, lvl1 can be less than zero}
00676         \textcolor{keywordflow}{if} abs(info[\textcolor{stringliteral}{'lvl1'}][itrans]) > 0 \textcolor{keywordflow}{and} info[\textcolor{stringliteral}{'lvl2'}][itrans] > 0:
00677             totalAvalue[info[\textcolor{stringliteral}{'lvl2'}][itrans] -1] += avalue
00678 
00679     \textcolor{keywordflow}{for} itrans, avalue \textcolor{keywordflow}{in} enumerate(info[\textcolor{stringliteral}{'avalue'}]):
00680         \textcolor{keywordflow}{if} avalue > 0.:
00681             branch = avalue/totalAvalue[info[\textcolor{stringliteral}{'lvl2'}][itrans] -1]
00682         \textcolor{keywordflow}{else}:
00683             branch = 0.
00684         \textcolor{keywordflow}{if} branch > minBranch \textcolor{keywordflow}{and} abs(info[\textcolor{stringliteral}{'lvl1'}][itrans]) > 0 \textcolor{keywordflow}{and} info[\textcolor{stringliteral}{'lvl2'}][itrans] > 0:
00685             \textcolor{keywordflow}{if} \textcolor{stringliteral}{'pretty1'} \textcolor{keywordflow}{in} info:
00686                 \textcolor{comment}{# generally only useful with NIST data}
00687                 \textcolor{keywordflow}{if} \textcolor{stringliteral}{'transType'} \textcolor{keywordflow}{in} info:
00688                     \textcolor{keywordflow}{if} info[\textcolor{stringliteral}{'transType'}][itrans] != \textcolor{stringliteral}{''}:
00689                         lbl2 = info[\textcolor{stringliteral}{'pretty2'}]+\textcolor{stringliteral}{'  '} + info[\textcolor{stringliteral}{'transType'}][itrans]
00690                 \textcolor{keywordflow}{else}:
00691                     lbl2= info[\textcolor{stringliteral}{'pretty2'}][itrans]
00692                 pstring= pformat%(info[\textcolor{stringliteral}{'lvl1'}][itrans], info[\textcolor{stringliteral}{'lvl2'}][itrans], info[\textcolor{stringliteral}{'wvl'}][itrans], info[\textcolor{stringliteral}{'gf
      '}][itrans], avalue, info[\textcolor{stringliteral}{'pretty1'}][itrans].rjust(30), lbl2.ljust(30))
00693                 out.write(pstring+\textcolor{stringliteral}{'\(\backslash\)n'})
00694             \textcolor{keywordflow}{else}:
00695                 pstring= pformat%(info[\textcolor{stringliteral}{'lvl1'}][itrans], info[\textcolor{stringliteral}{'lvl2'}][itrans], info[\textcolor{stringliteral}{'wvl'}][itrans], info[\textcolor{stringliteral}{'gf
      '}][itrans], avalue)
00696                 out.write(pstring+\textcolor{stringliteral}{'\(\backslash\)n'})
00697     out.write(\textcolor{stringliteral}{' -1\(\backslash\)n'})
00698     out.write(\textcolor{stringliteral}{'%filename:  '} + wgfaname + \textcolor{stringliteral}{'\(\backslash\)n'})
00699     \textcolor{keywordflow}{for} one \textcolor{keywordflow}{in} info[\textcolor{stringliteral}{'ref'}]:
00700         out.write(one+\textcolor{stringliteral}{'\(\backslash\)n'})
00701     out.write(\textcolor{stringliteral}{' -1\(\backslash\)n'})
00702     out.close()
00703     \textcolor{comment}{#}
00704     \textcolor{comment}{# -------------------------------------------------------------------------------------}
00705     \textcolor{comment}{#}
\hypertarget{__chianti__tools_8py_source_l00706}{}\hyperlink{namespacepyneb_1_1utils_1_1__chianti__tools_aa36b22e4e67bccdea3436dcc05e7ba5b}{00706} \textcolor{keyword}{def }\hyperlink{namespacepyneb_1_1utils_1_1__chianti__tools_aa36b22e4e67bccdea3436dcc05e7ba5b}{easplomRead}(ions, extension='.splom'):
00707     \textcolor{stringliteral}{"""read chianti splom files and returns}
00708 \textcolor{stringliteral}{    \{"lvl1":lvl1,"lvl2":lvl2,"deryd":de,"gf":gf,"eryd":eout,"omega":omout\}}
00709 \textcolor{stringliteral}{    currently only works for 5 point spline fit files"""}
00710     \textcolor{comment}{#}
00711     \textcolor{comment}{#}
00712     fname=\hyperlink{namespacepyneb_1_1utils_1_1__chianti__tools_ad4bc7b577fd4c3819ceb00b0a444351b}{ion2filename}(ions)
00713     omname=fname+extension
00714     input=open(omname,\textcolor{stringliteral}{'}\textcolor{stringliteral}{r')}
00715 \textcolor{stringliteral}{    lines=input.readlines()}
00716 \textcolor{stringliteral}{    input.close()}
00717 \textcolor{stringliteral}{    format=\hyperlink{classpyneb_1_1utils_1_1_fortran_format_1_1_fortran_format}{FortranFormat}(}\textcolor{stringliteral}{'5i3,8e10.3'})
00718     data=5
00719     iline=0
00720     lvl1=[]
00721     lvl2=[]
00722     ttype=[]
00723     gf=[]
00724     de=[]
00725     om=[]
00726     z=1
00727     \textcolor{keywordflow}{while} z > 0:
00728         omdat1=\hyperlink{classpyneb_1_1utils_1_1_fortran_format_1_1_fortran_line}{FortranLine}(lines[iline],format)
00729         z=omdat1[0]
00730         \textcolor{keywordflow}{if} z > 0:
00731             l1=omdat1[2]
00732             l2=omdat1[3]
00733             ttype1=omdat1[4]
00734             gf1=omdat1[5]
00735             de1=omdat1[6]
00736             btf1=omdat1[7]
00737             om1=omdat1[8:]
00738             \textcolor{comment}{#}
00739             lvl1.append(l1)
00740             lvl2.append(l2)
00741             ttype.append(ttype1)
00742             gf.append(gf1)
00743             de.append(de1)
00744             om.append(om1)
00745         iline=iline+1
00746     omout=np.asarray(om,\textcolor{stringliteral}{'Float64'})
00747     ref=lines[iline:-1]
00748 \textcolor{comment}{#        omout=np.transpose(omout)}
00749     \textcolor{keywordflow}{if} extension == \textcolor{stringliteral}{'.omdat'}:
00750         Splom=\{\textcolor{stringliteral}{"lvl1"}:lvl1,\textcolor{stringliteral}{"lvl2"}:lvl2,\textcolor{stringliteral}{'ttype'}:ttype,\textcolor{stringliteral}{"gf"}:gf, \textcolor{stringliteral}{"deryd"}:de,\textcolor{stringliteral}{"omega"}:omout, \textcolor{stringliteral}{'ref'}:ref\}
00751         \textcolor{keywordflow}{return} Splom
00752     \textcolor{keywordflow}{elif}  extension == \textcolor{stringliteral}{'.easplom'}:
00753         Easplom=\{\textcolor{stringliteral}{"lvl1"}:lvl1,\textcolor{stringliteral}{"lvl2"}:lvl2,\textcolor{stringliteral}{'ttype'}:ttype,\textcolor{stringliteral}{"gf"}:gf, \textcolor{stringliteral}{"deryd"}:de,\textcolor{stringliteral}{"omega"}:omout, \textcolor{stringliteral}{'ref'}:ref\}
00754         \textcolor{keywordflow}{return} Easplom
00755 
00756     \textcolor{keywordflow}{return}
00757     \textcolor{comment}{#}
00758     \textcolor{comment}{#-----------------------------------------------------------}
00759     \textcolor{comment}{#}
\hypertarget{__chianti__tools_8py_source_l00760}{}\hyperlink{namespacepyneb_1_1utils_1_1__chianti__tools_a391eb09a09769234e5759d5544d0bcaf}{00760} \textcolor{keyword}{def }\hyperlink{namespacepyneb_1_1utils_1_1__chianti__tools_a391eb09a09769234e5759d5544d0bcaf}{splomDescale}(splom, energy):
00761     \textcolor{stringliteral}{"""}
00762 \textcolor{stringliteral}{    Calculates the collision strength}
00763 \textcolor{stringliteral}{    for excitation-autoionization as a function of energy.}
00764 \textcolor{stringliteral}{    energy in eV}
00765 \textcolor{stringliteral}{    """}
00766     \textcolor{comment}{#}
00767     \textcolor{comment}{#}
00768     nenergy=energy.size
00769     nsplom=len(splom[\textcolor{stringliteral}{'deryd'}])
00770     \textcolor{comment}{# for these files, there are 5 spline points}
00771     nspl = 5
00772     \textcolor{keywordflow}{if} nenergy > 1:
00773         omega = np.zeros((nsplom,nenergy),\textcolor{stringliteral}{"float64"})
00774     \textcolor{keywordflow}{else}:
00775         omega = np.zeros(nsplom,\textcolor{stringliteral}{"float64"})
00776     \textcolor{comment}{#}
00777     dx = 1./(float(nspl)-1.)
00778     sxint = dx*np.arange(nspl)
00779     \textcolor{keywordflow}{for} isplom \textcolor{keywordflow}{in} range(0,nsplom):
00780         \textcolor{comment}{#}
00781         sx1 = energy/(splom[\textcolor{stringliteral}{'deryd'}][isplom]*const.ryd2Ev)
00782         good = sx1 >= 1.
00783         \textcolor{comment}{# make sure there are some valid energies above the threshold}
00784         \textcolor{keywordflow}{if} good.sum():
00785             nbad = nenergy - good.sum()
00786             c\_curr = splom[\textcolor{stringliteral}{'c'}][isplom]
00787             \textcolor{comment}{#}
00788             \textcolor{keywordflow}{if} splom[\textcolor{stringliteral}{'ttype'}][isplom] == 1:
00789                 sx = 1. - np.log(c\_curr)/np.log(sx1[good] - 1. + c\_curr)
00790                 y2 = interpolate.splrep(sxint,splom[\textcolor{stringliteral}{'splom'}][:, isplom],s=0)  \textcolor{comment}{#allow smoothing,s=0)}
00791                 som = interpolate.splev(sx,y2,der=0)
00792                 omega[isplom, nbad:] = som*np.log(sx -1. + np.exp(1.))
00793             \textcolor{comment}{#}
00794             \textcolor{keywordflow}{elif} splom[\textcolor{stringliteral}{'ttype'}][isplom] == 2:
00795                 sx =(sx1[good] - 1.)/(sx1[good] -1. + c\_curr)
00796                 y2 = interpolate.splrep(sxint,splom[\textcolor{stringliteral}{'splom'}][:, isplom],s=0)  \textcolor{comment}{#allow smoothing,s=0)}
00797                 som=interpolate.splev(sx,y2,der=0)
00798                 omega[isplom, nbad:] = som
00799             \textcolor{comment}{#}
00800             \textcolor{keywordflow}{elif} splom[\textcolor{stringliteral}{'ttype'}][isplom] == 3:
00801                 sx = (sx1[good] - 1.)/(sx1[good] -1. + c\_curr)
00802                 y2 = interpolate.splrep(sxint,splom[\textcolor{stringliteral}{'splom'}][:, isplom],s=0)  \textcolor{comment}{#allow smoothing,s=0)}
00803                 som = interpolate.splev(sx,y2,der=0)
00804                 omega[isplom, nbad:] = som/sx1[good]**2
00805             \textcolor{comment}{#}
00806             \textcolor{keywordflow}{elif} splom[\textcolor{stringliteral}{'ttype'}][isplom] == 4:
00807                 sx = 1. - np.log(c\_curr)/np.log(sx1[good] -1. + c\_curr)
00808                 y2 = interpolate.splrep(sxint,splom[\textcolor{stringliteral}{'splom'}][:, isplom],s=0)  \textcolor{comment}{#allow smoothing,s=0)}
00809                 som=interpolate.splev(sx,y2,der=0)
00810                 omega[isplom, nbad:] = som*np.log(sx1[good] -1. + c\_curr)
00811             \textcolor{comment}{#}
00812             \textcolor{comment}{#}
00813             \textcolor{comment}{#}
00814             \textcolor{keywordflow}{elif} ttype > 4:
00815                 print((\textcolor{stringliteral}{' splom t\_type ne 1,2,3,4 = %4i %4i %4i'}%(ttype,l1,l2)))
00816         \textcolor{keywordflow}{else}:
00817             \textcolor{comment}{# there are no energies above the threshold}
00818             \textcolor{keywordflow}{pass}
00819     \textcolor{comment}{#}
00820     \textcolor{comment}{#}
00821     omega=np.where(omega > 0.,omega,0.)
00822     \textcolor{comment}{#}
00823     \textcolor{keywordflow}{return} omega
00824     \textcolor{comment}{#}
00825     \textcolor{comment}{# --------------------------------------------------}
00826     \textcolor{comment}{#}
\hypertarget{__chianti__tools_8py_source_l00827}{}\hyperlink{namespacepyneb_1_1utils_1_1__chianti__tools_a8013cd9bdbf2f8ec5a5ac06e25b0e3a4}{00827} \textcolor{keyword}{def }\hyperlink{namespacepyneb_1_1utils_1_1__chianti__tools_a8013cd9bdbf2f8ec5a5ac06e25b0e3a4}{splomRead}(ions, ea=0, filename=None):
00828     \textcolor{stringliteral}{"""}
00829 \textcolor{stringliteral}{    read chianti .splom files and return}
00830 \textcolor{stringliteral}{    \{"lvl1":lvl1,"lvl2":lvl2,"ttype":ttype,"gf":gf,"deryd":de,"c":c,"splom":splomout,"ref":hdr\} not tested}
00831 \textcolor{stringliteral}{    """}
00832     \textcolor{comment}{#}
00833     \textcolor{keywordflow}{if} type(filename) == NoneType:
00834         fname=\hyperlink{namespacepyneb_1_1utils_1_1__chianti__tools_ad4bc7b577fd4c3819ceb00b0a444351b}{ion2filename}(ions)
00835         \textcolor{keywordflow}{if} ea:
00836             splomname=fname+\textcolor{stringliteral}{'.easplom'}
00837         \textcolor{keywordflow}{else}:
00838             splomname=fname+\textcolor{stringliteral}{'.splom'}
00839     \textcolor{keywordflow}{else}:
00840         splomname = filename
00841     input=open(splomname,\textcolor{stringliteral}{'}\textcolor{stringliteral}{r')}
00842 \textcolor{stringliteral}{    }\textcolor{comment}{#  need to read first line and see how many elements}
00843     line1=input.readline()
00844     indices=line1[0:15]
00845     remainder=line1[16:]
00846     nom=remainder.split(\textcolor{stringliteral}{' '})
00847     format=\hyperlink{classpyneb_1_1utils_1_1_fortran_format_1_1_fortran_format}{FortranFormat}(\textcolor{stringliteral}{'5i3,'}+str(len(nom))+\textcolor{stringliteral}{'E10.2'})
00848     \textcolor{comment}{#  go back to the beginning}
00849     input.seek(0)
00850     lines=input.readlines()
00851     data=5
00852     iline=0
00853     lvl1=[]
00854     lvl2=[]
00855     ttype=[]
00856     gf=[]
00857     de=[]
00858     f=[]
00859     splom=[]
00860     ntrans=0
00861     \textcolor{keywordflow}{while} data > 1:
00862         splomdat=\hyperlink{classpyneb_1_1utils_1_1_fortran_format_1_1_fortran_line}{FortranLine}(lines[iline],format)
00863         l1=splomdat[2]
00864         l2=splomdat[3]
00865         tt1=splomdat[4]
00866         gf1=splomdat[5]
00867         de1=splomdat[6]
00868         f1=splomdat[7]
00869         splom1=splomdat[8:]
00870         lvl1.append(int(l1))
00871         lvl2.append(int(l2))
00872         ttype.append(int(tt1))
00873         gf.append(float(gf1))
00874         de.append(float(de1))
00875         f.append(float(f1))
00876         splom.append(splom1)
00877         iline=iline+1
00878         data=len(lines[iline].split(\textcolor{stringliteral}{' '},2))
00879     hdr=lines[iline+1:-1]
00880     de=np.asarray(de,\textcolor{stringliteral}{'Float64'})
00881     splomout=np.asarray(splom,\textcolor{stringliteral}{'Float64'})
00882     splomout=np.transpose(splomout)
00883     input.close()
00884     \textcolor{comment}{# note:  de is in Rydbergs}
00885     splom=\{\textcolor{stringliteral}{"lvl1"}:lvl1,\textcolor{stringliteral}{"lvl2"}:lvl2,\textcolor{stringliteral}{"ttype"}:ttype,\textcolor{stringliteral}{"gf"}:gf,\textcolor{stringliteral}{"deryd"}:de,\textcolor{stringliteral}{"c"}:f
00886         ,\textcolor{stringliteral}{"splom"}:splomout,\textcolor{stringliteral}{"ref"}:hdr\}
00887     \textcolor{keywordflow}{return}  splom
00888     \textcolor{comment}{#}
00889     \textcolor{comment}{# --------------------------------------------------}
00890     \textcolor{comment}{#}
\hypertarget{__chianti__tools_8py_source_l00891}{}\hyperlink{namespacepyneb_1_1utils_1_1__chianti__tools_a6e5e0d51ad5dd162d51353c6d2e49d3b}{00891} \textcolor{keyword}{def }\hyperlink{namespacepyneb_1_1utils_1_1__chianti__tools_a6e5e0d51ad5dd162d51353c6d2e49d3b}{splupsRead}(ions, filename=0, prot=0, ci=0,  diel=0):
00892     \textcolor{stringliteral}{"""}
00893 \textcolor{stringliteral}{    read a chianti splups file and return}
00894 \textcolor{stringliteral}{    \{"lvl1":lvl1,"lvl2":lvl2,"ttype":ttype,"gf":gf,"de":de,"cups":cups,"bsplups":bsplups,"ref":ref\}}
00895 \textcolor{stringliteral}{    if prot >0, then reads the psplups file}
00896 \textcolor{stringliteral}{    if ci > 0, then reads cisplups file}
00897 \textcolor{stringliteral}{    if diel > 0, then reads dielsplups file}
00898 \textcolor{stringliteral}{    """}
00899     \textcolor{comment}{#}
00900     \textcolor{keywordflow}{if} filename:
00901         splupsname = filename
00902     \textcolor{keywordflow}{else}:
00903         fname=\hyperlink{namespacepyneb_1_1utils_1_1__chianti__tools_ad4bc7b577fd4c3819ceb00b0a444351b}{ion2filename}(ions)
00904         \textcolor{keywordflow}{if} prot:
00905             splupsname=fname+\textcolor{stringliteral}{'.psplups'}
00906         \textcolor{keywordflow}{elif} ci:
00907             splupsname=fname+\textcolor{stringliteral}{'.cisplups'}
00908         \textcolor{keywordflow}{elif} diel:
00909             splupsname=fname+\textcolor{stringliteral}{'.dielsplups'}
00910         \textcolor{keywordflow}{else}:
00911             splupsname=fname+\textcolor{stringliteral}{'.splups'}
00912     \textcolor{keywordflow}{if} \textcolor{keywordflow}{not} os.path.exists(splupsname):
00913         \textcolor{keywordflow}{if} prot:
00914             \textcolor{keywordflow}{return} \textcolor{keywordtype}{None}
00915         \textcolor{keywordflow}{elif} ci:
00916             \textcolor{keywordflow}{return} \textcolor{keywordtype}{None}
00917         \textcolor{keywordflow}{else}:
00918             \textcolor{keywordflow}{return} \textcolor{keywordtype}{None}
00919     \textcolor{comment}{# there is splups/psplups data}
00920     \textcolor{keywordflow}{else}:
00921         input=open(splupsname,\textcolor{stringliteral}{'}\textcolor{stringliteral}{r')}
00922 \textcolor{stringliteral}{        s1=input.readlines()}
00923 \textcolor{stringliteral}{        input.close()}
00924 \textcolor{stringliteral}{        nsplups=0}
00925 \textcolor{stringliteral}{        ndata=2}
00926 \textcolor{stringliteral}{        }\textcolor{keywordflow}{while} ndata > 1:
00927             s1a=s1[nsplups][:]
00928             s2=s1a.split()
00929             ndata=len(s2)
00930             nsplups=nsplups+1
00931         nsplups=nsplups-1
00932         lvl1=[0]*nsplups
00933         lvl2=[0]*nsplups
00934         ttype=[0]*nsplups
00935         gf = np.zeros(nsplups, \textcolor{stringliteral}{'float64'})
00936         de= np.zeros(nsplups, \textcolor{stringliteral}{'float64'})
00937         cups= np.zeros(nsplups, \textcolor{stringliteral}{'float64'})
00938         nspl=[0]*nsplups
00939 \textcolor{comment}{#        splups=np.zeros((nsplups,9),'Float64')}
00940         splups = [0.]*nsplups
00941         \textcolor{keywordflow}{if} prot:
00942 \textcolor{comment}{#            splupsFormat1 = FortranFormat('3i3,8e10.3')}
00943             splupsFormat2 = \hyperlink{classpyneb_1_1utils_1_1_fortran_format_1_1_fortran_format}{FortranFormat}(\textcolor{stringliteral}{'3i3,3e10.3'})
00944         \textcolor{keywordflow}{else}:
00945 \textcolor{comment}{#            splupsFormat1='(6x,3i3,8e10.3)'}
00946             splupsFormat2 = \hyperlink{classpyneb_1_1utils_1_1_fortran_format_1_1_fortran_format}{FortranFormat}(\textcolor{stringliteral}{'6x,3i3,3e10.3'})
00947         \textcolor{comment}{#}
00948         \textcolor{keywordflow}{for} i \textcolor{keywordflow}{in} range(0,nsplups):
00949             inpt=\hyperlink{classpyneb_1_1utils_1_1_fortran_format_1_1_fortran_line}{FortranLine}(s1[i],splupsFormat2)
00950             lvl1[i]=inpt[0]
00951             lvl2[i]=inpt[1]
00952             ttype[i]=inpt[2]
00953             gf[i]=inpt[3]
00954             de[i]=inpt[4]
00955             cups[i]=inpt[5]
00956             \textcolor{keywordflow}{if} prot:
00957                 as1 = s1[i][39:].rstrip()
00958             \textcolor{keywordflow}{else}:
00959                 as1 = s1[i][45:].rstrip()
00960             nspl[i] = len(as1)/10
00961             splupsFormat3 = \hyperlink{classpyneb_1_1utils_1_1_fortran_format_1_1_fortran_format}{FortranFormat}(str(nspl[i])+\textcolor{stringliteral}{'E10.2'})
00962 \textcolor{comment}{#            splupsFormat3 = '(' + str(nspl[i]) + 'e10.3' + ')'}
00963             inpt = \hyperlink{classpyneb_1_1utils_1_1_fortran_format_1_1_fortran_line}{FortranLine}(as1, splupsFormat3)
00964             spl1 = np.asarray(inpt[:], \textcolor{stringliteral}{'float64'})
00965             splups[i] = spl1
00966         \textcolor{comment}{#}
00967         ref=[]
00968         \textcolor{keywordflow}{for} i \textcolor{keywordflow}{in} range(nsplups+1,len(s1)-1):
00969             s1a=s1[i][:-1]
00970             ref.append(s1a.strip())
00971         \textcolor{keywordflow}{if} prot:
00972 \textcolor{comment}{#            self.Npsplups=nsplups}
00973 \textcolor{comment}{#            self.Psplups=\{"lvl1":lvl1,"lvl2":lvl2,"ttype":ttype,"gf":gf,"de":de,"cups":cups}
00974 \textcolor{comment}{#                ,"nspl":nspl,"splups":splups,"ref":ref\}}
00975             \textcolor{keywordflow}{return} \{\textcolor{stringliteral}{"lvl1"}:lvl1,\textcolor{stringliteral}{"lvl2"}:lvl2,\textcolor{stringliteral}{"ttype"}:ttype,\textcolor{stringliteral}{"gf"}:gf,\textcolor{stringliteral}{"de"}:de,\textcolor{stringliteral}{"cups"}:cups
00976                 ,\textcolor{stringliteral}{"nspl"}:nspl,\textcolor{stringliteral}{"splups"}:splups,\textcolor{stringliteral}{"ref"}:ref, \textcolor{stringliteral}{'filename'}:splupsname\}
00977         \textcolor{keywordflow}{else}:
00978 \textcolor{comment}{#            self.Splups=\{"lvl1":lvl1,"lvl2":lvl2,"ttype":ttype,"gf":gf,"de":de,"cups":cups}
00979 \textcolor{comment}{#                ,"nspl":nspl,"splups":splups,"ref":ref\}}
00980             \textcolor{keywordflow}{return} \{\textcolor{stringliteral}{"lvl1"}:lvl1,\textcolor{stringliteral}{"lvl2"}:lvl2,\textcolor{stringliteral}{"ttype"}:ttype,\textcolor{stringliteral}{"gf"}:gf,\textcolor{stringliteral}{"de"}:de,\textcolor{stringliteral}{"cups"}:cups
00981                 ,\textcolor{stringliteral}{"nspl"}:nspl,\textcolor{stringliteral}{"splups"}:splups,\textcolor{stringliteral}{"ref"}:ref, \textcolor{stringliteral}{'filename'}:splupsname\}
00982     \textcolor{comment}{#}
00983     \textcolor{comment}{# -------------------------------------------------------------------------------------}
00984     \textcolor{comment}{#}
\hypertarget{__chianti__tools_8py_source_l00985}{}\hyperlink{namespacepyneb_1_1utils_1_1__chianti__tools_a0811a3ad9efb623a53fd1ebfd8c5c73f}{00985} \textcolor{keyword}{def }\hyperlink{namespacepyneb_1_1utils_1_1__chianti__tools_a0811a3ad9efb623a53fd1ebfd8c5c73f}{cireclvlRead}(ions, filename=0, cilvl=0, reclvl=0, rrlvl=0):
00986     \textcolor{stringliteral}{'''}
00987 \textcolor{stringliteral}{    to read Chianti cilvl and reclvl files and return data}
00988 \textcolor{stringliteral}{    must specify type as either cilvl, reclvl or rrlvl}
00989 \textcolor{stringliteral}{    '''}
00990     \textcolor{keywordflow}{if} filename:
00991         fname = filename
00992     \textcolor{keywordflow}{else}:
00993         fname = \hyperlink{namespacepyneb_1_1utils_1_1__chianti__tools_ad4bc7b577fd4c3819ceb00b0a444351b}{ion2filename}(ions)
00994     \textcolor{keywordflow}{if} cilvl:
00995         paramname=fname+\textcolor{stringliteral}{'.cilvl'}
00996     \textcolor{keywordflow}{elif} reclvl:
00997         paramname = fname + \textcolor{stringliteral}{'.reclvl'}
00998     \textcolor{keywordflow}{elif} rrlvl:
00999         paramname = fname + \textcolor{stringliteral}{'.rrlvl'}
01000     \textcolor{keywordflow}{else}:
01001         print(\textcolor{stringliteral}{'either "cilvl", "reclvl" ir "rrlvl" must be specified'})
01002         \textcolor{keywordflow}{return} \{\}
01003     \textcolor{keywordflow}{if} os.path.exists(paramname):
01004         input=open(paramname,\textcolor{stringliteral}{'}\textcolor{stringliteral}{r')}
01005 \textcolor{stringliteral}{        lines = input.readlines()}
01006 \textcolor{stringliteral}{        input.close()}
01007 \textcolor{stringliteral}{    }\textcolor{keywordflow}{else}:
01008         print((\textcolor{stringliteral}{'file does not exist:  '}, paramname))
01009         \textcolor{keywordflow}{return} \{\textcolor{stringliteral}{'error'}:\textcolor{stringliteral}{'file does not exist: '} + paramname\}
01010     \textcolor{comment}{#}
01011     iline = 0
01012     idx = -1
01013     \textcolor{keywordflow}{while} idx < 0:
01014         aline=lines[iline][0:5]
01015         idx=aline.find(\textcolor{stringliteral}{'-1'})
01016         iline += 1
01017     ndata = iline - 1
01018     ntrans = ndata/2
01019     \textcolor{comment}{#}
01020     nref = 0
01021     idx = -1
01022     \textcolor{keywordflow}{while} idx < 0:
01023         aline=lines[iline][0:5]
01024         idx=aline.find(\textcolor{stringliteral}{'-1'})
01025         iline += 1
01026         nref += 1
01027     nref -= 1
01028     \textcolor{comment}{#}
01029     \textcolor{comment}{# need to find the maximum number of temperatures, not all lines are the same}
01030     \textcolor{comment}{#}
01031     ntemp = np.zeros(ntrans, \textcolor{stringliteral}{'int32'})
01032     iline = 0
01033     \textcolor{keywordflow}{for} jline \textcolor{keywordflow}{in} range(0, ndata, 2):
01034         dummy = lines[jline].replace(os.linesep, \textcolor{stringliteral}{''}).split()
01035         ntemp[iline] = len(dummy[4:])
01036         iline += 1
01037     maxNtemp = ntemp.max()
01038 \textcolor{comment}{#   print ' maxNtemp = ', maxNtemp}
01039     temp = np.zeros((ntrans,maxNtemp), \textcolor{stringliteral}{'float64'})
01040     iline = 0
01041     \textcolor{keywordflow}{for} jline \textcolor{keywordflow}{in} range(0, ndata, 2):
01042         recdat = lines[jline].replace(os.linesep, \textcolor{stringliteral}{''}).split()
01043         shortT = np.asarray(recdat[4:], \textcolor{stringliteral}{'float64'})
01044         \textcolor{comment}{# the result of the next statement is to continue to replicate t}
01045         t = np.resize(shortT, maxNtemp)
01046         \textcolor{keywordflow}{if} rrlvl:
01047             temp[iline] = t
01048         \textcolor{keywordflow}{else}:
01049             temp[iline] = 10.**t
01050         iline += 1
01051     \textcolor{comment}{#}
01052     lvl1 = np.zeros(ntrans, \textcolor{stringliteral}{'int64'})
01053     lvl2 = np.zeros(ntrans, \textcolor{stringliteral}{'int64'})
01054     ci = np.zeros((ntrans, maxNtemp), \textcolor{stringliteral}{'float64'})
01055     \textcolor{comment}{#}
01056     idat = 0
01057     \textcolor{keywordflow}{for} jline \textcolor{keywordflow}{in} range(1, ndata, 2):
01058         cidat = lines[jline].replace(os.linesep, \textcolor{stringliteral}{''}).split()
01059         shortCi = np.asarray(cidat[4:], \textcolor{stringliteral}{'float64'})
01060         lvl1[idat] = int(cidat[2])
01061         lvl2[idat] = int(cidat[3])
01062         ci[idat] = np.resize(shortCi, maxNtemp)
01063         idat += 1
01064     \textcolor{keywordflow}{return} \{\textcolor{stringliteral}{'temperature'}:temp, \textcolor{stringliteral}{'ntemp'}:ntemp,\textcolor{stringliteral}{'lvl1'}:lvl1, \textcolor{stringliteral}{'lvl2'}:lvl2, \textcolor{stringliteral}{'rate'}:ci,\textcolor{stringliteral}{'ref'}:lines[ndata+1:-1], \textcolor{stringliteral}{
      'ionS'}:ions\}
01065     \textcolor{comment}{#}
01066     \textcolor{comment}{# ----------------------------------------------------------}
01067     \textcolor{comment}{#}
\hypertarget{__chianti__tools_8py_source_l01068}{}\hyperlink{namespacepyneb_1_1utils_1_1__chianti__tools_a5d55ad976e899b30719ac9b053e34ceb}{01068} \textcolor{keyword}{def }\hyperlink{namespacepyneb_1_1utils_1_1__chianti__tools_a5d55ad976e899b30719ac9b053e34ceb}{dilute}(radius):
01069     \textcolor{stringliteral}{'''}
01070 \textcolor{stringliteral}{    to calculate the dilution factor as a function distance from}
01071 \textcolor{stringliteral}{    the center of a star in units of the stellar radius}
01072 \textcolor{stringliteral}{    a radius of less than 1.0 (incorrect) results in a dilution factor of 0.}
01073 \textcolor{stringliteral}{    '''}
01074     \textcolor{keywordflow}{if} radius >= 1.:
01075         d = 0.5*(1. - np.sqrt(1. - 1./radius**2))
01076     \textcolor{keywordflow}{else}:
01077         d = 0.
01078     \textcolor{keywordflow}{return} d
01079     \textcolor{comment}{#}
01080     \textcolor{comment}{#}
01081     \textcolor{comment}{# ------------------------------------------------------------------------------}
01082     \textcolor{comment}{#}
\hypertarget{__chianti__tools_8py_source_l01083}{}\hyperlink{namespacepyneb_1_1utils_1_1__chianti__tools_a1d4803c6e3b7cbb996b9da3f763ab6ee}{01083} \textcolor{keyword}{def }\hyperlink{namespacepyneb_1_1utils_1_1__chianti__tools_a1d4803c6e3b7cbb996b9da3f763ab6ee}{diCross}(diParams, energy=0, verbose=0):
01084     \textcolor{stringliteral}{'''}
01085 \textcolor{stringliteral}{    Calculate the direct ionization cross section.}
01086 \textcolor{stringliteral}{    diParams obtained by util.diRead with the following keys:}
01087 \textcolor{stringliteral}{    ['info', 'ysplom', 'xsplom', 'btf', 'ev1', 'ref', 'eaev']}
01088 \textcolor{stringliteral}{    Given as a function of the incident electron energy in eV}
01089 \textcolor{stringliteral}{    returns a dictionary - \{'energy':energy, 'cross':cross\}}
01090 \textcolor{stringliteral}{    '''}
01091     iso=diParams[\textcolor{stringliteral}{'info'}][\textcolor{stringliteral}{'iz'}] - diParams[\textcolor{stringliteral}{'info'}][\textcolor{stringliteral}{'ion'}] + 1
01092     energy = np.array(energy, \textcolor{stringliteral}{'float64'})
01093     \textcolor{keywordflow}{if} \textcolor{keywordflow}{not} energy.any():
01094         btenergy=0.1*np.arange(10)
01095         btenergy[0]=0.01
01096         dum=np.ones(len(btenergy))
01097         [energy, dum] = \hyperlink{namespacepyneb_1_1utils_1_1__chianti__tools_a47075ba90f01cbd7a3dbd08115544214}{descale\_bti}(btenergy, dum, 2., diParams[\textcolor{stringliteral}{'ev1'}][0])
01098         energy=np.asarray(energy, \textcolor{stringliteral}{'float64'})
01099     \textcolor{comment}{#}
01100     \textcolor{keywordflow}{if} iso == 1 \textcolor{keywordflow}{and} self.Z >= 6:
01101         \textcolor{comment}{#  hydrogenic sequence}
01102         ryd=27.2113845/2.
01103         u=energy/self.Ip
01104         ev1ryd=self.Ip/ryd
01105         a0=0.5291772108e-8
01106         a\_bohr=const.pi*a0**2   \textcolor{comment}{# area of bohr orbit}
01107         \textcolor{keywordflow}{if} self.Z >= 20:
01108             ff = (140.+(self.Z/20.)**3.2)/141.
01109         \textcolor{keywordflow}{else}:
01110             ff = 1.
01111         qr = util.qrp(self.Z,u)*ff
01112         bb = 1.  \textcolor{comment}{# hydrogenic}
01113         qh = bb*a\_bohr*qr/ev1ryd**2
01114         diCross = \{\textcolor{stringliteral}{'energy'}:energy, \textcolor{stringliteral}{'cross'}:qh\}
01115     \textcolor{keywordflow}{elif} iso == 2 \textcolor{keywordflow}{and} self.Z >= 10:
01116         \textcolor{comment}{#  use}
01117         ryd=27.2113845/2.
01118         u=energy/self.Ip
01119         ev1ryd=self.Ip/ryd
01120         a0=0.5291772108e-8
01121         a\_bohr=const.pi*a0**2   \textcolor{comment}{# area of bohr orbit}
01122         \textcolor{keywordflow}{if} self.Z >= 20:
01123             ff=(140.+(self.Z/20.)**3.2)/141.
01124         \textcolor{keywordflow}{else}:
01125             ff=1.
01126         qr=util.qrp(self.Z,u)*ff
01127         bb=2.  \textcolor{comment}{# helium-like}
01128         qh=bb*a\_bohr*qr/ev1ryd**2
01129         diCross=\{\textcolor{stringliteral}{'energy'}:energy, \textcolor{stringliteral}{'cross'}:qh\}
01130     \textcolor{keywordflow}{else}:
01131         cross=np.zeros(len(energy), \textcolor{stringliteral}{'Float64'})
01132 
01133         \textcolor{keywordflow}{for} ifac \textcolor{keywordflow}{in} range(diParams[\textcolor{stringliteral}{'info'}][\textcolor{stringliteral}{'nfac'}]):
01134             \textcolor{comment}{# prob. better to do this with masked arrays}
01135             goode=energy > diParams[\textcolor{stringliteral}{'ev1'}][ifac]
01136             \textcolor{keywordflow}{if} goode.sum() > 0:
01137                 dum=np.ones(len(energy))
01138                 btenergy, btdum = \hyperlink{namespacepyneb_1_1utils_1_1__chianti__tools_a71b9295157832135424ea7dc0138fcd4}{scale\_bti}(energy[goode],dum[goode], diParams[\textcolor{stringliteral}{'btf'}][ifac], 
      diParams[\textcolor{stringliteral}{'ev1'}][ifac])
01139                 \textcolor{comment}{# these interpolations were made with the scipy routine used here}
01140                 y2=interpolate.splrep(diParams[\textcolor{stringliteral}{'xsplom'}][ifac], diParams[\textcolor{stringliteral}{'ysplom'}][ifac], s=0)
01141                 btcross=interpolate.splev(btenergy, y2, der=0)
01142                 energy1, cross1 = \hyperlink{namespacepyneb_1_1utils_1_1__chianti__tools_a47075ba90f01cbd7a3dbd08115544214}{descale\_bti}(btenergy, btcross, diParams[\textcolor{stringliteral}{'btf'}][ifac], diParams
      [\textcolor{stringliteral}{'ev1'}][ifac] )
01143                 offset=len(energy)-goode.sum()
01144                 \textcolor{keywordflow}{if} verbose:
01145                     pl.plot(diParams[\textcolor{stringliteral}{'xsplom'}][ifac], diParams[\textcolor{stringliteral}{'ysplom'}][ifac])
01146                     pl.plot(btenergy, btcross)
01147                 \textcolor{keywordflow}{if} offset > 0:
01148                     seq=[np.zeros(offset, \textcolor{stringliteral}{'Float64'}), cross1]
01149                     cross1=np.hstack(seq)
01150                 cross+=cross1*1.e-14
01151         \textcolor{keywordflow}{return} \{\textcolor{stringliteral}{'energy'}:energy, \textcolor{stringliteral}{'cross'}:cross\}
01152     \textcolor{comment}{#}
01153     \textcolor{comment}{#-----------------------------------------------------------}
01154     \textcolor{comment}{#}
\hypertarget{__chianti__tools_8py_source_l01155}{}\hyperlink{namespacepyneb_1_1utils_1_1__chianti__tools_a28552b5d18a1b604ed4512c435e648bc}{01155} \textcolor{keyword}{def }\hyperlink{namespacepyneb_1_1utils_1_1__chianti__tools_a28552b5d18a1b604ed4512c435e648bc}{diRead}(ions, filename=0):
01156     \textcolor{stringliteral}{"""}
01157 \textcolor{stringliteral}{    read chianti direct ionization .params files and return}
01158 \textcolor{stringliteral}{        \{"info":info,"btf":btf,"ev1":ev1,"xsplom":xsplom,"ysplom":ysplom,"ref":hdr\}}
01159 \textcolor{stringliteral}{        info=\{"iz":iz,"ion":ion,"nspl":nspl,"neaev":neaev\}}
01160 \textcolor{stringliteral}{    """}
01161     \textcolor{comment}{#}
01162     \textcolor{keywordflow}{if} filename:
01163         paramname = filename
01164     \textcolor{keywordflow}{else}:
01165         zion=\hyperlink{namespacepyneb_1_1utils_1_1__chianti__tools_a92cf299ad3407ee8923739e2761ab13f}{convertName}(ions)
01166         \textcolor{keywordflow}{if} zion[\textcolor{stringliteral}{'Z'}] < zion[\textcolor{stringliteral}{'Ion'}]:
01167             print(\textcolor{stringliteral}{' this is a bare nucleus that has no ionization rate'})
01168             \textcolor{keywordflow}{return}
01169         \textcolor{comment}{#}
01170         fname=\hyperlink{namespacepyneb_1_1utils_1_1__chianti__tools_ad4bc7b577fd4c3819ceb00b0a444351b}{ion2filename}(ions)
01171         paramname=fname+\textcolor{stringliteral}{'.diparams'}
01172     \textcolor{comment}{#}
01173     input=open(paramname,\textcolor{stringliteral}{'}\textcolor{stringliteral}{r')}
01174 \textcolor{stringliteral}{    }\textcolor{comment}{#  need to read first line and see how many elements}
01175     line1=input.readline()
01176     indices=line1.split()
01177     iz=int(indices[0])
01178     ion=int(indices[1])
01179     nspl=indices[2]
01180     nfac=int(indices[3])
01181     neaev=int(indices[4])
01182     nspl=int(nspl)
01183     format=\hyperlink{classpyneb_1_1utils_1_1_fortran_format_1_1_fortran_format}{FortranFormat}(str(nspl+1)+\textcolor{stringliteral}{'E10.2'})
01184     \textcolor{comment}{#}
01185     ev1=np.zeros(nfac,\textcolor{stringliteral}{'Float64'})
01186     btf=np.zeros(nfac,\textcolor{stringliteral}{'Float64'})
01187     xsplom=np.zeros([nfac, nspl],\textcolor{stringliteral}{'Float64'})
01188     ysplom=np.zeros([nfac, nspl],\textcolor{stringliteral}{'Float64'})
01189     \textcolor{comment}{#}
01190     \textcolor{keywordflow}{for} ifac \textcolor{keywordflow}{in} range(nfac):
01191         line=input.readline()
01192         paramdat=\hyperlink{classpyneb_1_1utils_1_1_fortran_format_1_1_fortran_line}{FortranLine}(line,format)
01193         btf[ifac]=paramdat[0]
01194         xsplom[ifac]=paramdat[1:]
01195         line=input.readline()
01196         paramdat=\hyperlink{classpyneb_1_1utils_1_1_fortran_format_1_1_fortran_line}{FortranLine}(line,format)
01197         ev1[ifac]=paramdat[0]
01198         ysplom[ifac]=paramdat[1:]
01199     \textcolor{keywordflow}{if} neaev:
01200         line=input.readline()
01201         eacoef=line.split()
01202 \textcolor{comment}{#            print ' eaev = ', type(eacoef), eacoef}
01203         eaev=[float(avalue) \textcolor{keywordflow}{for} avalue \textcolor{keywordflow}{in} eacoef]
01204 \textcolor{comment}{#            print ' eaev = ', type(eaev), eaev}
01205 \textcolor{comment}{#            print ' eaev = ', type(eaev), eaev}
01206 \textcolor{comment}{#            if len(eaev) == 1:}
01207 \textcolor{comment}{#                eaev=float(eaev[0])}
01208 \textcolor{comment}{#                eaev=np.asarray(eaev, 'float32')}
01209 \textcolor{comment}{#            else:}
01210 \textcolor{comment}{#                eaev=np.asarray(eaev, 'float32')}
01211     \textcolor{keywordflow}{else}:
01212         eaev=0.
01213     hdr=input.readlines()
01214     input.close()
01215     info=\{\textcolor{stringliteral}{"iz"}:iz,\textcolor{stringliteral}{"ion"}:ion,\textcolor{stringliteral}{"nspl"}:nspl,\textcolor{stringliteral}{"neaev"}:neaev, \textcolor{stringliteral}{'nfac'}:nfac\}
01216     \textcolor{keywordflow}{if} neaev:
01217         info[\textcolor{stringliteral}{'eaev'}] = eaev
01218     DiParams=\{\textcolor{stringliteral}{"info"}:info,\textcolor{stringliteral}{"btf"}:btf,\textcolor{stringliteral}{"ev1"}:ev1,\textcolor{stringliteral}{"xsplom"}:xsplom,\textcolor{stringliteral}{"ysplom"}:ysplom, \textcolor{stringliteral}{'eaev'}:eaev,\textcolor{stringliteral}{"ref"}:hdr\}
01219     \textcolor{keywordflow}{return} DiParams
01220     \textcolor{comment}{#}
01221     \textcolor{comment}{# -------------------------------------------------------------------------------------}
01222     \textcolor{comment}{#}
\hypertarget{__chianti__tools_8py_source_l01223}{}\hyperlink{namespacepyneb_1_1utils_1_1__chianti__tools_a5c2be70e50273734d319a433bf39ce5b}{01223} \textcolor{keyword}{def }\hyperlink{namespacepyneb_1_1utils_1_1__chianti__tools_a5c2be70e50273734d319a433bf39ce5b}{eaCross}(diparams, easplom, elvlc, energy=None, verbose=False):
01224     \textcolor{stringliteral}{'''}
01225 \textcolor{stringliteral}{    Provide the excitation-autoionization cross section.}
01226 \textcolor{stringliteral}{}
01227 \textcolor{stringliteral}{    Energy is given in eV.}
01228 \textcolor{stringliteral}{    '''}
01229     energy = np.asarray(energy, \textcolor{stringliteral}{'float64'})
01230     \textcolor{keywordflow}{if} \textcolor{keywordflow}{not} energy.any():
01231         btenergy=0.1*np.arange(10)
01232         btenergy[0]=0.01
01233         dum=np.ones(len(btenergy))
01234         [energy, dum] = \hyperlink{namespacepyneb_1_1utils_1_1__chianti__tools_a47075ba90f01cbd7a3dbd08115544214}{descale\_bti}(btenergy, dum, 2., min(easplom[\textcolor{stringliteral}{'deryd'}]))
01235         energy=np.asarray(energy, \textcolor{stringliteral}{'float64'})
01236     \textcolor{comment}{#}
01237     omega = \hyperlink{namespacepyneb_1_1utils_1_1__chianti__tools_a391eb09a09769234e5759d5544d0bcaf}{splomDescale}(easplom, energy)
01238     \textcolor{comment}{#}
01239     \textcolor{comment}{#  need to replicate neaev}
01240 
01241     \textcolor{keywordflow}{if} diparams[\textcolor{stringliteral}{'info'}][\textcolor{stringliteral}{'neaev'}] > 0:
01242         f1 = np.ones(omega.shape[0])
01243     \textcolor{keywordflow}{else}:
01244         f1 = diparams[\textcolor{stringliteral}{'info'}][\textcolor{stringliteral}{'eaev'}]
01245 
01246     totalCross = np.zeros\_like(energy)
01247     ntrans = omega.shape[0]
01248     \textcolor{keywordflow}{for} itrans \textcolor{keywordflow}{in} range(ntrans):
01249         lvl1 = easplom[\textcolor{stringliteral}{'lvl1'}][itrans]
01250         mult = 2.*elvlc[\textcolor{stringliteral}{'j'}][lvl1 - 1] + 1.
01251         cross = f1[itrans]*const.bohrCross*omega[itrans]/(mult.energy/const.ryd2Ev)
01252         totalCross += cross
01253     \textcolor{keywordflow}{return} \{\textcolor{stringliteral}{'energy'}:energy, \textcolor{stringliteral}{'cross'}:totalCross\}
01254     \textcolor{comment}{#}
01255     \textcolor{comment}{# -------------------------------------------------------------------------------------}
01256     \textcolor{comment}{#}
\hypertarget{__chianti__tools_8py_source_l01257}{}\hyperlink{namespacepyneb_1_1utils_1_1__chianti__tools_ab4f78a2584a1ddf189af644e961f0f83}{01257} \textcolor{keyword}{def }\hyperlink{namespacepyneb_1_1utils_1_1__chianti__tools_ab4f78a2584a1ddf189af644e961f0f83}{eaRead}(ions, filename=0):
01258     \textcolor{stringliteral}{'''}
01259 \textcolor{stringliteral}{    read a chianti excitation-autoionization file and return the EA ionization rate data}
01260 \textcolor{stringliteral}{    derived from splupsRead}
01261 \textcolor{stringliteral}{    \{"lvl1":lvl1,"lvl2":lvl2,"ttype":ttype,"gf":gf,"de":de,"cups":cups,"bsplups":bsplups,"ref":ref\}}
01262 \textcolor{stringliteral}{    '''}
01263     \textcolor{keywordflow}{if} filename:
01264         splupsname = filename
01265     \textcolor{keywordflow}{else}:
01266         zion=\hyperlink{namespacepyneb_1_1utils_1_1__chianti__tools_a92cf299ad3407ee8923739e2761ab13f}{convertName}(ions)
01267         \textcolor{keywordflow}{if} zion[\textcolor{stringliteral}{'Z'}] < zion[\textcolor{stringliteral}{'Ion'}]:
01268             print(\textcolor{stringliteral}{' this is a bare nucleus that has no ionization rate'})
01269             \textcolor{keywordflow}{return}
01270         \textcolor{comment}{#}
01271         fname=\hyperlink{namespacepyneb_1_1utils_1_1__chianti__tools_ad4bc7b577fd4c3819ceb00b0a444351b}{ion2filename}(ions)
01272         splupsname=fname+\textcolor{stringliteral}{'.easplups'}
01273     \textcolor{keywordflow}{if} \textcolor{keywordflow}{not} os.path.exists(splupsname):
01274         print((\textcolor{stringliteral}{' could not find file:  '}, splupsname))
\hypertarget{__chianti__tools_8py_source_l01275}{}\hyperlink{namespacepyneb_1_1utils_1_1__chianti__tools_ae4b82a587c953b2207888a96d3364677}{01275}         self.Splups=\{\textcolor{stringliteral}{"lvl1"}:-1\}
01276         \textcolor{keywordflow}{return} \{\textcolor{stringliteral}{"lvl1"}:-1\}
01277     \textcolor{comment}{# there is splups/psplups data}
01278     \textcolor{keywordflow}{else}:
01279         input=open(splupsname,\textcolor{stringliteral}{'}\textcolor{stringliteral}{r')}
01280 \textcolor{stringliteral}{        s1=input.readlines()}
01281 \textcolor{stringliteral}{        dum=input.close()}
01282 \textcolor{stringliteral}{        nsplups=0}
01283 \textcolor{stringliteral}{        ndata=2}
01284 \textcolor{stringliteral}{        }\textcolor{keywordflow}{while} ndata > 1:
01285             s1a=s1[nsplups][:]
01286             s2=s1a.split()
01287             ndata=len(s2)
01288             nsplups=nsplups+1
01289         nsplups=nsplups-1
01290         lvl1=[0]*nsplups
01291         lvl2=[0]*nsplups
01292         ttype=[0]*nsplups
01293         gf=[0.]*nsplups
01294         de=[0.]*nsplups
01295         cups=[0.]*nsplups
01296         nspl=[0]*nsplups
01297         splups=np.zeros((nsplups,9),\textcolor{stringliteral}{'Float64'})
01298         splupsFormat1=\textcolor{stringliteral}{'(6x,3i3,8e10.0)'}
01299         splupsFormat2=\textcolor{stringliteral}{'(6x,3i3,12e10.0)'}
01300         \textcolor{comment}{#}
01301         \textcolor{keywordflow}{for} i \textcolor{keywordflow}{in} range(0,nsplups):
01302             \textcolor{keywordflow}{try}:
01303                 inpt=\hyperlink{classpyneb_1_1utils_1_1_fortran_format_1_1_fortran_line}{FortranLine}(s1[i],splupsFormat1)
01304             \textcolor{keywordflow}{except}:
01305                 inpt=\hyperlink{classpyneb_1_1utils_1_1_fortran_format_1_1_fortran_line}{FortranLine}(s1[i],splupsFormat2)
01306             lvl1[i]=inpt[0]
01307             lvl2[i]=inpt[1]
01308             ttype[i]=inpt[2]
01309             gf[i]=inpt[3]
01310             de[i]=inpt[4]
01311             cups[i]=inpt[5]
01312             \textcolor{keywordflow}{if} len(inpt)  > 13:
01313                 nspl[i]=9
01314                 splups[i].put(range(9),inpt[6:])
01315             \textcolor{keywordflow}{else}:
01316                 nspl[i]=5
01317                 splups[i].put(range(5),inpt[6:])
01318         \textcolor{comment}{#}
01319         ref=[]
01320         \textcolor{keywordflow}{for} i \textcolor{keywordflow}{in} range(nsplups+1,len(s1)-1):
01321             s1a=s1[i][:-1]
01322             ref.append(s1a.strip())
01323 \textcolor{comment}{#        self.EaParams=\{"lvl1":lvl1,"lvl2":lvl2,"ttype":ttype,"gf":gf,"de":de,"cups":cups}
01324 \textcolor{comment}{#                ,"nspl":nspl,"splups":splups,"ref":ref\}}
01325         \textcolor{keywordflow}{return} \{\textcolor{stringliteral}{"lvl1"}:lvl1,\textcolor{stringliteral}{"lvl2"}:lvl2,\textcolor{stringliteral}{"ttype"}:ttype,\textcolor{stringliteral}{"gf"}:gf,\textcolor{stringliteral}{"de"}:de,\textcolor{stringliteral}{"cups"}:cups
01326                 ,\textcolor{stringliteral}{"nspl"}:nspl,\textcolor{stringliteral}{"splups"}:splups,\textcolor{stringliteral}{"ref"}:ref\}
01327     \textcolor{comment}{#}
01328     \textcolor{comment}{# -------------------------------------------------------------------------------------}
01329     \textcolor{comment}{#}
\hypertarget{__chianti__tools_8py_source_l01330}{}\hyperlink{namespacepyneb_1_1utils_1_1__chianti__tools_a71865c846f3d3b1c9b1371c6779c8f3e}{01330} \textcolor{keyword}{def }\hyperlink{namespacepyneb_1_1utils_1_1__chianti__tools_a71865c846f3d3b1c9b1371c6779c8f3e}{rrRead}(ions):
01331     \textcolor{stringliteral}{"""read chianti radiative recombination .rrparams files and return}
01332 \textcolor{stringliteral}{        \{'rrtype','params','ref'\}"""}
01333     \textcolor{comment}{#}
01334     \textcolor{comment}{#}
01335     fname=\hyperlink{namespacepyneb_1_1utils_1_1__chianti__tools_ad4bc7b577fd4c3819ceb00b0a444351b}{ion2filename}(ions)
01336     paramname=fname+\textcolor{stringliteral}{'.rrparams'}
01337     \textcolor{keywordflow}{if} os.path.isfile(paramname):
01338         input=open(paramname,\textcolor{stringliteral}{'}\textcolor{stringliteral}{r')}
01339 \textcolor{stringliteral}{        }\textcolor{comment}{#  need to read first line and see how many elements}
01340         lines=input.readlines()
01341         input.close()
01342         rrtype=int(lines[0])
01343         ref=lines[3:-2]
01344         \textcolor{comment}{#}
01345         \textcolor{keywordflow}{if} rrtype == 1:
01346             \textcolor{comment}{# a Badnell type}
01347             fmt=\hyperlink{classpyneb_1_1utils_1_1_fortran_format_1_1_fortran_format}{FortranFormat}(\textcolor{stringliteral}{'3i5,e12.4,f10.5,2e12.4'})
01348             params=\hyperlink{classpyneb_1_1utils_1_1_fortran_format_1_1_fortran_line}{FortranLine}(lines[1],fmt)
01349             RrParams=\{\textcolor{stringliteral}{'rrtype'}:rrtype, \textcolor{stringliteral}{'params'}:params, \textcolor{stringliteral}{'ref'}:ref\}
01350         \textcolor{keywordflow}{elif} rrtype == 2:
01351             \textcolor{comment}{# a Badnell type}
01352             fmt=\hyperlink{classpyneb_1_1utils_1_1_fortran_format_1_1_fortran_format}{FortranFormat}(\textcolor{stringliteral}{'3i5,e12.4,f10.5,2e11.4,f10.5,e12.4'})
01353             params=\hyperlink{classpyneb_1_1utils_1_1_fortran_format_1_1_fortran_line}{FortranLine}(lines[1],fmt)
01354             RrParams=\{\textcolor{stringliteral}{'rrtype'}:rrtype, \textcolor{stringliteral}{'params'}:params, \textcolor{stringliteral}{'ref'}:ref\}
01355         \textcolor{keywordflow}{elif} rrtype == 3:
01356             \textcolor{comment}{# a Shull type}
01357             fmt=\hyperlink{classpyneb_1_1utils_1_1_fortran_format_1_1_fortran_format}{FortranFormat}(\textcolor{stringliteral}{'2i5,2e12.4'})
01358             params=\hyperlink{classpyneb_1_1utils_1_1_fortran_format_1_1_fortran_line}{FortranLine}(lines[1],fmt)
01359             RrParams=\{\textcolor{stringliteral}{'rrtype'}:rrtype, \textcolor{stringliteral}{'params'}:params, \textcolor{stringliteral}{'ref'}:ref\}
01360         \textcolor{keywordflow}{else}:
01361             RrParams=\textcolor{keywordtype}{None}
01362             print((\textcolor{stringliteral}{' for ion %5s unknown RR type = %5i'} %(ions, rrtype)))
01363         \textcolor{keywordflow}{return} RrParams
01364     \textcolor{keywordflow}{else}:
01365         \textcolor{keywordflow}{return} \{\textcolor{stringliteral}{'rrtype'}:-1\}
01366 
01367     \textcolor{comment}{#}
01368     \textcolor{comment}{# -------------------------------------------------------------------------------------}
01369     \textcolor{comment}{#}
\hypertarget{__chianti__tools_8py_source_l01370}{}\hyperlink{namespacepyneb_1_1utils_1_1__chianti__tools_aa29da4afe4adccd8de62c0dd97799dcd}{01370} \textcolor{keyword}{def }\hyperlink{namespacepyneb_1_1utils_1_1__chianti__tools_aa29da4afe4adccd8de62c0dd97799dcd}{drRead}(ions):
01371     \textcolor{stringliteral}{"""read chianti dielectronic recombination .drparams files and return}
01372 \textcolor{stringliteral}{        \{'rrtype','params','ref'\}"""}
01373     \textcolor{comment}{#}
01374     \textcolor{comment}{#}
01375     fname=\hyperlink{namespacepyneb_1_1utils_1_1__chianti__tools_ad4bc7b577fd4c3819ceb00b0a444351b}{ion2filename}(ions)
01376     paramname=fname+\textcolor{stringliteral}{'.drparams'}
01377     \textcolor{keywordflow}{if} os.path.isfile(paramname):
01378         input=open(paramname,\textcolor{stringliteral}{'}\textcolor{stringliteral}{r')}
01379 \textcolor{stringliteral}{        }\textcolor{comment}{#  need to read first line and see how many elements}
01380         lines=input.readlines()
01381         input.close()
01382         drtype=int(lines[0])
01383         ref=lines[4:-1]
01384         \textcolor{comment}{#}
01385         \textcolor{keywordflow}{if} drtype == 1:
01386             \textcolor{comment}{# a Badnell type}
01387             fmt=\hyperlink{classpyneb_1_1utils_1_1_fortran_format_1_1_fortran_format}{FortranFormat}(\textcolor{stringliteral}{'2i5,8e12.4'})
01388             eparams=np.asarray(\hyperlink{classpyneb_1_1utils_1_1_fortran_format_1_1_fortran_line}{FortranLine}(lines[1],fmt)[2:], \textcolor{stringliteral}{'float64'})
01389             cparams=np.asarray(\hyperlink{classpyneb_1_1utils_1_1_fortran_format_1_1_fortran_line}{FortranLine}(lines[2],fmt)[2:], \textcolor{stringliteral}{'float64'})
01390             DrParams=\{\textcolor{stringliteral}{'drtype'}:drtype, \textcolor{stringliteral}{'eparams'}:eparams,\textcolor{stringliteral}{'cparams'}:cparams,  \textcolor{stringliteral}{'ref'}:ref\}
01391         \textcolor{keywordflow}{elif} drtype == 2:
01392             \textcolor{comment}{# shull type}
01393             fmt=\hyperlink{classpyneb_1_1utils_1_1_fortran_format_1_1_fortran_format}{FortranFormat}(\textcolor{stringliteral}{'2i5,4e12.4'})
01394             params=np.asarray(\hyperlink{classpyneb_1_1utils_1_1_fortran_format_1_1_fortran_line}{FortranLine}(lines[1],fmt)[2:], \textcolor{stringliteral}{'float64'})
01395             DrParams=\{\textcolor{stringliteral}{'drtype'}:drtype, \textcolor{stringliteral}{'params'}:params, \textcolor{stringliteral}{'ref'}:ref\}
01396         \textcolor{keywordflow}{else}:
01397             DrParams = \textcolor{keywordtype}{None}
01398             print((\textcolor{stringliteral}{' for ion %5s unknown DR type = %5i'} %(ions, drtype)))
01399     \textcolor{keywordflow}{else}:
01400         DrParams=\textcolor{keywordtype}{None}
01401     \textcolor{keywordflow}{return} DrParams
01402     \textcolor{comment}{#}
01403     \textcolor{comment}{# -------------------------------------------------------------------------------------}
01404     \textcolor{comment}{#}
\hypertarget{__chianti__tools_8py_source_l01405}{}\hyperlink{namespacepyneb_1_1utils_1_1__chianti__tools_a8b6257cfe133ac906966b20c8721f82a}{01405} \textcolor{keyword}{def }\hyperlink{namespacepyneb_1_1utils_1_1__chianti__tools_a8b6257cfe133ac906966b20c8721f82a}{ioneqRead}(ioneqname='', verbose=0):
01406     \textcolor{stringliteral}{"""}
01407 \textcolor{stringliteral}{    reads an ioneq file and stores temperatures and ionization}
01408 \textcolor{stringliteral}{    equilibrium values in self.IoneqTemperature and self.Ioneq and returns}
01409 \textcolor{stringliteral}{    a dictionary containing these value and the reference to the literature}
01410 \textcolor{stringliteral}{    """}
01411     \textcolor{keywordflow}{pass}
01412     \textcolor{comment}{#}
01413     \textcolor{comment}{# -------------------------------------------------------------------------------------}
01414     \textcolor{comment}{#}
\hypertarget{__chianti__tools_8py_source_l01415}{}\hyperlink{namespacepyneb_1_1utils_1_1__chianti__tools_af524b50fae5a347eff4976b7ce895ab1}{01415} \textcolor{keyword}{def }\hyperlink{namespacepyneb_1_1utils_1_1__chianti__tools_af524b50fae5a347eff4976b7ce895ab1}{gffRead}():
01416     \textcolor{stringliteral}{'''to read the free-free gaunt factors of Sutherland, 1998, MNRAS, 300, 321.}
01417 \textcolor{stringliteral}{    this function reads the file and reverses the values of g2 and u'''}
01418     xuvtop = os.environ[\textcolor{stringliteral}{'XUVTOP'}]
01419     fileName = os.path.join(xuvtop, \textcolor{stringliteral}{'continuum'},\textcolor{stringliteral}{'gffgu.dat'} )
01420     input = open(fileName)
01421     lines = input.readlines()
01422     input.close()
01423     \textcolor{comment}{#}
01424     \textcolor{comment}{#  the 1d stuff below is to make it easy to use interp2d}
01425     ngamma=41
01426     nu=81
01427     nvalues = ngamma*nu
01428     g2 = np.zeros(ngamma, \textcolor{stringliteral}{'float64'})
01429     g21d = np.zeros(nvalues, \textcolor{stringliteral}{'float64'})
01430     u = np.zeros(nu, \textcolor{stringliteral}{'float64'})
01431     u1d = np.zeros(nvalues, \textcolor{stringliteral}{'float64'})
01432     gff = np.zeros((ngamma, nu), \textcolor{stringliteral}{'float64'})
01433     gff1d = np.zeros(nvalues, \textcolor{stringliteral}{'float64'})
01434     \textcolor{comment}{#}
01435     iline = 5
01436     ivalue = 0
01437     \textcolor{keywordflow}{for} ig2 \textcolor{keywordflow}{in} range(ngamma):
01438         \textcolor{keywordflow}{for} iu \textcolor{keywordflow}{in} range(nu):
01439             values = lines[iline].split()
01440             u[iu] = float(values[1])
01441             u1d[ivalue] = float(values[1])
01442             g2[ig2] = float(values[0])
01443             g21d[ivalue] = float(values[0])
01444             gff[ig2, iu] = float(values[2])
01445             gff1d[ivalue] = float(values[2])
01446             iline+=1
01447             ivalue += 1
01448     \textcolor{comment}{#}
01449     \textcolor{keywordflow}{return} \{\textcolor{stringliteral}{'g2'}:g2, \textcolor{stringliteral}{'g21d'}:g21d,  \textcolor{stringliteral}{'}\textcolor{stringliteral}{u':u, '}u1d':u1d,  'gff':gff,  'gff1d':gff1d\}
01450     \textcolor{comment}{#}
01451     \textcolor{comment}{# ----------------------------------------------------------------------------------------}
01452     \textcolor{comment}{#}
\hypertarget{__chianti__tools_8py_source_l01453}{}\hyperlink{namespacepyneb_1_1utils_1_1__chianti__tools_a27deea720e4534912f04b54fe5c17b93}{01453} \textcolor{keyword}{def }\hyperlink{namespacepyneb_1_1utils_1_1__chianti__tools_a27deea720e4534912f04b54fe5c17b93}{gffintRead}():
01454     \textcolor{stringliteral}{'''to read the integrated free-free gaunt factors of Sutherland, 1998, MNRAS, 300, 321.'''}
01455     xuvtop = os.environ[\textcolor{stringliteral}{'XUVTOP'}]
01456     fileName = os.path.join(xuvtop, \textcolor{stringliteral}{'continuum'},\textcolor{stringliteral}{'gffint.dat'} )
01457     input = open(fileName)
01458     lines = input.readlines()
01459     input.close()
01460     \textcolor{comment}{#}
01461     ngamma=41
01462     g2 = np.zeros(ngamma, \textcolor{stringliteral}{'float64'})
01463     gffint = np.zeros(ngamma, \textcolor{stringliteral}{'float64'})
01464     s1 = np.zeros(ngamma, \textcolor{stringliteral}{'float64'})
01465     s2 = np.zeros(ngamma, \textcolor{stringliteral}{'float64'})
01466     s3 = np.zeros(ngamma, \textcolor{stringliteral}{'float64'})
01467     \textcolor{comment}{#}
01468     ivalue = 0
01469     start = 4
01470     \textcolor{keywordflow}{for} iline \textcolor{keywordflow}{in} range(start,start+ngamma):
01471         values = lines[iline].split()
01472         g2[ivalue] = float(values[0])
01473         gffint[ivalue] = float(values[1])
01474         s1[ivalue] = float(values[2])
01475         s2[ivalue] = float(values[3])
01476         s3[ivalue] = float(values[4])
01477         ivalue += 1
01478     \textcolor{comment}{#}
01479     \textcolor{keywordflow}{return} \{\textcolor{stringliteral}{'g2'}:g2, \textcolor{stringliteral}{'gffint'}:gffint, \textcolor{stringliteral}{'s1'}:s1, \textcolor{stringliteral}{'s2'}:s2, \textcolor{stringliteral}{'s3'}:s3\}
01480     \textcolor{comment}{#}
01481     \textcolor{comment}{# ----------------------------------------------------------------------------------------}
01482     \textcolor{comment}{#}
\hypertarget{__chianti__tools_8py_source_l01483}{}\hyperlink{namespacepyneb_1_1utils_1_1__chianti__tools_a3dc5585b3508d8baf58ba62d0f728f16}{01483} \textcolor{keyword}{def }\hyperlink{namespacepyneb_1_1utils_1_1__chianti__tools_a3dc5585b3508d8baf58ba62d0f728f16}{itohRead}():
01484     \textcolor{stringliteral}{'''to read in the free-free gaunt factors of Itoh et al. (ApJS 128, 125, 2000)'''}
01485     xuvtop = os.environ[\textcolor{stringliteral}{'XUVTOP'}]
01486     itohName = os.path.join(xuvtop, \textcolor{stringliteral}{'continuum'}, \textcolor{stringliteral}{'itoh.dat'})
01487     input = open(itohName)
01488     lines = input.readlines()
01489     input.close()
01490     gff = np.zeros((30, 121), \textcolor{stringliteral}{'float64'})
01491     \textcolor{keywordflow}{for} iline \textcolor{keywordflow}{in} range(30):
01492         gff[iline]= np.asarray(lines[iline].split(), \textcolor{stringliteral}{'float64'})
01493     \textcolor{keywordflow}{return} \{\textcolor{stringliteral}{'itohCoef'}:gff\}
01494     \textcolor{comment}{#}
01495     \textcolor{comment}{#}
01496     \textcolor{comment}{# ----------------------------------------------------------------------------------------}
01497     \textcolor{comment}{#}
\hypertarget{__chianti__tools_8py_source_l01498}{}\hyperlink{namespacepyneb_1_1utils_1_1__chianti__tools_aa4cdc5fd04cf99d7600e480f0dc95ae4}{01498} \textcolor{keyword}{def }\hyperlink{namespacepyneb_1_1utils_1_1__chianti__tools_aa4cdc5fd04cf99d7600e480f0dc95ae4}{klgfbRead}():
01499     \textcolor{stringliteral}{'''to read CHIANTI files file containing the free-bound gaunt factors for n=1-6 from Karzas and Latter,
       1961, ApJSS, 6, 167}
01500 \textcolor{stringliteral}{    returns \{pe, klgfb\}, the photon energy and the free-bound gaunt factors'''}
01501     xuvtop = os.environ[\textcolor{stringliteral}{'XUVTOP'}]
01502     fname = os.path.join(xuvtop, \textcolor{stringliteral}{'continuum'}, \textcolor{stringliteral}{'klgfb.dat'})
01503     input = open(fname)
01504     lines = input.readlines()
01505     input.close()
01506     \textcolor{comment}{#}
01507     ngfb = int(lines[0].split()[0])
01508     nume = int(lines[0].split()[1])
01509 
01510     gfb = np.zeros((ngfb, ngfb, nume), \textcolor{stringliteral}{'float64'})
01511     nlines = len(lines)
01512 \textcolor{comment}{#        print 'nlines, nume, ngfb = ', nlines,  nume, ngfb}
01513     pe = np.asarray(lines[1].split(), \textcolor{stringliteral}{'float64'})
01514     \textcolor{keywordflow}{for} iline \textcolor{keywordflow}{in} range(2, nlines):
01515         data = lines[iline].split()
01516         n = int(data[0])
01517         l = int(data[1])
01518         gfb[n-1, l] = np.array(data[2:], \textcolor{stringliteral}{'float64'})
01519     \textcolor{keywordflow}{return} \{\textcolor{stringliteral}{'pe'}:pe, \textcolor{stringliteral}{'klgfb'}:gfb\}
01520 \textcolor{comment}{#}
01521 \textcolor{comment}{#  ---------------------------------------------------------}
01522 \textcolor{comment}{#}
\hypertarget{__chianti__tools_8py_source_l01523}{}\hyperlink{namespacepyneb_1_1utils_1_1__chianti__tools_a680536bf77d8f8baa8434c1bf84350a4}{01523} \textcolor{keyword}{def }\hyperlink{namespacepyneb_1_1utils_1_1__chianti__tools_a680536bf77d8f8baa8434c1bf84350a4}{listFiles}(path):
01524     \textcolor{stringliteral}{'''}
01525 \textcolor{stringliteral}{    walks the path and subdirectories to return a list of files}
01526 \textcolor{stringliteral}{    '''}
01527     alist=os.walk(path)
01528     print(\textcolor{stringliteral}{' getting file list'})
01529     listname=[]
01530     \textcolor{keywordflow}{for} (dirpath,dirnames,filenames) \textcolor{keywordflow}{in} alist:
01531         \textcolor{keywordflow}{if} len(dirnames) == 0:
01532             \textcolor{keywordflow}{for} f \textcolor{keywordflow}{in} filenames:
01533                 file=os.path.join(dirpath,f)
01534                 \textcolor{keywordflow}{if} os.path.isfile(file):
01535                     listname.append(file)
01536         \textcolor{keywordflow}{else}:
01537             \textcolor{keywordflow}{for} f \textcolor{keywordflow}{in} filenames:
01538                 file=os.path.join(dirpath,f)
01539                 \textcolor{keywordflow}{if} os.path.isfile(file):
01540                     listname.append(file)
01541     \textcolor{keywordflow}{return} listname
01542     \textcolor{comment}{#}
01543     \textcolor{comment}{# ----------------------------------------------------------------------------------------}
01544     \textcolor{comment}{#}
\hypertarget{__chianti__tools_8py_source_l01545}{}\hyperlink{namespacepyneb_1_1utils_1_1__chianti__tools_a63235035cef376c49a1f4e81933452a6}{01545} \textcolor{keyword}{def }\hyperlink{namespacepyneb_1_1utils_1_1__chianti__tools_a63235035cef376c49a1f4e81933452a6}{fblvlRead}(filename, verbose=False):
01546     \textcolor{stringliteral}{""" read a chianti energy level file and returns}
01547 \textcolor{stringliteral}{    \{"lvl":lvl,"conf":conf,"term":term,"spin":spin,"l":l,"spd":spd,"j":j}
01548 \textcolor{stringliteral}{    ,"mult":mult,"ecm":ecm,"eryd":eryd,"ref":ref\}"""}
01549 \textcolor{comment}{#        #  ,format='(i5,a20,2i5,a3,i5,2f20.3)'}
01550     fstring=\textcolor{stringliteral}{'i5,a20,2i5,a3,i5,2f20.3'}
01551     elvlcFormat=\hyperlink{classpyneb_1_1utils_1_1_fortran_format_1_1_fortran_format}{FortranFormat}(fstring)
01552     \textcolor{comment}{#}
01553     \textcolor{keywordflow}{if} os.path.exists(filename):
01554         input=open(filename,\textcolor{stringliteral}{'}\textcolor{stringliteral}{r')}
01555 \textcolor{stringliteral}{        s1=input.readlines()}
01556 \textcolor{stringliteral}{        input.close()}
01557 \textcolor{stringliteral}{        nlvls=0}
01558 \textcolor{stringliteral}{        ndata=2}
01559 \textcolor{stringliteral}{        }\textcolor{keywordflow}{while} ndata > 1:
01560             s1a=s1[nlvls][:-1]
01561             s2=s1a.split()
01562             ndata=len(s2)
01563             nlvls=nlvls+1
01564         nlvls-=1
01565         \textcolor{keywordflow}{if} verbose:
01566             print((\textcolor{stringliteral}{' nlvls = %5i'}%(nlvls)))
01567         lvl=[0]*nlvls
01568         conf=[0]*nlvls
01569         pqn=[0]*nlvls
01570         l=[0]*nlvls
01571         spd=[0]*nlvls
01572         mult=[0]*nlvls
01573         ecm=[0]*nlvls
01574         ecmth=[0]*nlvls
01575         \textcolor{keywordflow}{for} i \textcolor{keywordflow}{in} range(0,nlvls):
01576             \textcolor{keywordflow}{if} verbose:
01577                 print((s1[i]))
01578             inpt=\hyperlink{classpyneb_1_1utils_1_1_fortran_format_1_1_fortran_line}{FortranLine}(s1[i],elvlcFormat)
01579             lvl[i]=inpt[0]
01580             conf[i]=inpt[1].strip()
01581             pqn[i]=inpt[2]
01582             l[i]=inpt[3]
01583             spd[i]=inpt[4].strip()
01584             mult[i]=inpt[5]
01585             \textcolor{keywordflow}{if} inpt[6] == 0.:
01586                 ecm[i]=inpt[7]
01587             \textcolor{keywordflow}{else}:
01588                 ecm[i]=inpt[6]
01589                 ecmth[i]=inpt[7]
01590         ref=[]
01591         \textcolor{keywordflow}{for} i \textcolor{keywordflow}{in} range(nlvls+1,len(s1)-1):
01592             s1a=s1[i][:-1]
01593             ref.append(s1a.strip())
01594         \textcolor{keywordflow}{return} \{\textcolor{stringliteral}{"lvl"}:lvl,\textcolor{stringliteral}{"conf"}:conf,\textcolor{stringliteral}{'pqn'}:pqn,\textcolor{stringliteral}{"l"}:l,\textcolor{stringliteral}{"spd"}:spd,\textcolor{stringliteral}{"mult"}:mult,
01595             \textcolor{stringliteral}{"ecm"}:ecm,\textcolor{stringliteral}{'ecmth'}:ecmth, \textcolor{stringliteral}{'ref'}:ref\}
01596     \textcolor{keywordflow}{else}:
01597         \textcolor{keywordflow}{return} \{\textcolor{stringliteral}{'errorMessage'}:\textcolor{stringliteral}{' fblvl file does not exist'}\}
01598     \textcolor{comment}{#}
01599     \textcolor{comment}{# -----------------------------------------------------------------}
01600     \textcolor{comment}{#}
\hypertarget{__chianti__tools_8py_source_l01601}{}\hyperlink{namespacepyneb_1_1utils_1_1__chianti__tools_addacad8fe4cf3dff7f457b8b4a58054f}{01601} \textcolor{keyword}{def }\hyperlink{namespacepyneb_1_1utils_1_1__chianti__tools_addacad8fe4cf3dff7f457b8b4a58054f}{vernerRead}():
01602     \textcolor{stringliteral}{'''Reads the Verner & Yakovlev (A&AS 109, 125, 1995) photoionization cross-section data'''}
01603     xuvtop = os.environ[\textcolor{stringliteral}{'XUVTOP'}]
01604     fname = os.path.join(xuvtop, \textcolor{stringliteral}{'continuum'}, \textcolor{stringliteral}{'verner\_short.txt'})
01605     input = open(fname)
01606     lines = input.readlines()
01607     input.close()
01608     \textcolor{comment}{#}
01609     nlines=465
01610     maxZ = 30+1
01611     maxNel = 30 +1\textcolor{comment}{# also equal max(stage)}
01612     \textcolor{comment}{#}
01613     \textcolor{comment}{#z = np.array(nlines,'int32')}
01614     \textcolor{comment}{#nel = np.array(nlines,'int32')}
01615     pqn = np.zeros((maxZ,maxNel),\textcolor{stringliteral}{'int32'})
01616     l = np.zeros((maxZ,maxNel),\textcolor{stringliteral}{'int32'})
01617     eth = np.zeros((maxZ,maxNel),\textcolor{stringliteral}{'float64'})
01618     e0 = np.zeros((maxZ,maxNel),\textcolor{stringliteral}{'float64'})
01619     sig0 = np.zeros((maxZ,maxNel),\textcolor{stringliteral}{'float64'})
01620     ya = np.zeros((maxZ,maxNel),\textcolor{stringliteral}{'float64'})
01621     p = np.zeros((maxZ,maxNel),\textcolor{stringliteral}{'float64'})
01622     yw = np.zeros((maxZ,maxNel),\textcolor{stringliteral}{'float64'})
01623     \textcolor{comment}{#}
01624     fstring=\textcolor{stringliteral}{'i2,i3,i2,i2,6f11.3'}
01625     vernerFormat=\hyperlink{classpyneb_1_1utils_1_1_fortran_format_1_1_fortran_format}{FortranFormat}(fstring)
01626     \textcolor{comment}{#}
01627     \textcolor{keywordflow}{for} iline \textcolor{keywordflow}{in} range(nlines):
01628         out=\hyperlink{classpyneb_1_1utils_1_1_fortran_format_1_1_fortran_line}{FortranLine}(lines[iline],vernerFormat)
01629         z = out[0]
01630         nel = out[1]
01631         stage = z - nel + 1
01632         pqn[z,stage] = out[2]
01633         l[z,stage] = out[3]
01634         eth[z,stage] = out[4]
01635         e0[z,stage] = out[5]
01636         sig0[z,stage] = out[6]
01637         ya[z,stage] = out[7]
01638         p[z,stage] = out[8]
01639         yw[z,stage] = out[9]
01640     \textcolor{comment}{#}
01641     \textcolor{keywordflow}{return} \{\textcolor{stringliteral}{'pqn'}:pqn, \textcolor{stringliteral}{'l'}:l, \textcolor{stringliteral}{'eth'}:eth, \textcolor{stringliteral}{'e0'}:e0, \textcolor{stringliteral}{'sig0'}:sig0, \textcolor{stringliteral}{'ya'}:ya, \textcolor{stringliteral}{'p'}:p, \textcolor{stringliteral}{'yw'}:yw\}
01642     \textcolor{comment}{#}
01643     \textcolor{comment}{#-----------------------------------------------------------}
01644     \textcolor{comment}{#}
\hypertarget{__chianti__tools_8py_source_l01645}{}\hyperlink{namespacepyneb_1_1utils_1_1__chianti__tools_ae03bdc8f81142c34a58f9567f3359322}{01645} \textcolor{keyword}{def }\hyperlink{namespacepyneb_1_1utils_1_1__chianti__tools_ae03bdc8f81142c34a58f9567f3359322}{twophotonHRead}():
01646     \textcolor{stringliteral}{''' to read the two-photon A values and distribution function for the H seq'''}
01647     xuvtop = os.environ[\textcolor{stringliteral}{'XUVTOP'}]
01648     fName = os.path.join(xuvtop, \textcolor{stringliteral}{'continuum'}, \textcolor{stringliteral}{'hseq\_2photon.dat'})
01649     dFile = open(fName, \textcolor{stringliteral}{'}\textcolor{stringliteral}{r')}
01650 \textcolor{stringliteral}{    a = dFile.readline()}
01651 \textcolor{stringliteral}{    y0 = np.asarray(a.split())}
01652 \textcolor{stringliteral}{    a = dFile.readline()}
01653 \textcolor{stringliteral}{    z0 = np.asarray(a.split())}
01654 \textcolor{stringliteral}{    nz = 30}
01655 \textcolor{stringliteral}{    avalue = np.zeros(nz, }\textcolor{stringliteral}{'float64'})
01656     asum = np.zeros(nz, \textcolor{stringliteral}{'float64'})
01657     psi0 = np.zeros((nz, 17), \textcolor{stringliteral}{'float64'})
01658     \textcolor{keywordflow}{for} iz \textcolor{keywordflow}{in} range(nz):
01659         a=dFile.readline().split()
01660         avalue[iz] = float(a[1])
01661         asum[iz] = float(a[2])
01662         psi = np.asarray(a[3:])
01663         psi0[iz] = psi
01664     dFile.close()
01665     \textcolor{keywordflow}{return} \{\textcolor{stringliteral}{'y0'}:y0, \textcolor{stringliteral}{'z0'}:z0, \textcolor{stringliteral}{'avalue'}:avalue, \textcolor{stringliteral}{'asum'}:asum, \textcolor{stringliteral}{'psi0'}:psi0.reshape(30, 17)\}
01666     \textcolor{comment}{#}
01667     \textcolor{comment}{#-----------------------------------------------------------}
01668     \textcolor{comment}{#}
\hypertarget{__chianti__tools_8py_source_l01669}{}\hyperlink{namespacepyneb_1_1utils_1_1__chianti__tools_a79401d478ca07c45f4546a82634e6eb9}{01669} \textcolor{keyword}{def }\hyperlink{namespacepyneb_1_1utils_1_1__chianti__tools_a79401d478ca07c45f4546a82634e6eb9}{twophotonHeRead}():
01670     \textcolor{stringliteral}{''' to read the two-photon A values and distribution function for the He seq'''}
01671     xuvtop = os.environ[\textcolor{stringliteral}{'XUVTOP'}]
01672     fName = os.path.join(xuvtop, \textcolor{stringliteral}{'continuum'}, \textcolor{stringliteral}{'heseq\_2photon.dat'})
01673     dFile = open(fName, \textcolor{stringliteral}{'}\textcolor{stringliteral}{r')}
01674 \textcolor{stringliteral}{    a = dFile.readline()}
01675 \textcolor{stringliteral}{    y0 = np.asarray(a.split())}
01676 \textcolor{stringliteral}{    nz = 30}
01677 \textcolor{stringliteral}{    avalue = np.zeros(nz, }\textcolor{stringliteral}{'float64'})
01678     psi0 = np.zeros((nz, 41), \textcolor{stringliteral}{'float64'})
01679     \textcolor{keywordflow}{for} iz \textcolor{keywordflow}{in} range(1, nz):
01680         a=dFile.readline().split()
01681         avalue[iz] = float(a[1])
01682         psi = np.asarray(a[2:])
01683         psi0[iz] = psi
01684     dFile.close()
01685     \textcolor{keywordflow}{return} \{\textcolor{stringliteral}{'y0'}:y0, \textcolor{stringliteral}{'avalue'}:avalue, \textcolor{stringliteral}{'psi0'}:psi0.reshape(30, 41)\}
01686     \textcolor{comment}{#}
01687     \textcolor{comment}{#-----------------------------------------------------------}
01688     \textcolor{comment}{#}
\hypertarget{__chianti__tools_8py_source_l01689}{}\hyperlink{namespacepyneb_1_1utils_1_1__chianti__tools_a4fab6d9973653d9ca82088006f2a12fe}{01689} \textcolor{keyword}{def }\hyperlink{namespacepyneb_1_1utils_1_1__chianti__tools_a4fab6d9973653d9ca82088006f2a12fe}{upsdatRead}(upsdatFileName):
01690     \textcolor{stringliteral}{'''}
01691 \textcolor{stringliteral}{    to read the standard Chianti upsdat file}
01692 \textcolor{stringliteral}{    '''}
01693     inpt = open(upsdatFileName)
01694     lines = inpt.readlines()
01695     inpt.close()
01696     nTemp = int(lines[0].strip(\textcolor{stringliteral}{'\(\backslash\)n'}))
01697     minusOne = 0
01698     counter = 1
01699     ll = lines[1].split()
01700     temp = np.asarray(ll[3:], \textcolor{stringliteral}{'float64'})
01701     \textcolor{keywordflow}{while} \textcolor{keywordflow}{not} minusOne:
01702         \textcolor{keywordflow}{if} \textcolor{stringliteral}{'-1'} \textcolor{keywordflow}{in} lines[counter][:4]:
01703             minusOne = 1
01704         \textcolor{keywordflow}{else}:
01705             counter += 1
01706     ntrans = (counter)/2
01707     lvl1 = []
01708     lvl2 = []
01709     de = []
01710     gf = []
01711     upsilon = []
01712     counter = 1
01713     \textcolor{keywordflow}{for} itrans \textcolor{keywordflow}{in} range(ntrans):
01714         ll1 = lines[counter].split()
01715         lvl1.append(int(ll1[0]))
01716         lvl2.append(int(ll1[1]))
01717         de.append(float(ll1[2]))
01718         ll2 = lines[counter+1].split()
01719 \textcolor{comment}{#        print ' ll2 = ', ll2}
01720 \textcolor{comment}{#        print ' gf = ', ll2[2]}
01721         gf.append(float(ll2[2]))
01722         upsilon.append(np.asarray(ll2[3:], \textcolor{stringliteral}{'float64'}))
01723         counter += 2
01724     counter += 1
01725     ref = []
01726     \textcolor{keywordflow}{for} aline \textcolor{keywordflow}{in} lines[counter:-1]:
01727         ref.append(aline.strip(\textcolor{stringliteral}{'\(\backslash\)n'}))
01728     \textcolor{keywordflow}{return} \{\textcolor{stringliteral}{'lvl1'}:lvl1, \textcolor{stringliteral}{'lvl2'}:lvl2, \textcolor{stringliteral}{'de'}:de, \textcolor{stringliteral}{'gf'}:gf, \textcolor{stringliteral}{'upsilon'}:upsilon, \textcolor{stringliteral}{'temperature'}:temp, \textcolor{stringliteral}{'ref'}:ref\}
01729     \textcolor{comment}{#}
01730     \textcolor{comment}{# -----------------------------------------------------}
01731     \textcolor{comment}{#}
\hypertarget{__chianti__tools_8py_source_l01732}{}\hyperlink{namespacepyneb_1_1utils_1_1__chianti__tools_a349618b1b79ca902de8a030592d6d43a}{01732} \textcolor{keyword}{def }\hyperlink{namespacepyneb_1_1utils_1_1__chianti__tools_a349618b1b79ca902de8a030592d6d43a}{versionRead}():
01733     \textcolor{stringliteral}{""" read the version number of the CHIANTI database"""}
01734     xuvtop = os.environ[\textcolor{stringliteral}{'XUVTOP'}]
01735     vFileName = os.path.join(xuvtop, \textcolor{stringliteral}{'VERSION'})
01736     vFile = open(vFileName)
01737     versionStr = vFile.readline()
01738     vFile.close()
01739     \textcolor{keywordflow}{return} versionStr.strip()
01740     \textcolor{comment}{#}
01741     \textcolor{comment}{#-----------------------------------------------------------}
01742     \textcolor{comment}{#}
\hypertarget{__chianti__tools_8py_source_l01743}{}\hyperlink{namespacepyneb_1_1utils_1_1__chianti__tools_a71b9295157832135424ea7dc0138fcd4}{01743} \textcolor{keyword}{def }\hyperlink{namespacepyneb_1_1utils_1_1__chianti__tools_a71b9295157832135424ea7dc0138fcd4}{scale\_bti}(evin,crossin,f,ev1):
01744     \textcolor{stringliteral}{"""}
01745 \textcolor{stringliteral}{    apply BT ionization scaling to (energy, cross-section)}
01746 \textcolor{stringliteral}{    returns [bte,btx]}
01747 \textcolor{stringliteral}{    """}
01748     u=evin/ev1
01749     bte=1.-np.log(f)/np.log(u-1.+f)
01750     btx=u*crossin*(ev1**2)/(np.log(u)+1.)
01751     \textcolor{keywordflow}{return} [bte,btx]
01752     \textcolor{comment}{#}
01753     \textcolor{comment}{#-----------------------------------------------------------}
01754     \textcolor{comment}{#}
\hypertarget{__chianti__tools_8py_source_l01755}{}\hyperlink{namespacepyneb_1_1utils_1_1__chianti__tools_a47075ba90f01cbd7a3dbd08115544214}{01755} \textcolor{keyword}{def }\hyperlink{namespacepyneb_1_1utils_1_1__chianti__tools_a47075ba90f01cbd7a3dbd08115544214}{descale\_bti}(bte,btx,f,ev1):
01756     \textcolor{stringliteral}{"""}
01757 \textcolor{stringliteral}{    descale BT ionization scaling}
01758 \textcolor{stringliteral}{    returns [energy,cross-section]}
01759 \textcolor{stringliteral}{    """}
01760     u=1.-f+np.exp(np.log(f)/(1.-bte))
01761     energy=u*ev1
01762     cross=(np.log(u)+1.)*btx/(u*ev1**2)
01763     \textcolor{keywordflow}{return} [energy,cross]
01764     \textcolor{comment}{#}
01765     \textcolor{comment}{#-----------------------------------------------------------}
01766     \textcolor{comment}{#}
\hypertarget{__chianti__tools_8py_source_l01767}{}\hyperlink{namespacepyneb_1_1utils_1_1__chianti__tools_ad34ed06dfc613c2f1ace88fab4ac4392}{01767} \textcolor{keyword}{def }\hyperlink{namespacepyneb_1_1utils_1_1__chianti__tools_ad34ed06dfc613c2f1ace88fab4ac4392}{descale\_bt}(bte,btomega,f,ev1):
01768     \textcolor{stringliteral}{"""}
01769 \textcolor{stringliteral}{    descale BT excitation scaling}
01770 \textcolor{stringliteral}{    returns [energy,collision strength]}
01771 \textcolor{stringliteral}{    """}
01772     u=1.-f+np.exp(np.log(f)/(1.-bte))
01773     energy=u*ev1
01774     omega=(np.log(u)-1.+np.exp(1.))*btomega
01775     \textcolor{keywordflow}{return} [energy,omega]
01776     \textcolor{comment}{#}
01777     \textcolor{comment}{#-----------------------------------------------------------}
01778     \textcolor{comment}{#}
\hypertarget{__chianti__tools_8py_source_l01779}{}\hyperlink{namespacepyneb_1_1utils_1_1__chianti__tools_abdf42d0f9176edb418859e7a12fa4bce}{01779} \textcolor{keyword}{def }\hyperlink{namespacepyneb_1_1utils_1_1__chianti__tools_abdf42d0f9176edb418859e7a12fa4bce}{scale\_bt}(evin,omega,f,ev1):
01780     \textcolor{stringliteral}{"""}
01781 \textcolor{stringliteral}{    apply BT excitation scaling to (energy, collision strength)}
01782 \textcolor{stringliteral}{    returns [bte,btomega]}
01783 \textcolor{stringliteral}{    """}
01784     u=evin/ev1
01785     bte=1.-np.log(f)/np.log(u-1.+f)
01786     btomega=omega/(np.log(u)-1.+np.exp(1.))
01787     \textcolor{keywordflow}{return} [bte,btomega]
\end{DoxyCode}
